\documentclass{letter}
%\usepackage{fancyhdr}

\oddsidemargin=.2in
\evensidemargin=.2in
\textwidth=5.9in
\topmargin=-.5in
\textheight=9in


% \pagestyle{fancy}
% \fancyhf{}
% \rhead{Ryan Giordano}
% \lhead{Research Statement}
% \rfoot{Page \thepage}

\usepackage{etoolbox}

\usepackage{microtype}
\usepackage{graphicx}
%\usepackage{subfigure}
\usepackage{booktabs} % for professional tables
\usepackage{xcolor}
\usepackage[hidelinks=True]{hyperref}
\usepackage{xargs}[2008/03/08]

% Documentation
% http://ftp.math.purdue.edu/mirrors/ctan.org/macros/latex/contrib/refstyle/refstyle.pdf
\usepackage{refstyle}
\usepackage{varioref} % Use refstyle instead of varioref directly.

\usepackage{amsmath}
\usepackage{amssymb}
\usepackage{amsfonts}
\usepackage{amsthm}
\usepackage{mathrsfs} % For mathscr
\usepackage{mathtools}

%\usepackage[authoryear]{natbib}
%\bibliographystyle{apalike}


\begin{document}
\begin{letter}{2021 Alvarez-Hopper Selection Committee\\
Lawrence Berkeley National Laboratory\\
Berkeley, CA, 94720}


\opening{Dear Alvarez-Hopper Selection Committee,}

I am writing to enthusiastically apply for the 2021 Alvarez-Hopper postdoctoral
fellowship.  Through my work as an engineer at Google, my PhD in statistics at
Berkeley, and my current postdoctoral position at the Massachusetts Institute of
Technology, my academic ambition has consistently been to bring conceptually
simple, theoretically sound, and computationally tractable tools to bear on the
large scale data science problems of the twenty-first century.  The US
Department of Energy (DOE), and the Berkeley Lab in particular, are currently
aggressively pursuing new frontiers in data science, as expressed, for example,
by the Advanced Scientific Computing Advisory Committee Subcommittee on AI/ML
report of September 2020.  The Lab's expressed need for scalable and reliable
uncertainty quantification and model interogation are a perfect fit for my
research, and I am confident that close collaboration between Berkeley
Lab scientists and myself would produce both valuable science and
methodological advances in data science.

My work on large-scale data problems, both at Google and on the Celeste project
(producing astronomical catalogs on the NERSC supercomputing cluster), have
acquanited me with the difficulty of applying traditional statistical ideas at
modern scales.  My research has been devoted to easing these difficulties.
My PhD began with an effort to overcome the fact that classical Bayesian Markov
Chain Monte Carlo (MCMC) techniques would not scale to the problems I
encountered at Google. This research question led me to develop the linear
response correction for variational Bayes approximations, a perturbative
correction which provides reliable, approximate Bayesian inference on problems
far too massive for MCMC (like the construction of astronomical catalogs).
During the second half of my PhD, I found that similar perturbative ideas
provide scalable solutions to a wide array of practical machine learning
problems involving uncertainty quantification or model checking, including cross
validation, prior sensitivity, adversarial data sensitivity, and sensitivity to
prior or model specification.

Good ideas are made better by being put into practice, and I believe that the my
research will be best developed and advanced in the context of massive,
practically relevant scientific applications.  In addition to revisiting the
production of astronomical catalogs in light of my variational Bayes work, both
the High Energy Physics and Nuclear Physics groups of Berkeley Lab have
expressed explicit need for uncertainty quantification as part of the ``AI for
Science Initiative'' in the above-mentioned report.  As the report states,
``While  AI algorithms have proliferated,  general  techniques  for
understanding  their accuracy  and stability in complex environments have not.''
I believe that collaboration between methodologists such as myself and
scientists is the best way to fill this need, and it is for this reason
that I am applying to become an Alvarez-Hopper fellow.


\closing{Sincerely,}

Ryan Giordano\\
Department of Computer Science\\
Massachusetts Institute of Technology\\
Cambridge, MA\\
(805) 501-6754 (cell)\\
rgiordan@mit.edu\\
\url{https://rgiordan.github.io/}
\end{letter}

% \bibliography{references}
% \bibliographystyle{plainnat}

\end{document}
