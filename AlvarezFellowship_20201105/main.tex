\documentclass{article}
\usepackage{fancyhdr}


\pagestyle{fancy}
\fancyhf{}
\rhead{Ryan Giordano}
\lhead{Research Statement}
\rfoot{Page \thepage}

\usepackage{tabularx}

% \usepackage{fancyhdr}
% \pagestyle{fancy}

% \fancyfoot{}
% \fancyfoot[C]{Rough draft---do not distribute}

\usepackage{etoolbox}


\usepackage{microtype}
\usepackage{graphicx}
\usepackage{subfigure}
\usepackage{booktabs} % for professional tables
\usepackage{xcolor}
\usepackage[hidelinks=True]{hyperref}
\usepackage{xargs}[2008/03/08]

% Documentation
% http://ftp.math.purdue.edu/mirrors/ctan.org/macros/latex/contrib/refstyle/refstyle.pdf
\usepackage{refstyle}
\usepackage{varioref} % Use refstyle instead of varioref directly.

\usepackage{amsmath}
\usepackage{amssymb}
\usepackage{amsfonts}
\usepackage{amsthm}
\usepackage{mathrsfs} % For mathscr
\usepackage{mathtools}

\usepackage[authoryear]{natbib}
\bibliographystyle{apalike}

\usepackage{geometry}
\geometry{margin=1.5in}


\title{Research Statement}

\author{
  Ryan Giordano \\ \texttt{rgiordan@mit.edu }
}

\begin{document}


\subsection*{Overview}

Deriving scientific information from large, complex datasets can motivate large,
complex statistical models; below we will give examples from our work on
problems in astronomy, genomics, phylogenetics, econometrics, internet
advertising, and ecology.  As models grow in complexity, the need to interrogate
their assumptions, to propagate uncertainty amongst their components, and to
perform non-parametric checks on their data sensitivity grows commensurately.
However, so does the computational cost of doing so using traditional
statistical methods. Many classical procedures designed to address these
concerns, such as Markov Chain Monte Carlo (MCMC), cross validation (CV), or
re-estimating a model under a range of modeling assumptions, can be
prohibitively expensive in many modern problems.

To help fill this gap, my research focuses on applications of {\em sensitivity
analysis}, applied not merely in the traditional sense of assessing sensitivity
to imprecise modeling assumptions (though I do pursue this traditional role as
well), but also to assess freqeuntist sampling properties and propagate
uncertainty in Bayesian procedures.  At its core, my methodological work is all
based on using Taylor series approximations, constructed only from properties of
a single model fit, computed using either optimization or MCMC, to extrapolate
to alternatives.  In this way, I provide approximations to Bayesian posterior
uncertainty, CV loss, and model sensitivity, while avoiding expensive
re-estimation.  Sensitivity analysis is a venerable idea with a rich existing
literature, and I show that it has wide-ranging and fundamental applications in
modern, computationally intensive statistical problems.  Furthermore, I motivate
the re-examination of theoretical ideas concerning the role of differential
approximations in statistical analysis as practical tools in the age of big data
and big computing.

I will divide my research statement into three parts: first, I will cover a more
traditional application of sensitivity analysis to the assessment of prior
sensitivity in Bayesian and variational Bayesian analysis.  I will then discuss
approximate CV and bootstrapping, which can be viewed as a kind of sensitivity
analysis.  Finally, I discuss how sensitivity analysis can recover accurate
posterior covariances from variational Bayesian approximations, tying sensitiity
analysis to the uniquitous scientific goal of uncertainty propagation.


%%%%%%%%%%%%%%%%%%%%%%%%%%%%%%%%%%%%%%%%%%%%%%%%%%%%%%%%%%%%%%%%%%%%%%%%%%%%%%%
%%%%%%%%%%%%%%%%%%%%%%%%%%%%%%%%%%%%%%%%%%%%%%%%%%%%%%%%%%%%%%%%%%%%%%%%%%%%%%%
%%%%%%%%%%%%%%%%%%%%%%%%%%%%%%%%%%%%%%%%%%%%%%%%%%%%%%%%%%%%%%%%%%%%%%%%%%%%%%%
\subsection*{Prior sensitivity in Bayesian analysis}

Bayesian techniques allows analysts to reason coherently about unknown
parameters, but only if the user specifies a complete generating process for the
parameters and data, including both prior distributions for the parameters and
precise likelihoods for the data.  Often, aspects of this model are at best a
considered simplifcation, and at worst chosen only for computational
convenience.  It is critical to ask whether the analysis would have changed
substantively had different modeling choices been made.

\paragraph{Bayesian nonparametrics.}

A commonly question in unsupervised clustering is how many distinct clusters are
present in a dataset.  Discrete Bayesian nonparametrics (BNP) allows the answer
to be inferred using Bayesian inference, but one must specify a prior on how
distinct clusters are generated \citep{ghosh:2003:bnp,
gershman:2012:bnptutorial}.  A particularly common modeling choice is the
stick-breaking representation of a Dirichlet process prior
\citep{sethuraman:1994:constructivedp}, a mathematical abstraction which is
arguably better justified by its computational convenience than its realism. Our
workshop paper, \citet{giordano:2018:bnpsensitivity}, fits a BNP model with
variational Bayes \citep{blei:2006:dirichletbnp} using the standard,
computationally convenient stick-breaking prior, but then uses sensitivity
analysis to allow the user to explore alternative functional forms an order of
magnitude faster than would be possible with refitting. In work currently in
progress, we apply our method to a human genome dataset in phylogenetics taken
from \citep{huang:2011:haplotype}, and find that our method accurately discovers
real excess prior sensitivity in a BNP version of the model
\texttt{fastSTRUCTURE} \citep{raj:2014:faststructure}.



\paragraph{Partial pooling in meta-analysis.}

A popular form of meta-analysis in econometrics is to place a hierarchial model
on a set of related experimental results, which both ``shrinks'' the individual
estimates towards a common mean, potentially decreasing mean squared error, and
allowing direct estimation of the average effect and diversity of effects
\citep{rubin:1981:estimation,gelman:1992:inference}. These advantages come at
the cost of positing a precise generative process for the effects in question,
and it is reasonable to interrogate whether the estimation procedure is robust
to varaibility in these effects.  In \citet{giordano:2016:microcredit}, we apply
sensitivity analysis to a published meta-analysis of the effectiveness of
microcredit interventions in seven developing countries
\citep{meager:2019:microcredit}.  We find that the conclusion are highly
sensitive to the assumed covariance structure between the base level of business
profitability and the microcredit effect, a covariance which is {\em a priori}
difficult to ascertain, automatically diagnosing an important source of
epistemic uncertainty not captured by the Bayesian posterior.


\paragraph{Hyperparameter sensitivity for MCMC.}

MCMC is arguably the most commonly used computational tool to estimate Bayesian
posteriors, and modern black-box MCMC tools such as \texttt{Stan} \citep{rstan,
carpenter:2017:stan}.  However, MCMC still often takes a long time to run, and
systematically exploring alternative prior parameterizations by re-running MCMC
would be computationally prohibitive for all but the simplest models. A
classical result from Bayesian robustness states that the sensitivity of a
posterior expectation is given by a particular posterior covariance
\citep{gustafson:1996:localposterior, basu:1996:local}, though the result has
not been widely used, arguably due in part to the lack of an automatic
implementation. In my software package, \citet{giordano:2020:rstansensitivity},
I take advantage of the automatic differentiation capacities of
\texttt{Stan} to provide automatic hyperparameter sensitivity for
generic Stan models.  In examples in the package \texttt{git} repository,
I demonstrate the efficacy of the package in detecting excess prior
sensitivity, particularly in a social sciences model taken from
\citet[Chapter 13.5]{gelman:2006:arm}.


%%%%%%%%%%%%%%%%%%%%%%%%%%%%%%%%%%%%%%%%%%%%%%%%%%%%%%%%%%%%%%%%%%%%%%%%%%%%%%%
%%%%%%%%%%%%%%%%%%%%%%%%%%%%%%%%%%%%%%%%%%%%%%%%%%%%%%%%%%%%%%%%%%%%%%%%%%%%%%%
%%%%%%%%%%%%%%%%%%%%%%%%%%%%%%%%%%%%%%%%%%%%%%%%%%%%%%%%%%%%%%%%%%%%%%%%%%%%%%%
\subsection*{Data sensitivity: cross validation and frequentist variance}

Frequentist variability is ultimately concerned with the outcome of an
estimation procedure if the data were drawn from the same distribution as but
different from that observed.  Similarly, all forms of cross-validation (CV)
evaluates a statistic if parts of the observed data had been ablated.  Both of
these procedures can be treated by sensitivity analysis, where sensitivity
is to the dataset itself.


\paragraph{Accuracy bounds for approximate cross validation.}

Our paper: \citet{giordano:2019:ij}
See IJ section for more potential references.
We observed that these differnetial methods could speed up the evaluation of
cross validation in large machine learning models which are expensive to re-fit.
In CITE, we bridged the gap from some of the classical literature, providing
finite-sample accuracy bounds for approximate leave-k-out cross validation, even
when the derivatives of the objective function are unbounded.

A follow-on work in progress (CITE) expands the results to higher-order
expansions and to larger perturbations, including k-fold CV and the bootstrap.
They key to all this work is a set complexity condition, in light of which it is
clear that one can provide accuracy bounds uniformly over small perturbations,
and over randomly-chosen large perturbations, but not for uniform bounds over
large perturbations.  Because we show that the linear approximation approaches
the bootstrap closer than the bootstrap approaches the truth, our work
should allow for practical differential approximations to prohibitively
expensive procedures such as the bootstrap-after-the-bootstrap.
\citet{giordano2019:hoij}
\citet{efron:1994:bootstrap}
\citet{hall:2013:bootstrap}


\paragraph{Adversarial sensitivity for M-estimators.}

In the social sciences, it is common to run randomized trials on a particular
population to esitmate an effect that we hope generalizes to different
populations.  Classical standard errors measure the variability that one
would expect from random sampling from the same distribution as that that
gave rise to the observed data.  However, in order to generalize to radically
different contexts, one might expect the conclusions to be robust to more
adversarial perturbations.  In CITE, co-authors an I ask whether some
common econometrics analyses are robust to the removal of a small proportion
(e.g. one tenth of one percent of the data).  Again using the empirical
influence function, we provide an automated method and software package
to answer this question.

We found that some common ``gold-standard'' applied econometrics papers
can have central claims reversed by removing only a very small number of
data points, even though there is no clear evidence of misspecification.
A key theoretical takeaway of our work is that datsets with a low signal
to noise ratio (defined as the effect size over the estimator standard
deviation times $\sqrt{N}$) will always be sensitive to adversarial removal
of a very small number of datapoints.  Though the precise meaning will
depend on the context, the broad implication is that studies which
attempt to overcome low SNR with large sample sizes will be inherently
non-robust, even in the absence of misspecification or gross errors.



\paragraph{Frequentist properties of Bayesian posteriors}

By combining the above approach to frequentist variance with the MCMC-based
measures of sensitivity, we are able to derive the Bayesian infinitesimal
jackknife (IJ), which can be used to compute the frequentist variability
of Bayeisan posterior means without bootstrapping or computing a maximum
a-posteriori (MAP) estimate.  Such frequentist variances are particularly
important when there is a possibility of model misspecification (in which
case the Bayesian posterior variance is not particularly meaningful),
or when the data comes from a random sample, the variability of which
is meaningful in its own right.  In a work in progress, we prove the consistency
of the Bayesian IJ and show its accuracy as an approximation to the bootstrap
for a larger number of examples.
\citet{huggins:2019:bayesbag}
\citet{waddell:2002:bayesphyloboot}
\citet{kleijn:2006:misspecification}
\citet{kass:1990:posteriorexpansions} % Same expansion, stricter conditions
\citet[Chapter 6]{lehman:1983:pointestimation}


%%%%%%%%%%%%%%%%%%%%%%%%%%%%%%%%%%%%%%%%%%%%%%%%%%%%%%%%%%%%%%%%%%%%%%%%%%%%%%%
%%%%%%%%%%%%%%%%%%%%%%%%%%%%%%%%%%%%%%%%%%%%%%%%%%%%%%%%%%%%%%%%%%%%%%%%%%%%%%%
%%%%%%%%%%%%%%%%%%%%%%%%%%%%%%%%%%%%%%%%%%%%%%%%%%%%%%%%%%%%%%%%%%%%%%%%%%%%%%%
\subsection*{Propagation of uncertainty in scalable Bayesian inference}

Complex scientific inference procedures, such as the creation of astronomical
catalogues, often exhibit uncertainty in many aspects of the model.  For
instance, in order to infer whether a handful of pixels on a telescopic image is
a dim star or a distant galaxy, one must know the distortion (aka the point
spread function) of the telescope, the lightness of the sky background, the
noise of the photoreceptors, and the identity of nearby celestial objects, all
of which quantities must themselves be inferred with some uncertainty.

Bayesian procedures coherently propagate uncertainty between all such model
quantities, but classical MCMC procedures do not scale well, and are far beyond
computational reach for astronomical catalogues.  Researchers often turn to
optimization-based mean field Variational Bayes (MFVB) procedures as a scalable
alterative to MCMC, but MFVB does not estimate posterior correlations, and is
known to underestimate marginal posterior uncertainties.
\footnote{The frequentist expectation-maximization, or EM, algorithm,
can be understood as a MFVB procedure, and the present criticism applies
to it as well.}

In CITE, I develop a method to recover accurate posterior uncertainties from
MFVB approximations without needing to fit a more complex model, or indeed to
re-fit the original model.  The idea is to exploit a duality between posterior
covariances the sensitivity of posterior means and use the sensitivity of the
MFVB approximation to infinitesimal perturbations as an estimator of the
posterior covariance.  We call the method ``linear response variational Bayes''
(LRVB) after the idea's progenitor as a method in statistical mechanics for
inferring microscopic intensive thermodynamic quantities from macroscopic
perturbations of extensive quantities.  Computing the LRVB covaraince requires
solving a linear system, which in scientific applications is often sparse and
can be solved using iterative techniques such as conjugate gradient.

We compare LRVB covariances to MCMC on a large number of real-world datasets,
including logistic regression on internet-scale data, the Cormack-Jolly-Seber
model from ecology, and hierarchical generalized linear models from the
social sciences, and demonstrated accurate posterior covariances computed
over an order of magnitude faster than MCMC.


\subsection*{Selected Future work}

There is a lot to be done simply applying the above methodology to
applied problems in conjuction with collaborators with domain expertise.
However, in this section, I will focus instead on new methodological directions
suggested by the above work.

\paragraph{The bootstrap and the bootstrap after the bootstrap.}

Our work on the higher-order infinitesimal jackknife could be applied to other
random reweighting schemes, particularly the bootstrap.  The bootstrap is known
to have frequentist properties that are asymptotically more accurate than the
normal approximation in certain circumstances CITE, but the bootstrap requires
re-computing an estimator as many time as there are bootstrap samples. However,
a sufficiently high-order IJ estimate will approach the bootstrap estimator at a
rate faster than the bootstrap's extra accuracy, strongly suggesting that the IJ
will inherit all of the bootstrap's attractive properties at a fraction of the
computational cost.  The IJ is particularly appealing for
bootstrap-after-bootstrap procedures, which have attractive theoretical
properties but are computationally prohibitive even on medium-sized statistical
problems.  It seems plausible that the HOIJ could open up a range of bootstrap
applications that are presently out of reach.


\paragraph{Bayesian model criticism.}

Essentially all tools for checking the accuracy of Bayesian models are
frequentist in nature --- e.g. checking whether the data is likely under
draws from the prior or posterior, or evaluating its predicive performance
on a held-out dataset.  Predictive model checks, such as leave-one-out
CV are attractive, but currently have few practical implementations that
avoid multiple runs of the MCMC algorithm.  However, the possibility of
forming higher-order expansions of the Bayesian posterior as a function of
the empirical distribtuion could change that.


\paragraph{Partitioned Bayesian inference with theoretical bounds.}

Given a large, complicated problem, it is often computationally convenient to
perform inference in separate computatoinal steps, while still propagating
uncertainty from one step to another.  For example, in CITE, we first fit spline
regressions to time series of gene expression data, and then clustered the
spline fits to group together genes with similar behavior.  The Bayesian ideal,
which is the simultaneous estimation of the clusters and spline regression, was
too computationally prohibitive, and would intuitively have given a similar
result to the sequential analysis, to the extent that the cluster center priors
did not shrink the spline regressions too much.


\paragraph{Incorporating LRVB corrections into MFVB approximations.}



\paragraph{Bootstrapping simulation-based inference.}




\bibliography{references}
\bibliographystyle{plainnat}

\end{document}
