\documentclass{article}
\usepackage{fancyhdr}


\pagestyle{fancy}
\fancyhf{}
\rhead{Ryan Giordano}
\lhead{Research Statement}
\rfoot{Page \thepage}

\usepackage{tabularx}

% \usepackage{fancyhdr}
% \pagestyle{fancy}

% \fancyfoot{}
% \fancyfoot[C]{Rough draft---do not distribute}

\usepackage{etoolbox}


\usepackage{microtype}
\usepackage{graphicx}
\usepackage{subfigure}
\usepackage{booktabs} % for professional tables
\usepackage{xcolor}
\usepackage[hidelinks=True]{hyperref}
\usepackage{xargs}[2008/03/08]

% Documentation
% http://ftp.math.purdue.edu/mirrors/ctan.org/macros/latex/contrib/refstyle/refstyle.pdf
\usepackage{refstyle}
\usepackage{varioref} % Use refstyle instead of varioref directly.

\usepackage{amsmath}
\usepackage{amssymb}
\usepackage{amsfonts}
\usepackage{amsthm}
\usepackage{mathrsfs} % For mathscr
\usepackage{mathtools}

\usepackage[authoryear]{natbib}
\bibliographystyle{apalike}

\usepackage{geometry}
\geometry{margin=1.5in}


\title{The Bayesian Infinitesimal Jackknife for Variance}

\author{
  Ryan Giordano \\ \texttt{rgiordan@mit.edu }
}

\begin{document}

My PhD began as an effort to understand and solve some of the practical problems
that arose during my four and a half years as an engineer at Google.  My initial
goal was to understand how to propagate uncertainty in large problems for which
classical Bayesian techniques (namely Markov Chain Monte Carlo) are
prohibitively expensive.  This problem eventually led to a more general
investigation of sensitivity analysis in both Bayesian and frequentist
statistics.  My work is theoretical at times, but I have always maintained the
goal of providing practical and easy-to-understand tools for scientists who are
trying to rigorously solve difficult data problems.  Having developed a suite of
practical tools and theoretical frameworks, I now believe that my research will
be best served by applying these ideas to practical problems in close
collaboration with practicing scientists.  For this reason, I am applying to
become an Alvarez Fellow.

\subsection*{An Example Problem from Astronomy.}

During the initial years of my PhD I collaborated on an LBNL project that can
provide a useful practical context for describing potential applications of my
PhD work.  Astronomers build catalogues of celestial objects from sky surveys
consisting of very large numbers of telescopic images that tile the night sky.
Images are typically taken in a multiple different color bands, may overlap, and
are blurred by an unknown ``point spread function'' (PSF). It would be desirable
to combine images from multiple surveys which may be taken at different
resolutions.  From the pixels in these images, astronomers wish to infer the
location of celestial objects as well as some key properties, such as color,
brightness, and, in the case of galaxies, shape and orientation.  One way to do
so is to posit a parameterized generative model, from which the unknown true
location and properties of celestial objects combine with random noise to
produce the observed images.  Given this generative model, one tries to infer
what the unknown true catalogue might have been using the images and Bayesian
statistical techniques.

In this problem, uncertainty and correlation abounds.  Consider two dim stars
which are close together on the scale of the pixelized image.  It can be
difficult to ascertain whether the image is of two distinct stars, or a single
oblong galaxy.  The relative probabilities of each depends on the unknown PSF,
about which there may also be some uncertainty.  One may wish to borrow
strength between multiple images, and take into account how the multiple
images' different resolutions affects the amount of information they give.
Ideally, we would take all these tangled uncertainties into account when stating
our final probabilistic belief about the identity of the object(s).

\subsection*{Linear Response Covariances.}

On small problems, one would typically quantify all this uncertainty with Markov
Chain Monte Carlo (MCMC).  However, with the vast amount of data contained in a
sky survey, MCMC is far too computationally expensive.  Researchers instead turn
to ``mean field variational Bayes'' (VB), an optimzation-based approximation to
the full Bayesian solution.\footnote{ The frequentist
``expectation-maximization'', or EM algorithm, can in fact be understood as a
MFVB algorithm, and is so included in my discussion here.} Howvever, the very
reason that MFVB is tractacble is because it assumes that there is no
correlation between disparate aspects of the model.  For example, for the
purpose of inferring the identity of a star, the PSF is treated as known and
fixed, not uncertain.  By making the problem computationally tractable, we have
to discard some of the correlations we wish to account for in the first place.
Indeed, MFVB is notorious for producing unusuably small estimates of posterior
uncertainty.

The first section of my thesis addresses this problem with a technique which we
call ``linear response variational Bayes'' (LRVB).  By considering the
sensitivity of the MFVB approximation to perturbations of the objective, one can
recover an approximation to the full Bayesian posterior.  We showed that, in a
wide set of typical models (taken from the Stan Examples datasets), LRVB allows
Bayesian posterior covariances to be accurately approximated orders of magnitude
faser than MCMC.

The LRVB covariance estimate has a crucial computational property common to all
of my PhD research, which is that the required sensitivity can be computed
without ever re-solving the initial optimization problem.  What is needed,
instead, is the solution to a system of linear equations involving the
derivatives of the MFVB objective function at the optimum.  In the case of the
sky survey, this means LRVB requires the solution of only one optimization
problem---the original MFVB fit.  Since most of the inferred parameters don't
affect one another (e.g., the value of PSF on one night does not affect the
classicificaiton of a star on a different night), the linear system is sparse.
Furthermore, though even storing the full covariance matrix would be
prohibitively expensive, individual covariance estimates can be queried using
iterative methods such as the conjugate gradient algorithm.  Finally, the
required derivatives, which would be tedious and error-prone to compute by hand,
can be quickly and faultlessly computed automatically by modern automatic
differentiation.

\subsection*{Cross Validation.}

Modern automatic differentiation and scalable, iterative linear solvers
allow the automatic computation of the sensitivity needed to compute the LRVB
covariance estimates of the previous section.  It turns out that many other
key statistical tasks can also be accurately approximated by linear sensitivty
analysis.  In this section,



\bibliography{references}
\bibliographystyle{plainnat}

\end{document}
