
%%%%%%%%%%%%%%%%%%%%%%%%%%%%%%%%%%%%%%%%%%%%%%%%%%%%%%%
%%%%%%%%%%%%%%%%%%%%%%%%%%%%%%%%%%%%%%%%%%%%%%%%%%%%%%%
%%%%%%%%%%%%%%%%%%%%%%%%%%%%%%%%%%%%%%%%%%%%%%%%%%%%%%%

\begin{frame}[t]{Covariate balance}



When $\muhat(\y)$ is a calibration estimator, this is the same as covariate balance
in expectation:
$$
\begin{aligned}
    \expect{}{\muhat(\ytil) - \muhat(\y) | \x} =
    % \meansur \w_s \left(\expect{}{\ytil | \x_p} - \expect{}{\y | \x_p}\right) =
    \delta \meansur \w_s f(\x_p) \overset{\textrm{check}}{=}
        \delta \meantar f(\x_p).
\end{aligned}
$$
%
But now all we need to do is compare $\muhat(\ytil) - \muhat(\y)$ for
``nearby'' $\ytil$ and $\y$.



\end{frame}


%%%%%%%%%%%%%%%%%%%%%%%%%%%%%%%%%%%%%%%%%%%%%%%%%%%%%%%
%%%%%%%%%%%%%%%%%%%%%%%%%%%%%%%%%%%%%%%%%%%%%%%%%%%%%%%
%%%%%%%%%%%%%%%%%%%%%%%%%%%%%%%%%%%%%%%%%%%%%%%%%%%%%%%




%%%%%%%%%%%%%%%%%%%%%%%%%%%%%%%%%%%%%%%%%%%%%%%%%%%%%%%
%%%%%%%%%%%%%%%%%%%%%%%%%%%%%%%%%%%%%%%%%%%%%%%%%%%%%%%
%%%%%%%%%%%%%%%%%%%%%%%%%%%%%%%%%%%%%%%%%%%%%%%%%%%%%%%

\begin{frame}[t]{MrPaw}

We need to approximate $\muhat_{\mrp}(\ytil) - \muhat_{\mrp}(\y)$.

\vspace{1em}
\hrulefill\\
\textbf{Step one: Define weights.}

Noting that $w_s = \frac{d}{d \y_s} \muhat_{\cal}$, we can define

$$
\wmrp_s := \frac{d}{d\y_s} \muhat_{\mrp}. %= \meantar \frac{d}{d\y_s} \yhat_p.
$$

It happens that the needed derivatives are given
by simple a posterior covariances involving only the inverse
link function $m(\x; \theta)$ and
log likelihood \parencite{giordano:2018:covariances}:

$$
\frac{d \yhat_p}{d\y_s}  =
    \cov{\p(\theta \vert \textrm{Survey data})}{
        m(\x_p; \theta),
        \frac{\partial}{\partial \y} \log p(\y \vert \theta, \x_s)}
$$

These can be computed using standard MCMC software \parencite{brms}.

No other weight definition will do --- in some cases,
MrP is exactly a calibration estimator (e.g. linear regression with flat priors),
and we want the definitions to coincide in that case.

\end{frame}



%%%%%%%%%%%%%%%%%%%%%%%%%%%%%%%%%%%%%%%%%%%%%%%%%%%%%%%
%%%%%%%%%%%%%%%%%%%%%%%%%%%%%%%%%%%%%%%%%%%%%%%%%%%%%%%
%%%%%%%%%%%%%%%%%%%%%%%%%%%%%%%%%%%%%%%%%%%%%%%%%%%%%%%

\begin{frame}[t]{Covariate balance}


How to form a notion of covariate balance for estimators that are not weighted averages?

\splitpage{
    \centering
    \textbf{Calibration weights}\\
    $\muhat_{\cal} = \meansur w_s y_s$
}{
    \centering
    \textbf{MrP}\\
    Take $\yhat_p = \expect{\p(\theta \vert \textrm{Survey data})}{\y | \x_p}$ and\\
    $\muhat_{\mrp} = \meantar \yhat_p$
}

\vspace{1em}
\hrulefill\\
\textbf{Step two: Specify a Taylor series.}

\def\new{\mathrm{new}}
Suppose we wanted to re--compute MrP with new
survey responses $\y_s^{\new}$.

$$
\muhat_{\mrp}(\y_1^\new, \ldots, \y_{\nsur}^\new) =
\sumsur \w_s^\mrp (\y_s^\new  - \y_s) + \mathrm{Residual}
$$

In general, MrP is truly nonlinear. The residual is only small when $\y_s^\new \approx \y_s$!


\end{frame}


%%%%%%%%%%%%%%%%%%%%%%%%%%%%%%%%%%%%%%%%%%%%%%%%%%%%%%%
%%%%%%%%%%%%%%%%%%%%%%%%%%%%%%%%%%%%%%%%%%%%%%%%%%%%%%%
%%%%%%%%%%%%%%%%%%%%%%%%%%%%%%%%%%%%%%%%%%%%%%%%%%%%%%%

\begin{frame}[t]{Covariate balance}


\textbf{Step three: Define a data perturbation that captures regression balance.}

Recall that our $\y$ is binary.  How can we produce a $\ytil$ such that
$\expect{}{\ytil \vert \x} = \expect{}{\y \vert \x} + \delta f(\x)$?
%
\begin{itemize}
    \item Use an estimate of $\expect{}{\y \vert \x}$ to draw new binary data, or
    \item Allow $\ytil$ to take values other than $\{0,1\}$ and set
        $\ytil = \y + \delta \f(\x)$.
\end{itemize}
%
Focus on the second (we have examples of the first).

\end{frame}




%%%%%%%%%%%%%%%%%%%%%%%%%%%%%%%%%%%%%%%%%%%%%%%%%%%%%%%
%%%%%%%%%%%%%%%%%%%%%%%%%%%%%%%%%%%%%%%%%%%%%%%%%%%%%%%
%%%%%%%%%%%%%%%%%%%%%%%%%%%%%%%%%%%%%%%%%%%%%%%%%%%%%%%

\begin{frame}[t]{Generalized covariate balance for MrP}

\textbf{Step one:} Construct
$\ytil$ such that $\expect{}{\ytil \vert \x} = \expect{}{\y \vert \x} + \delta f(\x)$.
\pause

\textbf{Problem:} Our $\y$ is binary!  (We're motivated by hierarchical linear regression.)

\pause
Two possibilities:
%
\begin{itemize}
    \item Allow $\ytil$ to take values other than $\{0,1\}$ and set
        $\ytil = \y + \delta \f(\x)$, or
    \item Use an estimate of $\expect{}{\y \vert \x}$ to draw new binary $\ytil$.
\end{itemize}
%
Our approach:
%
\begin{itemize}
    \item Use $\ytil = \y + \delta \f(\x)$ to identify problematic ``imbalanced'' $\f(\x)$
    \item Sanity check by generating binary $\ytil$ using $\f(\x)$ (which is fast and easy)
\end{itemize}


\end{frame}

