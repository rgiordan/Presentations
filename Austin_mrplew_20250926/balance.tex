
%%%%%%%%%%%%%%%%%%%%%%%%%%%%%%%%%%%%%%%%%%%%%%%%%%%%%%%
%%%%%%%%%%%%%%%%%%%%%%%%%%%%%%%%%%%%%%%%%%%%%%%%%%%%%%%
%%%%%%%%%%%%%%%%%%%%%%%%%%%%%%%%%%%%%%%%%%%%%%%%%%%%%%%

\begin{frame}[t]{What are we weighting for?\footnote{Pun attributable to \textcite{solon:2015:weightingfor}}}
$$
\tarcol{\textrm{Target average response} =
\meantar \y_j} \approx \surcol{\meansur \w_i \y_i
= \textrm{Weighted survey average response }}
$$
We can't check this, because we don't observe $\tarcol{\y_j}$.  \pause But we can check whether:
$$
% \textrm{Target average regeressor} =
    \tarcol{\meantar \x_j} = \surcol{\meansur \w_i \x_i}
% =    \textrm{Weighted survey average regressor}
$$

Such weights satisfy ``covariate balance'' for $\x$.

You can check covariate balance for any calibration weighting estimator,
and any function $\f(\x)$.

\pause
Even more, covariate balance is the criterion for a popular class of calibration
weight estimators:

\begin{block}{Raking calibration weights}
% Select calibration weights to satisfy
``Raking'' selects weights that
\vspace{-0.5em}
\begin{itemize}
    \item Are as ``close as possible'' to some reference weights
    \item Under the constraint that they balance some selected regressors.
\end{itemize}
%
\end{block}



\end{frame}




%%%%%%%%%%%%%%%%%%%%%%%%%%%%%%%%%%%%%%%%%%%%%%%%%%%%%%%
%%%%%%%%%%%%%%%%%%%%%%%%%%%%%%%%%%%%%%%%%%%%%%%%%%%%%%%
%%%%%%%%%%%%%%%%%%%%%%%%%%%%%%%%%%%%%%%%%%%%%%%%%%%%%%%

\begin{frame}[t]{Balance checks as sensitivity analysis}

% Why would you want covariate balance?  Some commonly stated reasons:
% %
% \begin{itemize}
% \item To reduce the variance of inverse propensity weights (IPW)
% \item To check the accuracy of IPW
% \item To exactly balance ``important regressors''
% \end{itemize}
%
% Common to these motivations is the following concen:

One reason to balance $f(\x)$ is because we think
$\expect{}{\y \vert \x}$ might plausibly vary $\propto f(\x)$,
and want to check whether our estimator can capture this variability.
%
\only<2>{
\begin{block}{Balance--informed sensitivity check (BISC) (informal)}
    Pick a small $\delta > 0$ and an $\f(\cdot)$.  Define a \emph{new response variable} $\ytil$ such that
    $$
    \expect{}{\ytil \vert \x} = \expect{}{\y \vert \x} + \delta f(\x).
    $$
    We know the change this is supposed to induce in the target population.\\[1em]

    Covariate balance checks whether our estimators produce the same change.
    %
\end{block}
}
%
\only<3->{
\begin{block}{Balance--informed sensitivity check (BISC) (formal)}
    Pick a small $\delta > 0$ and an $\f(\cdot)$.  Define a \emph{new response variable} $\ytil$ such that
    $$
    \expect{}{\ytil \vert \x} = \expect{}{\y \vert \x} + \delta f(\x).
    $$
    We know the expected change this perturbation produces in the target distribution:
    $$
    \begin{aligned}
        \tarcol{
            \expect{}{\mu(\ytil) - \mu(\y) | \x} =
            \meantar \left(\expect{}{\ytil | \x_p} - \expect{}{\y | \x_p}\right) =
            \delta \meantar f(\x_j)}
    \end{aligned}
    $$
    Then, check whether your estimator $\muhat(\cdot)$ produces
    the same change for observed \surcol{$\ytil, \y$}:
    $$
    \begin{aligned}
        \underbrace{
        \surcol{\muhat(\ytil) - \muhat(\y)}
        }_{
            \substack{
                \text{Replace weighted averages} \\
                \text{with changes in an estimator}
            }
        }
        \overset{\textrm{\textbf{check}}}{\approx}
        \tarcol{
        \delta \meantar f(\x_j).}
    \end{aligned}
    $$
\end{block}
}

\onslide<4->{
    When $\muhat(\cdot) = \muhat_\cal(\cdot)$,
    BISC recovers the standard covariate balance check. \\[1em]
    We will use $\muhat(\cdot) = \muhat_\mrp(\cdot)$.
}

\end{frame}



%%%%%%%%%%%%%%%%%%%%%%%%%%%%%%%%%%%%%%%%%%%%%%%%%%%%%%%
%%%%%%%%%%%%%%%%%%%%%%%%%%%%%%%%%%%%%%%%%%%%%%%%%%%%%%%
%%%%%%%%%%%%%%%%%%%%%%%%%%%%%%%%%%%%%%%%%%%%%%%%%%%%%%%

\begin{frame}[t]{BISC for MrP}

Suppose I have
$\ytil$ such that $\expect{}{\ytil \vert \x} = \expect{}{\y \vert \x} + \delta f(\x)$.\\
Now I need to evaluate \surcol{$\muhat_\mrp(\ytil) - \muhat_\mrp(\y)$}.
\pause

\textbf{Problem:} \surcol{$\muhat_\mrp(\cdot)$} is computed with MCMC.
%
\begin{itemize}
\item Each MCMC run typically takes hours, and
\item Output is noisy, and \surcol{$\muhat_\mrp(\ytil) - \muhat_\mrp(\y)$} may be small.
\end{itemize}
%

\pause
\begin{block}{MrP Local Equivalent Weights (MrPlew)}
Form the first-order Taylor series approximation
\surcol{
$$
\muhat_{\mrp}(\ytil) - \muhat_\mrp(\y) \approx
    \sumsur \w_i^\mrp (\ytil_i  - \y_i)
\quad\textrm{where}\quad
    \wmrp_i := \frac{d}{d\y_i} \muhat_{\mrp}(\y).
$$
}
\end{block}


\only<4>{
\textbf{Computation: }
The weights are given by weighted averages of posterior covariances  \footcite{giordano:2018:covariances}.
\\[1em]
They can be easily computed with standard software\footnote{We use \texttt{brms} \parencite{brms}.}
\textbf{without re--running MCMC}.\\
}



\only<5>{
\textbf{Use in BISC: } For a wide set of judiciously chosen $\f(\cdot)$, check
$$
\surcol{
 %   \underset{
 %       \approx \muhat_{\mrp}(\ytil) - \muhat_\mrp(\y)
 %   }{
    \delta \sumsur \w_i^\mrp \f(\x_i)}
 %   }
        \overset{\textrm{check}}{\approx}
%\underset{
%    \approx \tarcol{\expect{}{\mu(\ytil) - \mu(\y)}}
%}{
    \tarcol{\delta \meantar f(\x_j).}
%}
$$


}

\end{frame}


%%%%%%%%%%%%%%%%%%%%%%%%%%%%%%%%%%%%%%%%%%%%%%%%%%%%%%%
%%%%%%%%%%%%%%%%%%%%%%%%%%%%%%%%%%%%%%%%%%%%%%%%%%%%%%%
%%%%%%%%%%%%%%%%%%%%%%%%%%%%%%%%%%%%%%%%%%%%%%%%%%%%%%%

\begin{frame}[t]{Generating $\ytil$}
%
\begin{itemize}
\item We have defined BISC in terms of $\ytil$ such that
$\expect{}{\ytil \vert \x} = \expect{}{\y \vert \x} + \delta f(\x)$
\item We have approximated $\muhat_{\mrp}(\ytil) - \muhat_\mrp(\y)$ for $\ytil \approx \y$
\end{itemize}
%
How to get such a $\ytil$?  \textbf{Recall $\y$ is binary!}  Two approaches:


\splitpagenoline{
    \centering
    \textbf{Option 1:} Force $\ytil$ to be binary.
    %
    \begin{enumerate}
    \item Make \textit{some} guess $\hat{m}(\x) \approx \expect{}{\y \vert \x}$
    %
    \begin{itemize}
    \item E.g.~Posterior mean, or
    \item Shrunken posterior mean, or
    \item Some values that gives the same posterior
    \end{itemize}
    %
    \item Take $u_n \iid \mathrm{Unif}(0,1)$
    \item Assume $\y_n = \ind{u_n \le \hat{m}(\x_n)}$
    \item Draw $u_n \vert \y_n$
    \item Set $\ytil_n = \ind{u_n \le \hat{m}(\x_n)+ \delta \x_n}$
%
\end{enumerate}
%
} {
    \centering
    \textbf{Option 2:} Allow $\ytil$ to take generic values.
    \begin{enumerate}
        \item Set $\ytil_n = \y_n + \delta \f(\x_n)$.
    \end{enumerate}
} \pause
%
\\[1em]
\splitpagenoline{
%
\begin{itemize}
\item Realistic
\item Have to pick $\hat{m}(\x)$
\item $\Ytil - \Ysur$ not infinitesimally small
\item \textbf{Sanity check for theory}
\end{itemize}
%
}
{
\begin{itemize}
\item Not realistic
\item No additional assumptions
\item $\Ytil - \Ysur$ may be infinitesimally small
\item \textbf{Use for theory}
\end{itemize}
}



\end{frame}

%%%%%%%%%%%%%%%%%%%%%%%%%%%%%%%%%%%%%%%%%%%%%%%%%%%%%%%
%%%%%%%%%%%%%%%%%%%%%%%%%%%%%%%%%%%%%%%%%%%%%%%%%%%%%%%
%%%%%%%%%%%%%%%%%%%%%%%%%%%%%%%%%%%%%%%%%%%%%%%%%%%%%%%

\begin{frame}[c]{Theory}



\begin{block}{BISC Theorem: (sketch)}
    Take $\ytil_n = \y_n + \delta \f(\x_n)$.\\[1em]
    We state conditions for Bayesian hierarchical logistic regression under which
$$
\onslide<4->{\sup_{f \in \mathcal{F}}}
\abs{
    \muhat_{\mrp}(\Ysur) - \muhat_\mrp(\Ysur)
    - \delta \sumsur \w_i^\mrp \f(\x_i)
}
\only<1>{ = \textrm{Small?}}
\only<2->{ = O(\delta^2)}
\onslide<3->{\textrm{ as }N \rightarrow \infty}
$$
\onslide<4->{
    For a very broad class\footnote{Measurable functions with uniformly bounded
    $\expect{}{\x \f(\x)}$.} of $\mathcal{F}$.
}
\end{block}

\onslide<4->{
    \textbf{Uniformity justifies searching for ``imabalanced'' $\f$.}
}

\onslide<5->{
The uniformity result builds on our earlier work on uniform
and finite--sample error bounds
for Bernstein--von Mises theorem--like results
\footcite{giordano:2024:bayesij,kasprzak:2025:laplace}.
}


\end{frame}
