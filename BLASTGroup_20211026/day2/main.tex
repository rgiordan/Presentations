\documentclass[8pt]{beamer}\usepackage[]{graphicx}\usepackage[]{color}
%% maxwidth is the original width if it is less than linewidth
%% otherwise use linewidth (to make sure the graphics do not exceed the margin)
\makeatletter
\def\maxwidth{ %
  \ifdim\Gin@nat@width>\linewidth
    \linewidth
  \else
    \Gin@nat@width
  \fi
}
\makeatother

\definecolor{fgcolor}{rgb}{0.345, 0.345, 0.345}
\newcommand{\hlnum}[1]{\textcolor[rgb]{0.686,0.059,0.569}{#1}}%
\newcommand{\hlstr}[1]{\textcolor[rgb]{0.192,0.494,0.8}{#1}}%
\newcommand{\hlcom}[1]{\textcolor[rgb]{0.678,0.584,0.686}{\textit{#1}}}%
\newcommand{\hlopt}[1]{\textcolor[rgb]{0,0,0}{#1}}%
\newcommand{\hlstd}[1]{\textcolor[rgb]{0.345,0.345,0.345}{#1}}%
\newcommand{\hlkwa}[1]{\textcolor[rgb]{0.161,0.373,0.58}{\textbf{#1}}}%
\newcommand{\hlkwb}[1]{\textcolor[rgb]{0.69,0.353,0.396}{#1}}%
\newcommand{\hlkwc}[1]{\textcolor[rgb]{0.333,0.667,0.333}{#1}}%
\newcommand{\hlkwd}[1]{\textcolor[rgb]{0.737,0.353,0.396}{\textbf{#1}}}%
\let\hlipl\hlkwb

\usepackage{framed}
\makeatletter
\newenvironment{kframe}{%
 \def\at@end@of@kframe{}%
 \ifinner\ifhmode%
  \def\at@end@of@kframe{\end{minipage}}%
  \begin{minipage}{\columnwidth}%
 \fi\fi%
 \def\FrameCommand##1{\hskip\@totalleftmargin \hskip-\fboxsep
 \colorbox{shadecolor}{##1}\hskip-\fboxsep
     % There is no \\@totalrightmargin, so:
     \hskip-\linewidth \hskip-\@totalleftmargin \hskip\columnwidth}%
 \MakeFramed {\advance\hsize-\width
   \@totalleftmargin\z@ \linewidth\hsize
   \@setminipage}}%
 {\par\unskip\endMakeFramed%
 \at@end@of@kframe}
\makeatother

\definecolor{shadecolor}{rgb}{.97, .97, .97}
\definecolor{messagecolor}{rgb}{0, 0, 0}
\definecolor{warningcolor}{rgb}{1, 0, 1}
\definecolor{errorcolor}{rgb}{1, 0, 0}
\newenvironment{knitrout}{}{} % an empty environment to be redefined in TeX

\usepackage{alltt}
\usetheme{metropolis}           % Use metropolis theme

\usepackage{graphicx}

\DeclareGraphicsExtensions{.pdf,.jpeg,.jpg,.isba_2021/*.tex,.png}

\usepackage{subcaption}
\usepackage{amsmath}
\usepackage{mathtools}
\usepackage[normalem]{ulem} % For \sout

\usepackage[authoryear]{natbib}

\usepackage{tikz}
\usetikzlibrary{bayesnet}
\usepackage{pgfplots}
\pgfplotsset{compat=1.13}

\usepackage[framemethod=TikZ, xcolor=RGB]{mdframed}
\definecolor{mydarkblue}{rgb}{0,.06,.5}
\definecolor{mydarkred}{rgb}{.5,0,.1}
\definecolor{myroyalblue}{rgb}{0,.1,.8}
\mdfdefinestyle{MyFrame}{%
    linecolor=mydarkblue,
    outerlinewidth=0.5pt,
    roundcorner=2pt,
    innertopmargin=2pt,
    innerbottommargin=2pt,
    innerrightmargin=2pt,
    innerleftmargin=2pt,
    backgroundcolor=blue!10}

% Set a transparent background to match ggplot figures
\setbeamercolor{background canvas}{bg=}

\usepackage{xargs} % For def with default arguments


\def\undernote#1#2{\underbrace{#1}_{\mathclap{\substack{#2}}}}
\def\overnote#1#2{\overbrace{#1}^{\mathclap{\substack{#2}}}}
\def\question#1{
\begin{mdframed}[style=MyFrame]
\begin{center}
#1
\end{center}
\end{mdframed}
}

\definecolor{theoremcolor}{RGB}{255,245,210}

\def\theorem#1{
\begin{mdframed}[style=MyFrame,backgroundcolor=theoremcolor]
    #1
\end{mdframed}
}

% Paper-specific math macros.
\def\wtil{\tilde{w}}


\def\red#1{\textcolor{red}{#1}}
\def\blue#1{\textcolor{blue}{#1}}


% Operators
\def\mbe{\mathbb{E}}%
\def\ind#1{\mathbb{I}\left(#1\right)}
\def\norm#1{\left\Vert #1\right\Vert }
\def\evalat#1#2{\left.#1\right|_{#2}}
\def\fracat#1#2#3{\left.\frac{#1}{#2}\right\vert_{#3}}
\def\iid{\overset{iid}{\sim}}
\def\indep{\overset{indep}{\sim}}
%\def\argsup{\mathrm{argsup}\,}
%\def\argmax{\mathrm{argmax}\,}
\DeclareMathOperator*{\argmax}{\mathrm{argmax}}
\DeclareMathOperator*{\argsup}{\mathrm{argsup}}
\DeclareMathOperator*{\esssup}{\mathrm{esssup}}

% Dimension of g(\theta)
\def\gdim{D_g}

% Expectation and centering
\def\expect#1#2{\underset{#1}{\mathbb{E}}\left[#2\right]}
\def\expectop#1{\underset{#1}{\mathbb{E}}}
%\def\cent#1#2{\underset{#1}{\mathbb{C}}\left[#2\right]}
\def\cov#1#2{\underset{#1}{\mathrm{Cov}}\left(#2\right)}
\def\var#1#2{\underset{#1}{\mathrm{Var}}\left(#2\right)}
\def\covhat#1#2{\underset{#1}{\widehat{\mathrm{Cov}}}\left(#2\right)}
\def\varhat#1#2{\underset{#1}{\widehat{\mathrm{Var}}}\left(#2\right)}

\def\p{p}
\def\post{p(\theta | \xvec)}
\def\postz{p(\theta, \z | \xvec)}
\def\postpert#1{p(\theta | #1)}
\def\postzpert#1{p(\theta, \z | #1)}
\def\sumn{\sum_{n=1}^N}
\def\meann{\frac{1}{N}\sum_{n=1}^N}
\def\r{r}
\def\y{y}
\def\x{x}
\def\xvec{X}
\def\z{z}
\def\w{w}
\def\onevec{1_N}

\def\fdist{\mathbb{F}}
\def\xfdist{\xvec\iid\fdist}

% von Mises stuff
\def\gdist{\mathbb{G}}
\def\fndist{\mathbb{F}_N}
\def\fwdist{\mathbb{F}_{N}^{w}}
\def\distdiff{\Delta}

\def\thetadom{\Omega_\theta}
\def\thetaball#1{B_{#1}}
\def\deltaulln{\delta_{LLN}}

% Information matrix stuff
\def\info{\mathcal{I}}
\def\infoev{\lambda_\mathcal{I}}
\def\infoevhat{\hat{\lambda}_\mathcal{I}}
\def\infohat{\hat{\mathcal{I}}}

% Optima and related features of the log likelihood
\def\scorecov{\Sigma}
\def\scorecovhat{\hat\Sigma}
\def\thetatrue{\theta_0}
\def\thetahat{\hat\theta}
\def\zhat{\hat\z}
\def\thetatil{\tilde{\theta}}

% BCLT residuals
\def\resid#1{\mathscr{R}_{#1}}

% Taylor series residual
\def\tresid#1#2{\mathscr{T}_{(#1)}\left(#2\right)}

% The regular event
\def\regevent{\mathfrak{R}}

% Covariances of g(\theta)
\def\gcovtrue{V_{\fdist}}
\def\gcovlim{V_{\infty}}
\def\gcovmaphat{\hat{V}^{\mathrm{MAP}}}
\def\gcovhat{\hat{V}_{\mathrm{CLT}}}
\def\gcovij{V_{\mathrm{IJ}}}
\def\gcovijhat{\hat{V}_{\mathrm{IJ}}}
\def\gcovboothat{\hat{V}_{\mathrm{Boot}}}
\def\gcovboot{V_{\mathrm{Boot}}}
\def\gcovbayeshat{\hat{V}_{\mathrm{Bayes}}}
\def\gcovbayes{V_{\mathrm{Bayes}}}

% Influence function)
\def\infl{\psi}
\def\inflalt#1{\psi^{#1}}
\def\inflx{\phi}
\def\inflhat{\hat{\psi}}
\def\err{\mathcal{E}}

% Modes of convergence
%\stackrel{p}{\rightarrow}
\def\plim{\xrightarrow[N\rightarrow\infty]{prob}\,}
\def\dlim{\xrightarrow[N\rightarrow\infty]{dist}\,}

% Taylor expansion and gradient notation
\def\ggrad#1{g_{(#1)}}
\def\psigrad#1{\psi_{(#1)}}
\def\ellgrad#1{\ell_{(#1)}}
\def\phigrad#1{\phi_{(#1)}}
\def\psigrad#1{\psi_{(#1)}}
%\def\tgrad{T_{(1)}}
\def\tgrad{\mathfrak{d}T}

% Order notatoin
\def\ord#1{O\left(#1\right)}
\def\ordlog#1{\tilde{O}\left(#1\right)}
\def\ordlogp#1{\tilde{O}_p\left(#1\right)}

% Log likelihood
\def\lik{\mathscr{L}}
\def\likhat{\hat{\mathscr{L}}}
\def\likk#1{\mathscr{L}_{(#1)}}
\def\likhatk#1{\hat{\mathscr{L}}_{(#1)}}

% Higher order notation
% Bar is posterior centering, underbar is frequentist centering
\def\gbar{\bar{\g}}
\def\gbargrad#1{\bar{\g}_{(#1)}}
\def\ghatgrad#1{\hat{\g}_{(#1)}}
\def\fbar{\bar{f}}
\def\thetabar{\fbar(\theta)}
\def\gammabar{\bar{\gamma}}

\def\ellbar{\bar{\ell}}
\def\ellunderbar{\underline{\ell}}
\def\ellbarbar{\underline{\ellbar}}
\def\ellbargrad#1{\bar{\ell}_{(#1)}}
\def\ellbarbargrad#1{\ellbarbar_{(#1)}}
\def\ellunderbargrad#1{\ellunderbar_{(#1)}}
\def\ellhatunderbar{\hat{\underline{\ell}}} % X-demeaned, centered at \thetahat
\def\ellhatunderbargrad#1{\ellhatunderbar_{(#1)}}

\newcommand{\spskip}{\vspace{1em}}
\usepackage{tikz}
%\usepackage{ulem} % for strikeout


% population colors: set2 from colorbrewer
\definecolor{pop1}{HTML}{66c2a5}
\definecolor{pop2}{HTML}{fc8d62}
\definecolor{pop3}{HTML}{8da0cb}
\definecolor{pop4}{HTML}{e78ac3}
\definecolor{pop5}{HTML}{a6d854}
\definecolor{pop6}{HTML}{ffd92f}
\definecolor{pop7}{HTML}{e5c494}
\definecolor{pop8}{HTML}{b3b3b3}

\title{Variational Methods for Latent Variable Problems (part 2)}
\author{Ryan Giordano (for Johns Hopkins Biostats BLAST working group)}
\date{Oct, 2021}
\institute{Massachusetts Institute of Technology}

\setbeamertemplate{Collaborators}[none]
\IfFileExists{upquote.sty}{\usepackage{upquote}}{}




\begin{document}

\maketitle

\begin{frame}{Dropping data: Motivation}

More data \& cheaper computation $\Rightarrow$\\
Statistical analyses are
playing larger roles in decision making.

\vspace{1em}
Decisions are important: We want \textbf{trustworthy} conclusions.\\
Data / models not always perfect: We want \textbf{robust} conclusions.

% \begin{itemize}
%     \item Gather + clean exchangeable data,
%     \item Specify and fit a model, and
%     \item Drawn a qualitative conclusion from your fit
%     \\(e.g., based on the sign / significance of
%         some estimated parameter).
% \end{itemize}

\vspace{1em}
Would you be concerned if you could \textbf{reverse your conclusion} by removing
a \textbf{small proportion} (say, $0.1\%$) of your data?

\vspace{1em}
\textbf{Running example:} \citet{angelucci2015microcredit}, a
randomized controlled trial study of the efficacy of microcredit based
on 16,560 data points.  \\We can reverse the studies qualitative
conclusions by removing $15$ observations ($< 0.1 \%$ of the data).

\vspace{1em}
\textbf{How do we find sets of influential points?}  Difficult in general!\\

\vspace{1em}
We provide a \textbf{automatic approximation} with finite-sample
guarantees.

\vspace{1em}
Studying the approximation reveals the causes of non-robustness.

\end{frame}

%%%%%%%%%%%%%%%%%%%%%%%%%%%%%%%%%%%%%%%%%%%%%%%%%%%%%%%%%%%%%%%%%%%%%%%%
%%%%%%%%%%%%%%%%%%%%%%%%%%%%%%%%%%%%%%%%%%%%%%%%%%%%%%%%%%%%%%%%%%%%%%%%
%%%%%%%%%%%%%%%%%%%%%%%%%%%%%%%%%%%%%%%%%%%%%%%%%%%%%%%%%%%%%%%%%%%%%%%%

\begin{frame}{Dropping data: Mexico Microcredit}

Consider \citet{angelucci2015microcredit}, a randomized controlled trial study
of the efficacy of microcredit in Mexico based on 16,560 data points.

The variable ``Beta" estimates the effect of microcredit in US dollars.

%\MicrocreditMexicoRerunTable{}

\begin{table}[ht]
\centering
\begin{tabular}{lll} \hline
  & \onslide<2->{Left out points} & Beta (SE) \\\hline
\hspace{0.05em} Original result & \onslide<2->{0} & -4.55 (5.88) \\ \hline
\onslide<2-> {\hspace{0.05em} Change sign & 1 & 0.4 (3.19) \\\hline }
\onslide<3-> {Change significance & 14 & -10.96 (5.57) \\\hline }
\onslide<4-> {Change sign and significance & 15 & 7.03 (2.55) \\\hline }
\end{tabular}
\end{table}

\vspace{-1em}
\onslide<1-> { \textbf{Original conclusion: }\\
There is no evidence that microcredit is effective.}

\vspace{1em}
\onslide<5-> { \textbf{Potential conclusions after data dropping: }\\
The effect of microcredit is positive (negative) \&
statistically significant.}

% \vspace{1em}
% \onslide<6-> { By removing very few data points ($15 / 16560 \approx 0.1\% $),
% we can reverse the qualitative conclusions of the original study! }

\vspace{1em}
\onslide<6-> { \textbf{The culprit is signal to noise ratio. }\\
By the end of the talk, we will see that the sensitivity is due to
\begin{itemize}
\item High variability of the outcome (hosehold profit) relative to
\item A small signal driving the conclusion
(statistical significance)
\end{itemize}
}



\end{frame}

% %%%%%%%%%%%%%%%%%%%%%%%%%%%%%%%%%%%%%%%%%%%%%%%%%%%%%%%%%%%%%%%%%%%%%%%%
% %%%%%%%%%%%%%%%%%%%%%%%%%%%%%%%%%%%%%%%%%%%%%%%%%%%%%%%%%%%%%%%%%%%%%%%%
% %%%%%%%%%%%%%%%%%%%%%%%%%%%%%%%%%%%%%%%%%%%%%%%%%%%%%%%%%%%%%%%%%%%%%%%%

\begin{frame}{Dropping data: Motivation}

Would you be concerned if you could \textbf{reverse your conclusion} by removing
a \textbf{small proportion} (say, $0.1\%$) of your data?

\pause
Not always!  But sometimes, surely yes.

Thinking without random noise can be helpful.

Suppose you have a farm, and want to know whether
your average yield is greater than 170 bushels per acre.
At harvest, you measure 200 bushels per acre.

\begin{itemize}
    \item Scenario one: If your yield is greater than 170 bushels
        per acre, you make a profit.
        \begin{itemize}
            \item Don’t care about sensitivity to small subsets
        \end{itemize}
    \item Scenario two: You want to recommend your farming
    methods to a friend across the valley.
    \begin{itemize}
        \item Might care about sensitivity to small subsets
    \end{itemize}
\end{itemize}

For example, often in economics:
%
\begin{itemize}
\item Small fractions of data are missing not-at-random,
\item Policy population is different from analyzed population,
\item We report a convenient summary (e.g. mean) of a complex effect,
\item Models are stylized proxies of reality.
\end{itemize}


\end{frame}


%%%%%%%%%%%%%%%%%%%%%%%%%%%%%%%%%%%%%%%%%%%%%%%%%%%%%%%%
%%%%%%%%%%%%%%%%%%%%%%%%%%%%%%%%%%%%%%%%%%%%%%%%%%%%%%%%
%%%%%%%%%%%%%%%%%%%%%%%%%%%%%%%%%%%%%%%%%%%%%%%%%%%%%%%%


\begin{frame}{What counts as variational inference?}

Lots of very different procedures go by the name ``variational inference.''
%
I propose an (idosyncratic) enompassing definition based on the use cases and
the name:

\begin{center}
\textbf{Variational inference is inference
using optimization.}
\end{center}

%
Think ``calculus of variations:'' an optimum $\hat{x} = \argmax_\theta f(x)$ is
characterized by $df / dx |_{\hat{x}} = 0$, i.e.  where small variations in
$\hat{x}$ result in no changes to the value of $f(\hat{x})$.
%

\begin{center}
\begin{tikzpicture}

\draw (0,0)--(5,0);
\foreach \x in {0,...,5}
  \draw (\x,0)--(\x,-.1) node[anchor=north]{};

\draw (0,0)--(0,3);
\foreach \y in {0,...,3}
  \draw (0,\y)--(-.1,\y) node[anchor=east] {};

\draw (0,3) node[anchor=east] {$f(x)$};
\draw (5,0) node[anchor=north] {$x$};

\coordinate (A) at (1.0, 0.5);
\coordinate (B) at (3.0, 2.0);
\coordinate (C) at (5.0, 0.7);

\draw    (A) to[out=0,in=180] (B);
\draw    (B) to[out=0,in=120] (C);

\draw (3, 2.0) node[anchor=south] {$\hat{x}$};
\node at (3, 2) [circle,fill=black,inner sep=0.9pt]{};
\draw [to-to] (3 - 0.5, 2.0) -- (3 + 0.5, 2.0);

\end{tikzpicture}
\end{center}


\begin{minipage}[t]{0.2\textwidth}
By this definition,
\end{minipage}
%
\begin{minipage}[t]{0.8\textwidth}
    \vspace{-0.65em}
\begin{itemize}
    \item The maximum likelihood estimator (MLE) is VI.
    \item The Laplace approximation to a Bayesian posterior is VI.
    \item Markov chain Monte Carlo (MCMC) is not VI.
\end{itemize}
\end{minipage}
\hspace{-2em}

\end{frame}


%%%%%%%%%%%%%%%%%%%%%%%%%%%%%%%%%%%%%%%%%%%%%%%%%%%%%%%%
%%%%%%%%%%%%%%%%%%%%%%%%%%%%%%%%%%%%%%%%%%%%%%%%%%%%%%%%
%%%%%%%%%%%%%%%%%%%%%%%%%%%%%%%%%%%%%%%%%%%%%%%%%%%%%%%%


\begin{frame}{What counts as variational inference?}

A more common definition of VI is the following.

Suppose we have a random variable $\xi$ and a distribution $\p(\xi)$
that we want to know.

Let $y$ denote data and $\theta$ a parameter. Examples:
\begin{itemize}
    \item The variable is $\theta$, and we wish to know the posterior
        $\p(\theta | y)$ (Bayes)
    %
    \item The variable is $y$, and we wish to know $\p(y)$ (MLE)
    %
    \item The variable is $y$, and we wish to know the map
        $\theta \mapsto  \p(y | \theta) = \int p(y, z | \theta)  dz$ (marginal MLE)
\end{itemize}

Let $\qdom$ be some class of distributions which may or may not contain
$\p(\xi)$.



\begin{center}
\textbf{Variational inference finds the distribution in $\qdom$
closest to $\p$ according to some measure of ``divergence''
between distributions:}
%
\begin{align*}
%
\qopt(\xi) = \argmin{\q \in \qdom} D(\q, \p).
%
\end{align*}
%
\end{center}

The most common choice of ``divergence'' is the \textbf{Kullback-Leibler} (KL)
divergence, though other choices are possible \citep{li2016variational,
liu2016stein, ambrogioni2018wasserstein}.

\end{frame}




%%%%%%%%%%%%%%%%%%%%%%%%%%%%%%%%%%%%%%%%%%%%%%%%%%%%%%%%
%%%%%%%%%%%%%%%%%%%%%%%%%%%%%%%%%%%%%%%%%%%%%%%%%%%%%%%%
%%%%%%%%%%%%%%%%%%%%%%%%%%%%%%%%%%%%%%%%%%%%%%%%%%%%%%%%


\begin{frame}{KL divergence}

The KL divergence is defined as:
%
\begin{align*}
%
\kl{\q}{\p} :=
% \expect{\q(\xi)}{\log\frac{\q(\xi)}{\p(\xi)}} =
\expect{\q(\xi)}{\log \q(\xi)} - \expect{\q(\xi)}{\log \p(\xi)}
%
\end{align*}
%

Some points to be aware of:
\begin{itemize}
    \item $\kl{\q}{\p} \ge 0$
    \item $\kl{\q}{\p} = 0 \Rightarrow \p = \q$
    \item $\kl{\q}{\p} \ne \kl{\p}{\q}$
    % \item $\kl{\q}{\p}$ does not satisfy the triangle inequality
    \item $\kl{\q}{\p}$ is a ``strict'' measure of closeness
        \citep{gibbs2002choosing}
    % \begin{itemize}
    %      \item If the KL divergence is small, other common
    %      measures of distance between distributions are small, but
    %      not vice-versa
    %  \end{itemize}
\end{itemize}

Why use KL divergence?

\textbf{Phony answer:}
The KL divergence has an information theoretic
interpretation \citep{kullback1951information}.

\textbf{Real answer:}
Mathematical convenience (normalizing constants pop out).

\textbf{Example: the MLE minimizes KL divergence.}  Suppose that
$x_n \iid \p(\cdot)$, and $\q(\cdot | \theta) \in \qdom$ is a parameteric
family of data distributions.
Then
%
% \begin{align*}
%
% \hat\theta = \argmin{\theta} \kl{\p}{\q} ={}&
% \argmin{\theta}\left(  - \expect{\p(x_1)}{\log \q(x_1 | \theta)}
%     + \expect{\p(x_1)}{\log \p(x_1)} \right) \\
%={}&
% \argmax{\theta} \expect{\p(x_1)}{\log \q(x_1 | \theta)} \\
%
% \end{align*}
%
\end{frame}


\input{kl_exercises}

\begin{frame}{Conclusions}

\footnotesize

\bibliographystyle{plainnat}
% Hide the references header
% https://tex.stackexchange.com/questions/22645/hiding-the-title-of-the-bibliography/370784
\begingroup
\renewcommand{\section}[2]{}%
\bibliography{references}
\endgroup

\end{frame}


\end{document}
