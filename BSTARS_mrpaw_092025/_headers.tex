%% maxwidth is the original width if it is less than linewidth
%% otherwise use linewidth (to make sure the graphics do not exceed the margin)
\makeatletter
\def\maxwidth{ %
  \ifdim\Gin@nat@width>\linewidth
    \linewidth
  \else
    \Gin@nat@width
  \fi
}
\makeatother

\definecolor{fgcolor}{rgb}{0.345, 0.345, 0.345}
\newcommand{\hlnum}[1]{\textcolor[rgb]{0.686,0.059,0.569}{#1}}%
\newcommand{\hlstr}[1]{\textcolor[rgb]{0.192,0.494,0.8}{#1}}%
\newcommand{\hlcom}[1]{\textcolor[rgb]{0.678,0.584,0.686}{\textit{#1}}}%
\newcommand{\hlopt}[1]{\textcolor[rgb]{0,0,0}{#1}}%
\newcommand{\hlstd}[1]{\textcolor[rgb]{0.345,0.345,0.345}{#1}}%
\newcommand{\hlkwa}[1]{\textcolor[rgb]{0.161,0.373,0.58}{\textbf{#1}}}%
\newcommand{\hlkwb}[1]{\textcolor[rgb]{0.69,0.353,0.396}{#1}}%
\newcommand{\hlkwc}[1]{\textcolor[rgb]{0.333,0.667,0.333}{#1}}%
\newcommand{\hlkwd}[1]{\textcolor[rgb]{0.737,0.353,0.396}{\textbf{#1}}}%
\let\hlipl\hlkwb

\usepackage{framed}
\makeatletter
\newenvironment{kframe}{%
 \def\at@end@of@kframe{}%
 \ifinner\ifhmode%
  \def\at@end@of@kframe{\end{minipage}}%
  \begin{minipage}{\columnwidth}%
 \fi\fi%
 \def\FrameCommand##1{\hskip\@totalleftmargin \hskip-\fboxsep
 \colorbox{shadecolor}{##1}\hskip-\fboxsep
     % There is no \\@totalrightmargin, so:
     \hskip-\linewidth \hskip-\@totalleftmargin \hskip\columnwidth}%
 \MakeFramed {\advance\hsize-\width
   \@totalleftmargin\z@ \linewidth\hsize
   \@setminipage}}%
 {\par\unskip\endMakeFramed%
 \at@end@of@kframe}
\makeatother

\definecolor{shadecolor}{rgb}{.97, .97, .97}
\definecolor{messagecolor}{rgb}{0, 0, 0}
\definecolor{warningcolor}{rgb}{1, 0, 1}
\definecolor{errorcolor}{rgb}{1, 0, 0}
\newenvironment{knitrout}{}{} % an empty environment to be redefined in TeX

\usepackage{alltt}
\usetheme{metropolis}           % Use metropolis theme

\usepackage{graphicx}

\DeclareGraphicsExtensions{.pdf,.jpeg,.jpg,.isba_2021/*.tex,.png}

\usepackage{subcaption}
\usepackage{amsmath}

\usepackage[authoryear]{natbib}

\usepackage{tikz}
\usetikzlibrary{bayesnet}
\usepackage{pgfplots}
\pgfplotsset{compat=1.13}

\usepackage[framemethod=TikZ, xcolor=RGB]{mdframed}
\definecolor{mydarkblue}{rgb}{0,.06,.5}
\definecolor{mydarkred}{rgb}{.5,0,.1}
\definecolor{myroyalblue}{rgb}{0,.1,.8}
\mdfdefinestyle{MyFrame}{%
    linecolor=mydarkblue,
    outerlinewidth=0.5pt,
    roundcorner=2pt,
    innertopmargin=2pt,
    innerbottommargin=2pt,
    innerrightmargin=2pt,
    innerleftmargin=2pt,
    backgroundcolor=blue!10}

% Set a transparent background to match ggplot figures
\setbeamercolor{background canvas}{bg=}

\usepackage{xargs} % For def with default arguments



% \input{static_figs.tex}
\newcommand{\spskip}{\vspace{1em}}
\usepackage{tikz}
%\usepackage{ulem} % for strikeout

% population colors: set2 from colorbrewer
\definecolor{pop1}{HTML}{66c2a5}
\definecolor{pop2}{HTML}{fc8d62}
\definecolor{pop3}{HTML}{8da0cb}
\definecolor{pop4}{HTML}{e78ac3}
\definecolor{pop5}{HTML}{a6d854}
\definecolor{pop6}{HTML}{ffd92f}
\definecolor{pop7}{HTML}{e5c494}
\definecolor{pop8}{HTML}{b3b3b3}


\usepackage{mathtools}

\def\subitem#1{\begin{itemize} \item[] #1 \end{itemize}}

\usepackage[export]{adjustbox} % for adjincludegraphics


% block
\setbeamercolor{block title}{fg=blue,bg=blue!20!bg}
\setbeamercolor{block body}{bg=block title.bg!30!bg}