
\begin{frame}{Future work}

How do you peform model checking with sensitivity analysis?

Existing methods evaluate whether the analysis changes ``a lot'' when you:
%
\begin{itemize}
\item Parametrically perturb the model (e.g.~fit a richer model class)
\item Non--parameterically perturb the data (e.g.~produce gross outliers)
\end{itemize}
%
The problem is:
%
\begin{itemize}
\item How much is ``a lot''?
\item Non--parametric data perturbations are hard to reason about
\item It's hard to say whether parametric model changes are enough
\end{itemize}
%


Instead, we
%
\begin{itemize}
\item Parametrically perturb the data
\item Observe whether our model could detect the change
\end{itemize}
%
\begin{itemize}
\item Know exactly the expected change (don't have to decide on what ``a lot'' means)
\item Easy to reason about whether the data perturbation is reasonable
\item Don't need to propose an alternative model, instead study the model you have
\end{itemize}


\end{frame}


%%%%%%%%%%%%%%%%%%%%%%%%%%%%%%%%%%%%%%%%
%%%%%%%%%%%%%%%%%%%%%%%%%%%%%%%%%%%%%%%%
%%%%%%%%%%%%%%%%%%%%%%%%%%%%%%%%%%%%%%%%


