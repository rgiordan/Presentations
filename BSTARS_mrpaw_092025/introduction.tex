
\begin{frame}{Are US non-voters becoming more Republican?}

\splitpage{
    \centering
    \textbf{Blue Rose research says yes:}
    \\[1em]
    ``Politically disengaged voters have become much more Republican,
        And because less-engaged voters swung away from [Democrats], an expanded
        electorate meant a more Republican electorate.''
    \\[1em]
    \parencite{blueroseresearch:2024} \\
    (major professional pollsters)
}{
    \textbf{\emph{On Data and Democracy} says no:}
    \\[1em]
    \centering
    ``Claims of a decisive pro-Republican shift among the overall non-voting population are
    not supported by the most reliable, large-scale post-election data currently available.''
    \\[1em]
    \parencite{datademocracyblog:2025} \\
    (major professional researchers)
}
\vspace{1em}
\pause
% Several factors drive the disagreement:
%
\hrule
\begin{itemize}
\item The problem is very hard (it's difficult to accurately poll non--voters)
\item Different data sources
%
% \begin{itemize}
%     \item Blue Rose aggregates its own private data
%     \item The \emph{On Data and Democracy} posts use public data, e.g.~the
%           cooperative election study (CES)
% \end{itemize}
%
\item \textbf{Very different statistical methods:} $\star$
\begin{itemize}
    \item Blue Rose uses Bayesian hierarchical modeling (MrP)
    \item The CES uses calibration weighting (CW)
\end{itemize}
%
\end{itemize}

\pause

\begin{block}{Our contribution}
    We define ``MrP local equivalent weights'' (MrPlew) that:
    %
\begin{itemize}
    \item Are easily computable from MCMC draws and standard software, and
    \item Provide MrP versions of key diagnostics that motivate calibration weighting.
\end{itemize}
%
\textbf{$\Rightarrow$ MrPlew provides direct comparisons between
MrP and calibration weighting.}
%
\end{block}

\end{frame}


%%%%%%%%%%%%%%%%%%%%%%%%%%%%%%%%%%%%%%%%
%%%%%%%%%%%%%%%%%%%%%%%%%%%%%%%%%%%%%%%%
%%%%%%%%%%%%%%%%%%%%%%%%%%%%%%%%%%%%%%%%


\begin{frame}{Outline}
%
\begin{itemize}
    \item Introduce the statisical problem and two methods (CW and MrP)
    \item Describe covariate balance, one of the classical CW diagnostics
    \item Define MrPlew weights and connect them to covariate balance
    \item Example of real-world results
    \item Future directions
\end{itemize}
%
\end{frame}
