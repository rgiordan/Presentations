
\begin{frame}{Are US non-voters becoming more Republican?}


\begin{minipage}{0.45\textwidth}
\centering
\textbf{Blue Rose research says yes:}
\\[1em]
``Politically disengaged voters have become much more Republican,
    And because less-engaged voters swung away from [Democrats], an expanded
    electorate meant a more Republican electorate.''
\\[1em]
\citep{blueroseresearch:2024} (On Ezra Klein show, major professional pollsters)
\end{minipage}
%
\hfill\vline\hfill
%
\begin{minipage}{0.45\textwidth}
\textbf{\emph{On Data and Democracy} says no:}
\\[1em]
\centering
``Claims of a decisive pro-Republican shift among the overall non-voting population are
  not supported by the most reliable, large-scale post-election data currently available.''
\\[1em]
\citep{datademocracyblog:2025} (Berkeley professor co--author, major professional researchers)
\end{minipage}

\vspace{1em}

Several factors drive the disagreement:
%
\begin{itemize}
\item The problem is very hard (it's difficult to poll non--voters)
\item Different data sources
%
\begin{itemize}
    \item Blue Rose aggregates its own private data
    \item The \emph{On Data and Democracy} posts use public data, e.g.~the
        cooperative election study (CES).
\end{itemize}
%
\item \textbf{Very different statistical methods:} $\star$
\begin{itemize}
    \item Blue Rose uses Bayesian hierarchical modeling
    \item The CES uses calibration weighting
\end{itemize}
%
\end{itemize}
%
\textbf{Our work won't resolve the dispute.}  (Anyway, we'd need access to Blue Rose's private
data and modeling to even try.)

But we can form a like--to--like comparison between the methodologies.  (And hope that
Blue Rose tries our software package.)

\end{frame}


%%%%%%%%%%%%%%%%%%%%%%%%%%%%%%%%%%%%%%%%
%%%%%%%%%%%%%%%%%%%%%%%%%%%%%%%%%%%%%%%%
%%%%%%%%%%%%%%%%%%%%%%%%%%%%%%%%%%%%%%%%


