
%%%%%%%%%%%%%%%%%%%%%%%%%%%%%%%%%%%%%%%%%%%%%%%%%%%%%%%%%%%%%%%%%%%%%%%%%%%
%%%%%%%%%%%%%%%%%%%%%%%%%%%%%%%%%%%%%%%%%%%%%%%%%%%%%%%%%%%%%%%%%%%%%%%%%%%
%%%%%%%%%%%%%%%%%%%%%%%%%%%%%%%%%%%%%%%%%%%%%%%%%%%%%%%%%%%%%%%%%%%%%%%%%%%

\begin{frame}{Observations and consequences}


\begin{minipage}{0.38\textwidth}
    \ElectionData{}
\end{minipage}
\begin{minipage}{0.38\textwidth}
    \ElectionResultsGlobal{}
\end{minipage}


\pause
%
\begin{itemize}
\item We use often use models of the form $\p(\gamma, \lambda \vert \xvec)$.
% \item There may be multiple ways to define ``exchangable unit'' in a given
%       problem.
% \item[] ... But
%         $\log \p(\x_n \vert \gamma, \lambda)$
%         is often the natural model-free exchangeable unit.
% \item[] ... and we have proven that the weight dependence is non--linear.
\item Even if the error $\red{\err(\w)}$ does not vanish,
      it can still be small enough in practice.
      \item[] ... Especially given the linear approximation's huge computational advantage.
\end{itemize}

% \textbf{Can we do better in the presence of high-dimensional latent variables?}

\textbf{Preprint: }\citet{giordano:2023:bayesij} (\texttt{arXiv:2305.06466})\\
(The preprint focuses on variance estimation, the present results are found in the proofs.)
    
\end{frame}




\begin{frame}

\footnotesize

\bibliographystyle{plainnat}
% Hide the references header
% https://tex.stackexchange.com/questions/22645/hiding-the-title-of-the-bibliography/370784
\begingroup
\renewcommand{\section}[2]{}%
\bibliography{references}
\endgroup

%
\end{frame}
