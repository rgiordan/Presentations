\begin{frame}{High dimensional problems}

\question{What about when parts of the posterior don't concentrate?}

Example: \textbf{Genearlized linear model with random effects (REs)
$\lambda$ and fixed effect $\gamma$.}

% If the observations per random effect remains bounded as $N \rightarrow
% \infty$, then

\begin{tabular}{ll}
%     Parameter $\lambda$ grows in dimension with $N$.& 
%     Parameter $\gamma$ is a scalar. \\
    Marginally, $\p(\lambda \vert \xvec)$ does not concentrate. &
    Marginally, $\p(\gamma \vert \xvec)$ concentrates. \\
\end{tabular}

% \pause
% \question{Does 
% $\w_n \mapsto \expect{\p(\lambda \vert \xvec, \w_n)}{f(\lambda)} - \expect{\p(\lambda \vert \xvec)}{f(\lambda)}$ 
% become linear as $N$ grows? \\
% (Note $\p(\lambda \vert \xvec)$ does not concentrate.)}

% \textbf{Not in general. }
% Since $\p(\lambda \vert \xvec)$ doesn't concentrate, both the slope $\blue{\infl_n}$ and
% error $\red{\err(\w_n)}$ are $O(1)$ in general.
% % $\expect{\p(\lambda \vert \xvec, \w_n)}{f(\lambda)}$
% %
% $\Rightarrow$ The map
% $\w_n \mapsto \expect{\p(\lambda \vert \xvec, \w_n)}{f(\lambda)}$ is nonlinear in general.

\pause
\spskip
\question{Does $\w_n \mapsto \expect{\p(\gamma \vert \xvec, \w_n)}{f(\gamma)} - \expect{\p(\gamma \vert \xvec)}{f(\gamma)}$ 
become linear as $N$ grows?\\
(Note $\p(\gamma \vert \xvec)$ \textit{does} concentrate.)}
%

\pause
\spskip

\theorem{
\textbf{Theorem 5 of \citet{giordano:2023:bayesij}  (paraphrase): } In general, \textbf{no!}\\
%
Specifically, if $p(\lambda \vert \xvec, \gamma)$ does not concentrate, then \\
--- even if $\p(\gamma \vert \xvec)$ concentrates marginally --- \\
% in the linear approximation to $\expect{\p(\gamma \vert \xvec, \w_n)}{f(\gamma)}$,
both the slope $\blue{\infl_n}$ and the error $\red{\err(\w_n)}$ are $O_p(N^{-1})$, and so \\
$N \left(\expect{\p(\gamma \vert \xvec, \w_n)}{f(\gamma)} - \expect{\p(\gamma \vert \xvec)}{f(\gamma)} \right) 
= \blue{N \infl_n (\w_n - 1)} + \red{N \err(\w_n)}$ is nonlinear.\\
\\
However, $\err(\w_n) \rightarrow 0$ as $\cov{p(\lambda \vert \xvec, \gamma)}{\lambda} \rightarrow 0$.
}
%
\end{frame}

