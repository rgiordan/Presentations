\begin{frame}{High dimensional problems}

\question{What about when the posterior doesn't obey a BCLT?}

Example: \textbf{Poisson model with random effects (REs)
$\lambda$ and fixed effect $\gamma$.}

If the observations per random effect remains bounded as $N \rightarrow
\infty$, then

\begin{tabular}{ll}
    Parameter $\lambda$ grows in dimension with $N$.& 
    Parameter $\gamma$ is a scalar. \\
    Marginally, $\p(\lambda \vert \xvec)$ does not concentrate. &
    Marginally, $\p(\gamma \vert \xvec)$ obeys a BCLT. \\
\end{tabular}
% \begin{itemize}
% %
% \item Parameter $\lambda$ (``local'') grows in dimension with $N$.
% \item Parameter $\gamma$ (``global'') is finite-dimensional.
% \item Marginally $\p(\lambda \vert \xvec)$ does not concentrate.
% \item Marginally, $\p(\gamma \vert \xvec)$ concentrates.
% % \item Conditionally, $\p(\lambda \vert \gamma, \xvec)$ does not concentrate either.
% %
% \end{itemize}
%
% \begin{align*}
% %
% \MoveEqLeft
% \expect{\p(\lambda \vert \xvec, \w_n)}{f(\lambda)} -
% \expect{\p(\lambda \vert \xvec)}{f(\lambda)}
% \\={}&
% \blue{\infl_n} (\w_n - 1)
% &&
% + \red{\err(\w_n)}
% \\={}&
% \undernote{
% \blue{\expect{\p(\lambda \vert \xvec)}
%              {\fbar(\lambda) \ellbar_n(\lambda)}}
% }{O_p(1)}
% (\w_n - 1)
% &&
% + \frac{1}{2}
% \undernote{
% \red{
% \expect{\p(\lambda \vert \xvec, \wtil_n)}{
%     \fbar(\lambda)
%     \ellbar_n(\lambda)
%     \ellbar_n(\lambda)}
% }
% }{O_p(1)}
% (\w_n - 1)^2.
% %
% \end{align*}
%

\question{Does $\w_n \mapsto \expect{\p(\lambda \vert \xvec, \w_n)}{f(\lambda)}$ become linear as $N$ grows?}

\textbf{Not in general. }
Since $\p(\lambda \vert \xvec)$ doesn't concentrate, both the slope $\blue{\infl_n}$ and
error $\red{\err(\w_n)}$ are $O(1)$ in general.
% $\expect{\p(\lambda \vert \xvec, \w_n)}{f(\lambda)}$
%
$\Rightarrow$ The map
$\w_n \mapsto \expect{\p(\lambda \vert \xvec, \w_n)}{f(\lambda)}$ is nonlinear in general.

\spskip
\question{Does $\w_n \mapsto \expect{\p(\gamma \vert \xvec, \w_n)}{f(\gamma)}$ become linear as $N$ grows?}
%


\theorem{
\textbf{Theorem 5 of \citet{giordano:2023:bayesij}  (paraphrase): }\\
In the linear approximation to $\expect{\p(\gamma \vert \xvec, \w_n)}{f(\gamma)}$,
both the slope $\blue{\infl_n}$ and the error $\red{\err(\w_n)}$ are $O_p(N^{-1})$ when
$p(\lambda \vert \xvec, \gamma)$ does not concentrate, even if $\p(\gamma \vert \xvec)$
obeys a BCLT marginally.  In general, \textbf{the posterior
expectation does not become linear in $\w_n$ as $N$ grows.}
}

\end{frame}


%%%%%%%%%%%%%%%%%%%%%%%%%%%%%%%%%%%%%%%%%%%%%%%%%%%%%%%%%%%%%%%%%%%%%%%%%%%
%%%%%%%%%%%%%%%%%%%%%%%%%%%%%%%%%%%%%%%%%%%%%%%%%%%%%%%%%%%%%%%%%%%%%%%%%%%
%%%%%%%%%%%%%%%%%%%%%%%%%%%%%%%%%%%%%%%%%%%%%%%%%%%%%%%%%%%%%%%%%%%%%%%%%%%

\begin{frame}{Experiments}

\HighDimAccuracyGraph{}

\end{frame}
