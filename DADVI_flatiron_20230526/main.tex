\documentclass[8pt]{beamer}\usepackage[]{graphicx}\usepackage[]{color}


\usetheme{metropolis}           % Use metropolis theme
\usepackage{amsmath}
\usepackage{mathrsfs}
\usepackage{tabularx}

\usepackage{cleveref}

% For custom oversets
\usepackage{accents}

% Algorithm notation
\usepackage{algorithm}
\usepackage{algpseudocode}
% algorithmicx package

\DeclareMathOperator*{\argmax}{\mathrm{argmax}}
\DeclareMathOperator*{\argmin}{\mathrm{argmin}}
\DeclareMathOperator*{\esssup}{\mathrm{esssup}}
\DeclareMathOperator*{\essinf}{\mathrm{essinf}}
\DeclareMathOperator*{\argsup}{\mathrm{argsup}}
\DeclareMathOperator*{\arginf}{\mathrm{arginf}}

\newcommand{\trans}{\intercal}
\newcommand{\norm}[1]{\left\Vert#1\right\Vert}
\newcommand{\abs}[1]{\left|#1\right|}
\newcommand{\normal}[2]{\mathcal{N}\left(#1 \vert #2\right)}
\newcommand{\bbr}{\mathbb{R}}
\newcommand{\bbz}{\mathbb{Z}}
\newcommand{\bbc}{\mathbb{C}}
\newcommand{\rdom}[1]{\mathbb{R}^{#1}}
\newcommand{\ind}[1]{\mathbb{I}^{#1}}

\newcommand{\p}[2]{\underset{#1}{\mathbb{P}}\left(#2\right)}

\newcommand{\expect}[2]{\underset{#1}{\mathbb{E}}\left[#2\right]}
\newcommand{\var}[2]{\underset{#1}{\mathrm{Var}}\left(#2\right)}
\newcommand{\cov}[2]{\underset{#1}{\mathrm{Cov}}\left(#2\right)}
\newcommand{\sumn}{\sum_{n=1}^N}
\newcommand{\meann}{\frac{1}{N} \sumn}

\newcommand{\trace}[1]{\mathrm{trace}\left(#1\right)}
\newcommand{\diag}[1]{\mathrm{Diag}\left(#1\right)}
\newcommand{\grad}[2]{\nabla_{#1} \left. #2 \right.}
\newcommand{\gradat}[3]{\nabla_{#1} \left. #2 \right|_{#3}}
\newcommand{\fracat}[3]{\left. \frac{#1}{#2} \right|_{#3}}


\newcommand{\calset}{\mathcal{Z}}
\newcommand{\zcal}{\tilde{Z}}
\newcommand{\znew}{Z^{*}}
\newcommand{\n}{N}
\newcommand{\x}{X}
\newcommand{\y}{Y}
\newcommand{\xnew}{X^{*}}
\newcommand{\ynew}{Y^{*}}
\newcommand{\yhatnew}{\hat{\y}^{*}}
\newcommand{\z}{Z}
\newcommand{\risk}{\mathscr{R}}
\newcommand{\riskhat}{\hat{\mathscr{R}}}
\newcommand{\rupper}{\hat{\mathscr{R}}^{+}}
\newcommand{\f}{f}
\newcommand{\cfun}{\mathcal{C}}
\newcommand{\s}{\mathcal{S}}
\newcommand{\shat}{\hat{\mathcal{S}}}
\newcommand{\yhat}{\hat{\y}}
\newcommand{\loss}{\mathcal{L}}
\newcommand{\lambdahat}{\hat{\lambda}}
\newcommand{\lambdastar}{\lambda^{*}}

\input{_knitr_header}

\title{Deterministic Automatic Differentiation Variational Inference}
\author{Ryan Giordano}
\date{Sep 23rd, 2022}
\institute{Massachusetts Institute of Technology}

\begin{document}




%%%%%%%%%%%%%%%%%%%%%%%%%%%%%%%%%%%%%%%%%%%%%%%%%%%%%%%%%%%%%%%%%%%%%%%
%%%%%%%%%%%%%%%%%%%%%%%%%%%%%%%%%%%%%%%%%%%%%%%%%%%%%%%%%%%%%%%%%%%%%%%
%%%%%%%%%%%%%%%%%%%%%%%%%%%%%%%%%%%%%%%%%%%%%%%%%%%%%%%%%%%%%%%%%%%%%%%

\begin{frame}{Problem statement}

We all want to do accurate Bayesian inference quickly:
%
\begin{itemize}
    \item In terms of compute (wall time, model evaluations, parallelism)
    \item In terms of analyst effort (tuning, algorithmic complexity)
\end{itemize}
%
\textbf{Markov Chain Monte Carlo (MCMC)} can be
straightforward and accurate but slow.

\vspace{-0.5em}
\hrulefill

\textbf{Black Box Variational Inference (BBVI)} can be faster alternative to MCMC.
%
\textbf{But...}

%
\begin{itemize}
    \item BBVI is cast as an optimization problem with an intractable objective $\Rightarrow$
    \item Most BBVI methods use \textbf{stochastic gradient} (SG) optimization $\Rightarrow$
    %
    \begin{itemize}
        \item SG algorithms can be hard to tune
        \item Assessing convergence and stochastic error can be difficult
        \item SG optimization can perform worse than second-order methods on tractable objectives
    \end{itemize}
    %
    \item Many BBVI methods employ a \textbf{mean-field (MF) approximation} $\Rightarrow$
    %
    \begin{itemize}
        \item Posterior variances are poorly estimated
    \end{itemize}
    %
\end{itemize}
%
\vspace{-0.5em}
\hrulefill

\textbf{Our proposal:}
replace the intractable BBVI objective with a fixed approximation.
%
\begin{itemize}
    \item Better optimization methods can be used (e.g. true second-order methods)
    \item Convergence and approximation error can be assessed directly
    \item Can correct posterior covariances with linear response covariances
    \item This technique is well-studied (but there's still work to do in the context of BBVI)
\end{itemize}
%
$\Rightarrow$
\textbf{Simpler, faster, and better BBVI posterior approximations ... in some cases.}
%
\end{frame}



\begin{frame}{Outline}
%
\begin{itemize}
    \item BBVI Background and our proposal
    \begin{itemize}
        \item Automatic differentiation variational inference (ADVI) (a BBVI method)
        \item Our approximation: ``Deterministic ADVI'' (DADVI)
        \item Linear response (LR) covariances
        \item Estimating approximation error
    \end{itemize}
    \item Experimental results: DADVI vs ADVI
    \begin{itemize}
        \item DADVI converges faster than ADVI, and requires no tuning
        \item DADVI's posterior mean estimates' accuracy are comparable to ADVI
        \item DADVI+LR provides more accurate posterior variance estimates than ADVI
        \item DADVI provides accurate estimates of its own approximation error
        \item ADVI often results in better objective function values (eventually)
    \end{itemize}
    \item Why don't we do DADVI all the time?
    \begin{itemize}
        \item DADVI fails for expressive BBVI approximations (e.g. full-rank ADVI)
        \item Pessimistic dimension dependence results from optimization theory
        \item ...which may not apply in certain BBVI settings.
    \end{itemize}
\end{itemize}
%
\end{frame}


\end{document}
