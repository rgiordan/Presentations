



%%%%%%%%%%%%%%%%%%%%%%%%%%%%%%%%%%%%%%%%%%%%%%%%%%%%%%%%%%%%%%%%%%%%%%%%%%%
%%%%%%%%%%%%%%%%%%%%%%%%%%%%%%%%%%%%%%%%%%%%%%%%%%%%%%%%%%%%%%%%%%%%%%%%%%%
%%%%%%%%%%%%%%%%%%%%%%%%%%%%%%%%%%%%%%%%%%%%%%%%%%%%%%%%%%%%%%%%%%%%%%%%%%%=


\begin{frame}[t]{Exchangeable units.  (A contradiction?)}
    %
    \vspace{-1em}
    \begin{align*}
    %
    &\text{\textbf{Negative binomial observations.}}
    &
    &\text{\textbf{Poisson observations with random effects.}}
    \\
    &\text{\textbf{Asymptotically linear in $\w$.}}
    &
    &\text{\textbf{Asymptotically non-linear in $\w$.}}
    %
    %
    \end{align*}
    
    \onslide<2->{
    \begin{center}
        % \large
    With Gamma REs, one RE per observation, and appropriate prior parameters,\\
    these are the same model, with the same $\p(\gamma \vert \xvec)$.
    
    % \spskip
    \textbf{Is $\expect{\p(\gamma \vert \xvec; \w)}{\gamma}$
    linear in the \only<1-2>{data weights}\only<3->{\red{data weights}}
    or not?}
    }
    
    
    \end{center}
    %
\end{frame}


%%%%%%%%%%%%%%%%%%%%%%%%%%%%%%%%%%%%%%%%%%%=


\begin{frame}[t]{Exchangeable units.  (A contradiction?)}
    %
    \vspace{-1em}
    \begin{align*}
    %
    &\text{\textbf{Negative binomial observations.}}
    &
    &\text{\textbf{Poisson observations with random effects.}}
    \\
    &\text{\textbf{Asymptotically linear in $\w$.}}
    &
    &\text{\textbf{Asymptotically non-linear in $\w$.}}
    \end{align*}
    %
    \begin{center}
        % \large
        With Gamma REs, one RE per observation, and appropriate prior parameters,\\
        these are the same model, with the same $\p(\gamma \vert \xvec)$.
        
    % \spskip
    \textbf{Is $\expect{\p(\gamma \vert \xvec; \w)}{\gamma}$
    linear in the \red{data weights}
    or not?}
    
    \spskip

    \textbf{\red{Trick question!}}  We weight a log likelihood
    contribution, not a datapoint.

    \begin{align*}
        &\log \p(\xvec \vert \gamma; \w^{m}) =
            \sumn \w^{m}_n 
            \textcolor{orange}{\log \p(\x_n \vert \gamma)}
        &
        &\log \p(\xvec \vert \gamma, \lambda; \w^{c}) =
            \sumn \w^{c}_n 
            \textcolor{purple}{\log \p(\x_n \vert \lambda, \gamma)}
        %
    \end{align*}
        

    % % \spskip
    \textbf{The two weightings are not equivalent in general.}

    \question{What is the right exchangeable unit for a particular problem?}
    
    \end{center}
    %
\end{frame}

%%%%%%%%%%%%%%%%%%%%%%%%%%%%%%%%%%%%%%%%%%%%%%%%%%%%%%%%%%%%%%%%%%%%%%%%%%%
%%%%%%%%%%%%%%%%%%%%%%%%%%%%%%%%%%%%%%%%%%%%%%%%%%%%%%%%%%%%%%%%%%%%%%%%%%%
%%%%%%%%%%%%%%%%%%%%%%%%%%%%%%%%%%%%%%%%%%%%%%%%%%%%%%%%%%%%%%%%%%%%%%%%%%%

\begin{frame}{Exchangeable units: Experimental results revisited}
    Our results were actually computed on \textbf{identical datasets}
    with $G = N$ and $g_n=n$.

    \begin{center}
    \begin{minipage}{0.34\textwidth}
        Uses $\log \p(\x_n \vert \gamma)$:\\
        $\infl_n = \expect{\p(\gamma \vert \xvec)}{\gammabar \ellbar_n(\gamma)}$

        \onslide<2->{
        \spskip
        Not easily computable from\\
        $\gamma, \lambda \sim \p(\gamma, \lambda \vert \xvec)$\\
        in general.
        }
    \end{minipage}
    \begin{minipage}{0.65\textwidth}
        \LowDimAccuracyGraph{}
        % \HighDimAccuracyGraph{}
    \end{minipage}
    %

    \begin{minipage}{0.34\textwidth}
        Uses $\log \p(\x_n \vert \gamma, \lambda)$:\\
        $\infl_n = \expect{\p(\gamma,\lambda \vert \xvec)}{\gammabar \ellbar_n(\gamma, \lambda)}$

        \onslide<2->{
        \spskip
        Easily computable from\\
        $\gamma, \lambda \sim \p(\gamma, \lambda \vert \xvec)$.
        }

        \onslide<3->{
        \spskip
        May still be useful when $\p(\lambda \vert \xvec)$
        is {\em somewhat} concentrated.
        }

    \end{minipage}
    \begin{minipage}{0.65\textwidth}
        %\LowDimAccuracyGraph{}
        \HighDimAccuracyGraph{}
    \end{minipage}
    %
\end{center}
    %
\end{frame}




% \begin{frame}{Observations and consequences}
%     \ManyPlotsOne{}
% \end{frame}

% \begin{frame}{Observations and consequences}
%     \ManyPlotsTwo{}
% \end{frame}

%%%%%%%%%%%%%%%%%%%%%%%%%%%%%%%%%%%%%%%%%%%%%%%%%%%%%%%%%%%%%%%%%%%%%%%%%%%
%%%%%%%%%%%%%%%%%%%%%%%%%%%%%%%%%%%%%%%%%%%%%%%%%%%%%%%%%%%%%%%%%%%%%%%%%%%
%%%%%%%%%%%%%%%%%%%%%%%%%%%%%%%%%%%%%%%%%%%%%%%%%%%%%%%%%%%%%%%%%%%%%%%%%%%

\begin{frame}{Observations and consequences}


% \begin{minipage}{0.48\textwidth}
%     \ElectionData{}
% \end{minipage}
% \begin{minipage}{0.48\textwidth}
%     \ElectionResultsGlobal{}
% \end{minipage}

\begin{itemize}
\item For finite--dimensional models which concentrate asymptotically:
\begin{itemize}
    \item Posterior expectations are approximately linear in data weights
    \item The linearized variance estimate (infinitesimal jackknife) is consistent
    \item The residual of the von Mises expansion vanishes
\end{itemize}
\item For high--dimensional models which marginally concentrate only asymptotically:
\begin{itemize}
    \item Posterior expectations are not approximately linear in data weights
    \item The linearized variance estimate (infinitesimal jackknife) is inconsistent
    \item The residual of the von Mises expansion does not vanish
    \item Even if the error $\red{\err(\w)}$ does not vanish,
            it can still be small enough in practice.
    \item[] ... Especially given the linear approximation's huge computational advantage.
\end{itemize}
\end{itemize}

\pause
\begin{itemize}
    \item When the weighting is linear, there are many other applications:
    \begin{itemize}
        \item Cross-validation
        \item Conformal inference
        \item Identification of influential subsets
    \end{itemize}
    \item When the weighting is non--linear, the inconsistency results 
    should apply more widely:
    \begin{itemize}
        \item The EM algorithm
        \item The nonparametric bootstrap
        \item Local prior sensitivity measures
    \end{itemize}
\end{itemize}

% \textbf{Can we do better in the presence of high-dimensional latent variables?}
\pause
\textbf{Preprint: }\citet{giordano:2023:bayesij} (\texttt{arXiv:2305.06466})\\
(Major update in progress, coming soon.)
    
\end{frame}




\begin{frame}{References}

\footnotesize

\bibliographystyle{plainnat}
% Hide the references header
% https://tex.stackexchange.com/questions/22645/hiding-the-title-of-the-bibliography/370784
\begingroup
\renewcommand{\section}[2]{}%
\bibliography{references}
\endgroup

%
\end{frame}
