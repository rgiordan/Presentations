\documentclass{article}
\usepackage{fancyhdr}


\pagestyle{fancy}
\fancyhf{}
\rhead{Ryan Giordano}
\lhead{Research Statement}
\rfoot{Page \thepage}

\usepackage{tabularx}

% \usepackage{fancyhdr}
% \pagestyle{fancy}

% \fancyfoot{}
% \fancyfoot[C]{Rough draft---do not distribute}

\usepackage{etoolbox}


\usepackage{microtype}
\usepackage{graphicx}
\usepackage{subfigure}
\usepackage{booktabs} % for professional tables
\usepackage{xcolor}
\usepackage[hidelinks=True]{hyperref}
\usepackage{xargs}[2008/03/08]

% Documentation
% http://ftp.math.purdue.edu/mirrors/ctan.org/macros/latex/contrib/refstyle/refstyle.pdf
\usepackage{refstyle}
\usepackage{varioref} % Use refstyle instead of varioref directly.

\usepackage{amsmath}
\usepackage{amssymb}
\usepackage{amsfonts}
\usepackage{amsthm}
\usepackage{mathrsfs} % For mathscr
\usepackage{mathtools}

\usepackage[authoryear]{natbib}
\bibliographystyle{apalike}

\usepackage{geometry}
\geometry{margin=1.5in}

\usepackage{enumitem}
\setlist{nolistsep}

\usepackage{geometry}
%\geometry{margin=1.2in}
\geometry{top=1.3in}
\geometry{left=1.5in}
\geometry{right=1.5in}

 \setlength{\parskip}{1em}

\title{Ryan Giordano COVID19 Impact Statement}

\author{
  Ryan Giordano \\ \texttt{rgiordan@mit.edu }
}

\begin{document}

\begin{minipage}[t]{0.7\textwidth}
\hspace{-2em} % Easier than doing it right!
{\bf \LARGE COVID19 Impact Statement}\\
\end{minipage}
\begin{minipage}[t]{0.3\textwidth}
%    \begin{flushright}
        \hspace{-1em} % Easier than doing it right!
        {\LARGE Ryan Giordano}
%    \end{flushright}
\end{minipage}


My postdoc at MIT began in Summer 2019, eight months before the COVID19 pandemic
broke out in the United States, and the resulting disruptions to childcare for
my young son have had a highly detrimental effect on my research output.
Throughout the pandemic, my wife’s work responsibilities and schedule (remote
teaching and service in two departments, alongside ordinary research) have been
relatively inflexible compared to the responsibilities of my postdoc. As a
consequence, I have taken on the bulk of the extra childcare during the last two
years, which severely constrained the time available for research and writing.
Additionally, the lack of a quiet, isolated workspace in our home, the
impossibility of visiting MIT in person, and the time constraints the pandemic
has placed on collaborators have all exacerbated the challenges to my ability to
be productive.

Our son, Kai, was 18 months old in March 2020, when we lost our childcare. We
did not regain any outside care until August 2020, when our nanny returned to
taking care of him part time. Our plan had been to enroll Kai in full-time
daycare when he turned two, in August 2020, but many childcare centers had no
space, and, prior to the development of the vaccines, it seemed too risky. We
were also unable to find a childcare pod to join, in part because of the COVID19
risks associated with seeing my mother in law, who was required for work to
interact with a large number of people, and who was, due to her own work
demands, unable to provide any help with childcare. We have no other family in
the area.

In the 2020–2021 academic year, our nanny continued to provide part time
childcare (she was unable to work full time, we were reluctant to introduce new
people into our quarantine bubble, and it would have been difficult for us to
afford a full time nanny), and I took on the bulk of the rest of the childcare.
In August 2021 Kai finally began preschool at the age of three. Though our
intention was for him to attend a full day, his complete lack of experience in a
childcare setting meant that he was behaviorally unprepared to join the full-day
program, and the school requested that we have him participate only in the
morning program. In November we tried to add three afternoons a week but were
again requested to reduce it to a single afternoon (when there was extra staff)
due to behavioral unreadiness. Thanks to increased staffing at the preschool we
will be able to have 3 full days and 2 half days of care starting in March 2022,
and we expect to be able to reach full time hours by the summer.

There is good reason that the professional difficulties associated with the
COVID19 pandemic will not be repeated, even as we have our second (and final)
baby due in April 2022. At this stage of the pandemic, it seems likely that any
disruptions to childcare can be minimized. Kai is now adapting well to school,
and is guaranteed a place next year in the full-day program.  We expect to be
able to have more childcare for our second baby. Additionally, now that my wife
is tenured, we are planning for me to be the one who works relatively more this
time around (she will also be on ASMD for the first year of our second baby’s
life, and is planning to take a sabbatical the year after that).  Finally, the
ability to see collaborators in person and to have my own workspace will surely
have a significant and positive impact on my productivity.



\end{document}
