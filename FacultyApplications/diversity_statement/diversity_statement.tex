\usetheme{metropolis}           % Use metropolis theme
\usepackage{amsmath}
\usepackage{mathrsfs}
\usepackage{tabularx}

\usepackage{mathtools}

\usepackage{cleveref}

% For custom oversets
\usepackage{accents}

% Algorithm notation
\usepackage{algorithm}
\usepackage{algpseudocode}
% algorithmicx package

% https://tex.stackexchange.com/questions/5017/center-column-with-specifying-width-in-table-tabular-enviroment
\usepackage{array}
\newcolumntype{x}[1]{>{\centering\let\newline\\\arraybackslash\hspace{0pt}}p{#1}}

\usepackage{natbib}
% \bibliographystyle{abbrvnat}
\usepackage{enumitem}
\setlist{nolistsep}

\usepackage{geometry}
%\geometry{margin=1.2in}
\geometry{top=0.5in}
\geometry{left=1.0in}
\geometry{right=1.0in}


\title{Ryan Giordano Diversity Statement}

\author{
  Ryan Giordano \\ \texttt{rgiordan@mit.edu }
}

\begin{document}

\begin{minipage}[t]{0.5\textwidth}
\hspace{-2em} % Easier than doing it right!
{\bf \LARGE Diversity Statement}\\
\end{minipage}
\begin{minipage}[t]{0.5\textwidth}
%    \begin{flushright}
        \hspace{8em} % Easier than doing it right!
        {\LARGE Ryan Giordano}
%    \end{flushright}
\end{minipage}




I have found that one over-privilege that accrues to me as a white man is to
have my opinion taken seriously by default in many situations, even if I do not
know what I am talking about.  Given this, if I am to live up to my
responsibility to help create a more just and inclusive society, I have to
actively work to ensure that my actions and words respond to the experiences of
the under-privileged, and not my own limited perspective. At the same time,
surrendering my own privilege does not simply mean passively taking no action.
How to balance the need to act, but in a way that does not simply amplify my own
privileged viewpoint?   My answer is to try to ``take action in response to
expressed, felt need.''


\paragraph{Expressed, Felt Need.}
%
Padre Greg of the San Lucas Tolimán mission in Guatemala taught me a wonderful
phrase which, to me, encapsulates the importance of listening before helping: to
act on behalf of another community only in response to ``expressed, felt need,''
meaning need which is spoken openly, not guessed at, and which is sincerely
felt, not based on the expectations of the powerful.  As a education Peace Corps
volunteer in Kazakhstan, I witnessed extreme versions of institutional failures
to listen. For example, a development organization provided the elementary
school I worked in with an expensive computed lab.  When an organization
representative visited (once, for a single day), the lab was unlocked, and a
show lesson was put on for their benefit.  Otherwise, the computers sat entirely
unused. The organization meant well, but responded to its own beliefs of what
our school needed, rather than responding to expressed, felt need.

Though the examples are extreme in international development, it is just as
important for university faculty and students to listen before taking action.
For example, I was asked by the Statistics Department at UC Berkeley to serve on
a student panel on increasing diversity and representation in the department.
Those of us on the panel had clear ideas of what would help, but, by their
nature, such panels usually consist of graduate students who are already
succeeding. Recognizing this bias in ourselves, we took it upon ourselves to
conduct open-ended interviews with all the grad students in the department to ask
what would improve the UC Berkeley grad school experience.  We took especially
seriously the recommendations of under-represented minorities and women. The
results were eye-opening.  None of the panel's original ideas (such as a student
paper seminar) were very popular, but there was broad and vocal support for
other reforms that we hadn't initially considered (such as clearer requirements
for the qualifying exam).
% The work to collate responses not only exposed
% weakness in our own assumptions, it also signaled to the department that we were
% serious enough about diversity to do a considerable amout of work, and provided
% numerical support to bolster our recommendations.

\paragraph{Taking action.}
%
In response to expressed, felt need, university faculty can take action to
promote equity and justice within the university in at least two qualitatively
distinct ways.  One is through personal relationships, such as via mentorship,
advising, and teaching.  The other is through official programs, such as
departmental outreach or diversity events.

I believe that personal relationships are the best opportunity for real social
change, and university faculty have the opportunity to forge many meaningful
relationships with students.  Professors can help students feel that they belong
by cultivating an inclusive, welcoming classroom environment. Students are more
likely to remain in a program of study when they have a sense of belonging and
community, which professors can facilitate by encouraging collaborative work
both in and out the classroom. For example, one of the initiatives that came out
of the aforementioned UC Berkeley diversity panel was a strengthened student
mentorship program, which I participated in as a mentor.

Finally, I believe that official programs to promote diversity, such as
department panels, outreach programs, and discussion-based events can be
effective when done well, in a way that engages supportive, productive
conversation, articulates and reinforces community standards, and leads to
concrete institutional changes.  I also believe that such efforts are most
fruitful when led by faculty who are members of the under-represented group whom
they are intended to serve, which, regrettably leads to an extra burden for such
faculty. In recognition of this, the more privileged faculty can play a useful
subordinate role without usurping leadership.  As an example from my own
experience, in response to the George Marcy sexual harassment scandal at UC
Berkeley, two female PhD students in statistics wanted to organize a
department-wide ``diversity lunch'' to discuss issues related to sexism in the
statistics field, and they asked me to help. I provided as much material help as
I could to ease the burden on the organizers (emailing, gathering resources,
presenting). And (in the words of the organizers) my active presence served to
signal that sexism was not simply a ``women's issue,'' but a problem that
affects and should engage everyone.  This is always the case with diversity,
inclusion, and justice!


\end{document}
