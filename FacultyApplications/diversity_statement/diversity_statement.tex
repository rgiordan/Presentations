\documentclass{article}
\usepackage{fancyhdr}


\pagestyle{fancy}
\fancyhf{}
\rhead{Ryan Giordano}
\lhead{Research Statement}
\rfoot{Page \thepage}

\usepackage{tabularx}

% \usepackage{fancyhdr}
% \pagestyle{fancy}

% \fancyfoot{}
% \fancyfoot[C]{Rough draft---do not distribute}

\usepackage{etoolbox}


\usepackage{microtype}
\usepackage{graphicx}
\usepackage{subfigure}
\usepackage{booktabs} % for professional tables
\usepackage{xcolor}
\usepackage[hidelinks=True]{hyperref}
\usepackage{xargs}[2008/03/08]

% Documentation
% http://ftp.math.purdue.edu/mirrors/ctan.org/macros/latex/contrib/refstyle/refstyle.pdf
\usepackage{refstyle}
\usepackage{varioref} % Use refstyle instead of varioref directly.

\usepackage{amsmath}
\usepackage{amssymb}
\usepackage{amsfonts}
\usepackage{amsthm}
\usepackage{mathrsfs} % For mathscr
\usepackage{mathtools}

\usepackage[authoryear]{natbib}
\bibliographystyle{apalike}

\usepackage{geometry}
\geometry{margin=1.5in}

\usepackage{enumitem}
\setlist{nolistsep}

\usepackage{geometry}
%\geometry{margin=1.2in}
\geometry{top=0.9in}
\geometry{left=1.4in}
\geometry{right=1.4in}


\title{Ryan Giordano Diversity Statement}

\author{
  Ryan Giordano \\ \texttt{rgiordan@mit.edu }
}

\begin{document}

\begin{minipage}[t]{0.5\textwidth}
\hspace{-2em} % Easier than doing it right!
{\bf \LARGE Diversity Statement}\\
\end{minipage}
\begin{minipage}[t]{0.5\textwidth}
%    \begin{flushright}
        \hspace{8em} % Easier than doing it right!
        {\LARGE Ryan Giordano}
%    \end{flushright}
\end{minipage}


%\maketitle

I will describe some of the experiences that have shaped my views on
institutional diversity and my role promoting it, concrete actions I have taken,
and how I would view my responsibilities as academic faculty.  As someone with
considerable social privelege (I'm a white, cis-gendered man), my responsibility
is first to listen, actively and with humility, and then to take action, both
through my personal relatinships as a teacher or mentor, and in in offical
projects such as departmental panels, diversity events, or outreach, ideally as
a subordinate to a representative of a targeted under-represented group.


\section{Expressed, Felt Need}

Padre Greg of the San Lucas Tolimán mission in Guatemala taught me a wonderful
phrase which, to me, encapsulates the importance of listening.  Padre Greg
described to us volunteers his story arriving in Guatemala as a young priest
with all sorts of plans to improve the lives of his parishioners, only to learn
the difficult lesson that his ideas were not always the best. Over the years, he
aquired the wisdom to act only in response to ``expressed, felt need:'' that is,
need which is spoken openly, not guessed at, and which is sincerely felt, not
based on the expectations of the powerful.

One aspect of privelege is having one's opinion taken seriously by default.
Consequently, even well-intentioned priveleged people can do harm by
aggressively promoting \emph{their} solutions, rather than listening with
humility to the needs and experience of under-priveleged.  Coming into contact
with expressed, felt need requires real effort. As a Peace Corps volunteer in
Kazakhstan, I witnessed miriad extreme versions of failure to listen. For
example, an NGO provided the elementary school I worked in with an expensive
computed lab.  When the NGO representative visited (once, for a single day), the
lab was unlocked, and a show lesson was put on for their benefit.  Otherwise,
the computers sat entirely unused for many reasons, including for fear of their
being broken, for lack of teachers to teach computer skills, and for lack of
time in the government-mandated curriculum.

Less extreme disconnects between the priveleged and under-priveleged all the
time, and the priveleged must make conscious effort to check their ego, to
listen, and to learn.  As an example of listening that I consider successful, I
was asked by the Statistics Department at UC Berkeley to serve on a student
panel on increasing diversity and representation in the department.  Those of us
on the panel had clear ideas of what would help, but, by their nature, such
panels usually consist of graduate students who are already succeeding.
Recognizing this bias in ourselves, we took it upon ourselves to coduct
open-ended interviews with all the grad students in the department to ask what
would improve the UC Berkeley grad school experience, especially for
under-represented minorities and women. The results were eye-opening.  None of
the panel's original ideas (such as a student paper seminar) were very popular,
but there was broad and vocal support for other reforms (such as clearer
requirements for the qualifying exam), and . The work to collate responses not
only exposed weakness in our own assumptions, it also signaled to the department
that we were serious enough about diversity to do a considerable amout of work,
and provided numerical support to bolster our recommendations.




\section{Taking Action}

University faculty can take action to promote equity and justice in at least two
qualitatively distinct ways.  One is through official programs, such as
departmental outreach or diversity events.  The other is through personal
relationships, such as via mentorship, advising, and teaching.

Official programs to promote diversity, such as department panels, outreach
programs, and diversity events are best led by the group whom they are intended
to serve.  However, the privelged can often play a useful subordinate role.  In
response to the Geroge Marcy sexual harassment scandal at UC Berkeley, two
female PhD students in statistics wanted to organize a department-wide
``diversity lunch'' to discuss issues related to sexism in the statistics field,
and they asked me to help.  Importantly, the impetus and key decisions were
entirely theirs.  I could of course provide material help (emailing, gathering
resources, presenting, etc.). But my active presence served an additional
purpose: to signal that sexism was not simply a ``women's issue,'' but a problem
that affects and should engage everyone.

It is my belief that personal relationships are the best opportunity for real
social change, and university faculty have the opportunity to forge many
meaningful relationships with students.  Everything begins with listening, and
professors can often provide a lot of help simply by acting as empathetic
sounding boards for struggling students.  The principal factor that is
negatively associated with URM dropout is a community, and professors can help
ensure that students, especially under-represnted students, feel they have
somewhere helpful to turn when they struggle.  (One of the initiatives that came
out of the aforementioned UC Berkeley diversity panel was a strengthened student
mentorship program, which I actively participated in as a mentor.)  Good
pedagogy benefits everyone, but can particularly under-represented groups. As I
discuss in my teaching statement, I make sure that expectations are as
unambiguous as possible, that the students receive frequent feedback, and employ
different modes of evaluation (papers, tests, in-class presentation, and so on).
And, of course, teachers who carefully monitor their students progress have the
opportunity to intervene proactively if they notice under-represented students
struggling.



\end{document}
