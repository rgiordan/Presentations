\documentclass{article}
\usepackage{fancyhdr}


\pagestyle{fancy}
\fancyhf{}
\rhead{Ryan Giordano}
\lhead{Research Statement}
\rfoot{Page \thepage}

\usepackage{tabularx}

% \usepackage{fancyhdr}
% \pagestyle{fancy}

% \fancyfoot{}
% \fancyfoot[C]{Rough draft---do not distribute}

\usepackage{etoolbox}


\usepackage{microtype}
\usepackage{graphicx}
\usepackage{subfigure}
\usepackage{booktabs} % for professional tables
\usepackage{xcolor}
\usepackage[hidelinks=True]{hyperref}
\usepackage{xargs}[2008/03/08]

% Documentation
% http://ftp.math.purdue.edu/mirrors/ctan.org/macros/latex/contrib/refstyle/refstyle.pdf
\usepackage{refstyle}
\usepackage{varioref} % Use refstyle instead of varioref directly.

\usepackage{amsmath}
\usepackage{amssymb}
\usepackage{amsfonts}
\usepackage{amsthm}
\usepackage{mathrsfs} % For mathscr
\usepackage{mathtools}

\usepackage[authoryear]{natbib}
\bibliographystyle{apalike}

\usepackage{geometry}
\geometry{margin=1.5in}

\usepackage{enumitem}
\setlist{nolistsep}

\usepackage{geometry}
%\geometry{margin=1.2in}
\geometry{top=1.0in}
\geometry{left=1.1in}
\geometry{right=1.1in}

\title{Ryan Giordano Research Statement}

\author{
  Ryan Giordano \\ \texttt{rgiordan@mit.edu }
}

\begin{document}

\begin{minipage}[t]{0.5\textwidth}
\hspace{-2em} % Easier than doing it right!
{\bf \LARGE Research Statement}\\
\end{minipage}
\begin{minipage}[t]{0.5\textwidth}
%    \begin{flushright}
        \hspace{8em} % Easier than doing it right!
        {\LARGE Ryan Giordano}
%    \end{flushright}
\end{minipage}

Many researchers would be concerned if they learned that some core conclusion of
their statistical analysis---such as the sign or statistical significance of
some key effect---could be overturned by removing a small fraction of their
data. Such non-robustness would be particularly concerning if the data were not
actually drawn randomly from precisely the population of interest, or if the
model may have been misspecified---circumstances that often occur in the social
sciences. For example, a study of microcredit in Mexico on 16,561 households
measured a negative but statistically insignificant effect of microcredit
\citep{angelucci:2015:microcredit}.  However, in recent work, my co-authors and
I show that one can make the estimated effect of microcredit positive and
statistically significant by removing just 15 households from the analysis,
reversing the paper's qualitative conclusions by removing only 0.1\% of the
data.  Since there are a combinatorially large number of ways to leave 15
datapoints out of 16,561 (over $10^{51}$), finding such influential subsets by
brute force would be impossible.  I circumvent this difficulty by forming a
\emph{linear approximation} to the dependence of the estimator on the dataset.
My technique works for a wide class of commonly used estimators, providing a
fast, automatic tool for identifying small but influential subsets with
finite-sample guarantees in many applied problems.

In fact, my research shows that many desirable but computationally demanding
data analysis tasks are similarly amenable to automatic approximation, providing
fast, easy-to-use computational tools for assessing whether small changes to
modeling assumptions or datasets can meaningfully alter an analyst's substantive
conclusions.  As I detail below, my work encompasses Bayesian approaches (such
as measuring robustness to prior specification), frequentist approaches (such as
fast approximations to leave-one-out cross validation), and combinations of the
two (such as measuring the robustness of Markov chain Monte Carlo procedures to
data resampling). For the remainder of this statement, I will describe some of
my specific past and future projects, emphasizing the ways in which I update
classical results for a modern computational environment using intuitive,
relevant theory, and with motivation drawn from practical applications.

\paragraph{Approximate cross validation.}
%
The error or variability of machine learning algorithms is often assessed by
repeatedly re-fitting a model with different weighted versions of the observed
data; cross-validation (CV) can be thought of as a particularly popular example
of this technique.
%
In \citet{giordano:2019:ij}, I use a linear approximation to the dependence of
the fitting procedure on the weights, producing results that can be faster by an
order of magnitude than repeated re-fitting. I provide explicit finite-sample
error bounds for the approximation in terms of a small number of simple,
verifiable assumptions.  My results apply whether the weights and data are
stochastic or deterministic, and so can be used as a tool for proving the
accuracy of the approximation on a wide variety of problems. As a
corollary, I state mild regularity conditions under which the approximation
consistently estimates true leave-$k$-out cross-validation for any fixed $k$. I
demonstrate the accuracy of the approximation on a range of simulated and real
datasets, including an unsupervised clustering problem from genomics.

\paragraph{Prior sensitivity for discrete Bayesian nonparametrics.}
%
% From BNP_sensitivity/writing/NIPS_2018_BNP_workshop
A central question in many probabilistic clustering problems is how many
distinct clusters are present in a particular dataset and which observations
cluster together. Discrete Bayesian nonparametric (BNP) mixture models address
this question by placing a generative process on cluster assignment, making the
number and composition of distinct clusters amenable to Bayesian inference.
However, like all Bayesian approaches, BNP requires the specification of a
prior, and this prior may favor different numbers and types of posterior
clusters.

In \citet{giordano:2021:bnpsensitivity}, I derive and analyze prior sensitivity
measures for variational Bayes (VB) approximations in general, with a practical
focus on discrete BNP models. Unlike much previous work on local Bayesian
sensitivity for BNP (e.g. \citet{Basu:2000:BNP_robustness}), I pay special
attention to the ability of the sensitivity measures to \emph{extrapolate} to
different priors, rather than treating the sensitivity as a measure of
robustness \textit{per se}.  I state conditions under which VB approximations
are Fr{\'e}chet differentiable functions of prior densities in a particular
vector space, while also proving that VB approximations are in fact
\emph{non-differentiable} in another wide class of vector space embeddings
popular in the classical Bayesian robustness literature.
My co-authors and I apply the sensitivity measures to a number of real-world
problems, including an unsupervised clustering problem from genomics using
fastSTRUCTURE.  We demonstrate that the approximation is accurate, orders of
magnitude faster than re-fitting, and capable of detecting meaningful prior
sensitivity in quantities of practical interest.

\paragraph{Uncertainty propagation in mean-field variational Bayes.}
%
Mean-field Variational Bayes (MFVB) is an approximate Bayesian posterior
inference technique that is increasingly popular due to its fast runtimes on
large-scale scientific data sets. For example, in
\citet{regier:2019:cataloging}, my co-authors and I use MFVB to construct an
approximate posterior for the identity of every astronomical object in 55TB
of image data from the Sloan Digital Sky Survey.  However, even when MFVB provides
accurate posterior means for certain parameters, it often mis-estimates
variances and covariances due to its inability to propagate Bayesian uncertainty
between statistical parameters.

In \citet{giordano:2015:linear, giordano:2018:covariances}, I derive a simple
formula for the effect of infinitesimal perturbations on MFVB posterior means,
thus providing improved covariance estimates and greatly expanding the practical
usefulness of MFVB posterior approximations. My approach builds on a result from
the classical Bayesian robustness literature relating posterior covariances to
the derivatives of posterior expectations, and includes the classical Laplace
approximation as a special case. In experiments on simulated and real-life
datasets,  including models from ecology, the social sciences, and on a massive
internet advertising dataset, I demonstrate that my method is simple, general,
and fast, providing accurate posterior uncertainty estimates and robustness
measures with runtimes that can be an order of magnitude faster than MCMC.

\paragraph{The empirical influence function.}
%
Much of my work (particularly \citet{giordano:2019:ij, giordano:2020:amip,
giordano:2021:bayesij}) has strong connections to the classical theory of von
Mises expansions and the closely related concept of the influence function,
which measures the effect of individual datapoints on an estimator
\citep{mises:1947:asymptotic, reeds:1976:thesis, hampel:1986:robustbook,
serfling:2009:approximation}.  But my focus on the influence function
evaluated at the observed data---i.e., the ``empirical influence function''
(EIF)---stands in contrast with much of the classical literature, which studies
the asymptotic behavior of estimators via their (unobserved) limiting influence
function.
%
In our present age of automatic differentiation, large datasets, and complex
models, I believe that the EIF will continue to provide practical benefits and
is relatively under-studied.

In \citet{giordano:2019:hoij}, I show that higher-order EIFs can be
easily and automatically evaluated and analyzed for M-estimators at a
computational cost comparable to the first-order EIF---that of forming and
factorizing a Hessian matrix of second-order derivatives.
%
Thus, the EIF ``amortizes'' the cost of evaluating an M-estimator large number
of alternative datasets: by paying a large fixed price up front (approximately
computing and factorizing a Hessian matrix), one can cheaply approximate
M-estimators at a very large number of alternative datasets.  Natural
applications of the idea include approximating the bootstrap-after-bootstrap,
evaluating the sampling properties of cross-validation, and computing
higher-order jackknife bias correction.

\newpage

\bibliography{references}
\bibliographystyle{plainnat}

\end{document}
