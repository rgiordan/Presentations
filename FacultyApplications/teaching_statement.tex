\documentclass{article}
\usepackage[utf8]{inputenc}
%\usepackage{multirow}
%\usepackage{multicol}
%\usepackage{array,booktabs}

\title{Teaching Statement}
\author{Ryan Giordano}
\date{September 2021}

\begin{document}

\maketitle

I think of teaching, like writing and oral presentations, to be an important
part of my research rather than a distraction from it.  Careful, empathetic
exposition of complex ideas is, of course, necessary for reasearch to reach a
wide audience.  But, even more, the very act of teaching well forces the teacher
to refine and examing their ideas.  The benefits of thoughtful teaching are
particularly great in a field like statistics, where it can be much harder to
truly understand the fundamental concepts (hypothesis testing, Bayes rule,
randomness in the real world) than it is to simply mechanically apply the
mathematical tools of the trade.

Accordingly, I have had a lifelong passion for teaching.  In addition to working
as a university-level teaching assistant both as an undergraduate in engineering
and as a doctoral student in statistics (for which I received a university
teaching award), I was a full-time teacher for two years at the middle and
elemenatary school level as a Peace Corps volunteer in Kazakhstan teaching math
and ESL.  During the Peace Corps, I also wrote a textbook and organized and
participated multiple extracurricular classes for my community, in math, music,
and in pedagogy itself for Kazakhstani teachers of ESL.  As a PhD student, I
volunteered for a year and a half teaching math courses to inmates at San
Quentin prison with the Prison University Project.

Many of my extra-curricular activities, though not explicitly in a classroom,
have had a teaching component.  I have acted as a formal and informal mentor to
numerous PhD students, both as part of the student mentorship program at UC
Berkeley and as a postdoctoral researcher at MIT.  At Google, I acted as an
official mentor for several junior engineers. For most of my PhD, I organized
and conducted my own reading group for any interested students on topics
including variational Bayes, Bayesian nonparametrics, differential geometry, the
bootstrap, and functional analysis. Consulting often has a pedagogical component
to it; I participated in the UC Berkeley statitsical consulting class, provided
statistical consulting as a fellow in the Berkeley Institute of Data Science,
and consulted professionally, including participating for several years in
Berkeley's chapter of the National Security Agency Statistical Advising Group.

Good teaching is an art that is never perfected.  Over the years I have made
many missteps, learned from at least some of my mistakes, and will continue to
learn for the rest of my life.  For the remainder of the essay, I will discuss
some principles that I have come to believe make for good teaching and a
little bit about the context.

Stories I want to tell:

Motivating code cleanliness in STAT215A

Proofs of angle sizes for 6A

Introduction to statics and axiomatic thinking

Lesson planning and keeping the class on track (PUP stats going too fast,
econ student talking about philosophy in a stats class, ESL story time)

Clear expectations (compare materials course vs 210A as writing classes)

Getting regular feedback on student progress (PUP, ESL)

Teaching for a range of abilities (open ended problem sets)



Organizing ideas:
- Have a plan
- Motivate the material
- Focus on the student




\end{document}
