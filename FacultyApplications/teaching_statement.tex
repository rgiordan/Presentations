\documentclass{article}
\usepackage[utf8]{inputenc}
%\usepackage{multirow}
%\usepackage{multicol}
%\usepackage{array,booktabs}

\title{Teaching Statement}
\author{Ryan Giordano}
\date{September 2021}

\begin{document}

\maketitle

I think of teaching, like writing and oral presentations, to be an important
part of my research rather than a distraction from it.  Careful, empathetic
exposition of complex ideas is, of course, necessary for reasearch to reach a
wide audience.  But, even more, the very act of teaching well forces the teacher
to refine and examing their ideas.  The benefits of thoughtful teaching are
particularly great in a field like statistics, where it can be much harder to
truly understand the fundamental concepts (hypothesis testing, Bayes rule,
randomness in the real world) than it is to simply mechanically apply the
mathematical tools of the trade.

Accordingly, I have had a lifelong passion for teaching.  In addition to working
as a university-level teaching assistant both as an undergraduate in engineering
and as a doctoral student in statistics (for which I received a university
teaching award), I was a full-time teacher for two years at the middle and
elemenatary school level as a Peace Corps volunteer in Kazakhstan teaching math
and ESL.  During the Peace Corps, I also wrote a textbook and organized and
participated multiple extracurricular classes for my community, in math, music,
and in pedagogy itself for Kazakhstani teachers of ESL.  As a PhD student, I
volunteered for a year and a half teaching math courses to inmates at San
Quentin prison with the Prison University Project.

Many of my extra-curricular activities, though not explicitly in a classroom,
have had a teaching component.  I have acted as a formal and informal mentor to
numerous PhD students, both as part of the student mentorship program at UC
Berkeley and as a postdoctoral researcher at MIT.  At Google, I acted as an
official mentor for several junior engineers. For most of my PhD, I organized
and conducted my own reading group for any interested students on topics
including variational Bayes, Bayesian nonparametrics, differential geometry, the
bootstrap, and functional analysis. Consulting often has a pedagogical component
to it; I participated in the UC Berkeley statitsical consulting class, provided
statistical consulting as a fellow in the Berkeley Institute of Data Science,
and consulted professionally, including participating for several years in
Berkeley's chapter of the National Security Agency Statistical Advising Group.

Good teaching is an art that is never perfected.  Over the years I have made
many missteps, learned from at least some of my mistakes, and will continue to
learn for the rest of my life.  For the remainder of the essay, I will discuss
some principles that I have come to believe make for good teaching and a
little bit about the context.


\section{Motivate the students}

During my second year as a PhD student at UC Berkeley I was asked by Prof. Bin
Yu to be her teaching assistant for the graduate-level course in applied
statistics.  The course was organized around a number labs using real-life
datasets, and my reponsibilties were to give weekly lectures, hold office hours,
and grade the written labs.  In addition, Prof. Yu asked me to add a
reproducible research component to the course based on my experience at Google,
to which end I incorporated Github, code readbility, and unit testing into
the lab requirements.

I quickly realized that simply teaching code readability and making it a
component of the grade was insufficient.  At Google, you cannot submit changes
that violate readability standards, but, in the class, not a single student was
carefully following readability guidelines in their submitted code.  The
students --- who were otherwise very highly motivated --- simply did not see the
importance of readability enough to change bad habits.  To address this, I
designed an in-class exercise in which the students had to ``reproduce'' a
simple analysis written by me.  In my code, I deliberately and systematically
violated all the code readbility guidelines I was trying to teach and, as a
result, it was quite difficult to understand what my analysis was doing.  To
sweeten the pot, I put a small but meaningful error in the code and challenged
the students to find it.  The students loved the puzzle-solving aspect of the
assignment and, to my delight, spent much of the hour complaining about my
terrible style.  Following this assignment, the labs' code readability improved
considerably.

Students, like all humans, love games and puzzles, and as long as the
pedagogical goal is served, I try to emphasize the game or puzzle aspect of a
problem whenever possible.  For example, oral production and comprehension of
mathematical concpets was a cornerstone of my seventh grade ESL lesson plan in
Kazakhstan.  I designed a game in which students in teams would read or describe
mathematical formulas or figures to a teammate, who had to understand their
description and transcribe the result on the board.  The game relied directly on
core skills, was built entirely around interactions between students, could be
tuned to be as hard or as easy as was required, and usually produced a little
bit of shouting chaos that middle schoolers love.

I learned early that it is important to be the teacher the students want, not
the teacher you would have wanted.  My first official teaching position was as
an undergraduate at UIUC, when I was offered the chance to act as a teaching
assistant for the core statics class in the school of engineering. My
responsibilities included a weekly lecture, grading, and office hours. At the
time, I was particularly excited about axiomatic thinking, and designed all my
lectures around this fascination, re-presenting the course material with an
emphasis on assumptions and consequences.  Sadly, I was not then in the habit of
getting regular feedback from my students, and didn't realize until much later
how little the students shared my interest.  They felt that the course material
was rigorous enough as is, and wanted context, motivation, and intuition from
me.  My course evaulations were poor, but I resolved to put the students'
needs before my own the next time around.

That said, however, being genuinely passionate about the subject material can be
extremely motivating for students.  Enthusiasm is infectious, and enthusiastic
students learn the best.  Whatever my shortcomings as a teacher, one cannot deny
my enthusiasm.  Thanks, maybe, to my theater background (I acted and directed
theater for many years through high school and college), I am confident in front
of a crowd, and able to bring a lot of physical energy into a classroom.  In
Kazakhstan, I organized an English-language math club for anyone from the
community who was intersted, and I always drew a surprisingly large and diverse
crowd.  At one point we were talking about what topics to discuss next and
I started asking the attendees why they were so interested in math club, and
one (himself a volunteer math teacher from Turkey) answered, generously,
``We come to be with you.''


\section{Have a plan and check it regularly}

Other than the Kazakh language, my Peace Corps in-country training was all about
pedagogy, and the cornerstone of the pedagogical curriculum was lesson planning.
Just as good expository writing has structure at many levels, a well-designed
class benefits from a relatively small number of articulable high-level goals,
with explicit connections to individual topics, which are made concrete in daily
lesson plans.  A clear plan helps the teacher remember what is essential and
what can be elided, and helps to motivate learning.

Only slightly less important than having a plan is knowing when to deviate from
it.  It is imperative for a teacher to regularly and meaningfully evaluate
whether the students are aquiring the skills they need to aquire to progress in
the curriculum.  Perhaps ironically, evaluation is extra important at the
university level, where imposter's syndrome is an epidemic and students are
highly motivated to give the appearance of understanding, whether or not they
do.

I find it useful to evaluate using many modalities, both to give students the
best chance to shine and to give myself the best chance of discovering
shortcomings.  For example, most techincal lectures have many points at which
minor inferential steps can be made into a short, minute-long exercise.
Explicitly pausing and then giving a quieter student the chance to fill in a
step both requires the students to remain actively engaged and can reveal if the
exposition is going too quickly.  (When teaching younger students, I like to
warn them that I will make a mistake in the next five minutes, and challenge
them to detect it.)  Homework, of course, has some value, but it is most
valuable when it requires the students to use concepts creatively, otherwise
even highly motivated students may copy from the book or form others out of fear
born of the imposter's syndrome.  I find short, low-stakes, written in-class
quizzes at the beginning of class to be particularly effective at checking in on
students.  I also like to use small group work, both because it gives the
opportunity for weaker students to learn from stronger ones and because I can
learn a lot by walking around and listening to the conversation. Relatedly, it
can be wonderful to have students evaluate each others' work, either in the form
of oral or written presentations; typically the teacher can learn a lot about
the students' state of mind both from the presentation and the feedback.

In my experience, one consequence of good evaluations in a math classroom is
seeing clearly that typical classrooms have a wide range of abilities, which
helps the teacher see the importance of designing a classroom that accomodates
different levels. For example, I taught an introductory statistics class at San
Quentin University through the Prison University Project (PUP).  The students
were all excited to learn but came from vastly different math backgrounds ---
some had been top students when they were younger, some had only learned to read
as adults through PUP.  To help accomodate the range of abilities and needs, I
reduced the proportion of the class devoted to lectures and increased the time
available for individual or group work while I walked around and answered
questions.  I would designed the problem sets with the expectation that
\emph{no} student would be able to complete the whole thing in the time
allotted, so that the faster students could quickly proceed to more challenging
problems, and the slower students could spend more time at their level.  When I
found the same question was being asked repeatedly, I would bring everyone
together for a brief lecture on the question, and then return to individual
work.

The primary purpose of evaluation, it must be remembered, is to give and get
useful feedback, and to motivate the students to stay engaged.  It is not to
sort students into wheat and chaff, and must not be presented that way.  If
nothing else, such a culture of intimidation and evaluation makes it more
difficult for a teacher to get the feedback they need to improve their lesson
plans.  It is true that, at the end of the day, we often have to give students
meaningful grades which have an evaluative component.  To my mind this is, at
best, a necessary evil.  In order to render final grades as fair and useful as
possible, I always work to make the standards for the final grades as clear and
open, to allow students to continuously monitor their grade, to communicate with
struggling students precisly what they need to do to improve their grade, and to
reduce, as much as possible, large, high-stakes assignments such as final exams.







\section{Notes}


Stories I want to tell:

X Motivating code cleanliness in STAT215A

X Oral comprehension game for 7A

X Introduction to statics and axiomatic thinking

X Lesson planning and keeping the class on track (PUP stats going too fast,
econ student talking about philosophy in a stats class, ESL story time)

X Clear expectations (compare materials course vs 210A as writing classes)

X Getting regular feedback on student progress (PUP, ESL)

X Teaching for a range of abilities (open ended problem sets)

Include anecdotes from other teaching things (consulting, \&c)?



\end{document}
