
%  At the university level community is particuarly important, since
% professors are cultivating future colleagues.  In a poorly-taught classroom,
% information flows only from teacher to student, assessment functions to separate
% ``good'' students from ``bad'', and the students' motivation is taken for
% granted.
In a classroom built on intellectual community, the teacher actively
seeks feedback from the students and encourages the students to mentor one
another, assessment allows students to monitor their own progress and for the
teacher to assess their own teaching, and the teacher works to contextualize the
material within the students' interests.

I have been teaching throughout my adult life, and my experience has helped me
learn both how to create community and why it is valuable.
%
\begin{itemize}
    %
    \item I worked as a university-level teaching assistant both as an
    undergraduate in engineering and, as a PhD student, for the
    graduate-level applied statistics course. I received a university teaching
    award for my teaching at the graduate level.
%
    \item I was a full-time teacher for two years as a Peace Corps volunteer in
    Kazakhstan teaching math and English as a second language (ESL).  During
    this time, I wrote a math textbook in simple English for middle-school ESL
    students.  I also organized multiple extracurricular classes for my
    community in math, music, and pedagogy.
%
    \item As a PhD student, I volunteered for a year and a half teaching math
    courses, including statistics, to inmates at San Quentin prison with the
    Prison University Project (PUP).
%
\end{itemize}

Furthermore, many of my other professional activities, though not explicitly in a
classroom, have had a teaching component.  In these activities, community-based
thinking is as valuable as in the classroom.
%
\begin{itemize}
    %
    \item I have acted as a formal and informal mentor to numerous PhD students,
    both as part of the student mentorship program at UC Berkeley and as a
    postdoctoral researcher at MIT.  As a senior engineer at Google, I acted as
    an official mentor for several junior engineers and was the technical lead
    for a small research team.
%
    \item For most of my PhD, I organized and conducted my own reading group for
    any interested students on topics including variational Bayes, Bayesian
    non-parametrics, differential geometry, the bootstrap, and functional
    analysis.
%
    \item I have provided statistical consulting services in many settings,
    including in the UC Berkeley statistical consulting class, as a fellow in
    the Berkeley Institute of Data Science, as a private contractor, and, for
    several years as a member of UC Berkeley's chapter of the National Security
    Agency Statistical Advising Group (NSASAG).
%
\end{itemize}
%
For the remainder of the essay, I will describe instances when I was able to
enact the community-based practices of two-way communication, non-judgmental
feedback, and the cultivation of intrinsic motivation.



Evaluating students' performance is a part of every classroom, but its role in a
community-based classroom is ideally productive and as non-threatening as
possible.

Useful evaluation is frequent, transparent, and conducted via many modalities.
When I taught in the Peace Corps, at PUP, and at UC Berkeley, as much as
possible I compute students' grades from many small projects rather than a few
large ones, let the students monitor their own progress, and based the grade on
many different modes of performance, including in-class participation, homework,
exams, and group work.  In this way, struggling students can ask for help early
and feel empowered to improve their grade; in the applied statistics course I
helped teach at UC Berkeley, some of the students who began with the weakest
backgrounds went on to become some of the strongest students through frequent
feedback and a lot of one-on-one help in office hours.


A teacher can learn from the students whether their teaching is effective, and
can even often gain valuable insights themselves when students interpret
material in new ways.  Further, by evaluating each others' work, students
can derive motivation from their peers and build community among themselves.
