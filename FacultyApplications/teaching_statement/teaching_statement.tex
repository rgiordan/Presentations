\usetheme{metropolis}           % Use metropolis theme
\usepackage{amsmath}
\usepackage{mathrsfs}
\usepackage{tabularx}

\usepackage{mathtools}

\usepackage{cleveref}

% For custom oversets
\usepackage{accents}

% Algorithm notation
\usepackage{algorithm}
\usepackage{algpseudocode}
% algorithmicx package

% https://tex.stackexchange.com/questions/5017/center-column-with-specifying-width-in-table-tabular-enviroment
\usepackage{array}
\newcolumntype{x}[1]{>{\centering\let\newline\\\arraybackslash\hspace{0pt}}p{#1}}

\usepackage{natbib}
% \bibliographystyle{abbrvnat}

\usepackage{enumitem}
\setlist{nolistsep}

\usepackage{geometry}
%\geometry{margin=1.2in}
\geometry{top=0.9in}
\geometry{left=1.4in}
\geometry{right=1.4in}

% \documentclass{article}
% \usepackage[utf8]{inputenc}
%\usepackage{multirow}
%\usepackage{multicol}
%\usepackage{array,booktabs}

\title{Ryan Giordano Teaching Statement}

\author{
  Ryan Giordano \\ \texttt{rgiordan@mit.edu }
}

\begin{document}


\begin{minipage}[t]{0.5\textwidth}
\hspace{-2em} % Easier than doing it right!
{\bf \LARGE Teaching Statement}\\
\end{minipage}
\begin{minipage}[t]{0.5\textwidth}
%    \begin{flushright}
        \hspace{8em} % Easier than doing it right!
        {\LARGE Ryan Giordano}
%    \end{flushright}
\end{minipage}

Good teaching does not only transfer knowledge, it creates intellectual
community. At the university level the community component is particuarly
important, since professors are cultivating future colleagues.  In a
poorly-taught classroom, information flows only from teacher to student,
assessment functions to separate ``good'' students from ``bad'', and the
students' motivation is taken for granted.  In a classroom built on intellectual
community, the teacher actively seeks feedback from the students and encourages
the students to mentor one another, assesment allows students to monitor their
own progress and for the teacher to assess their own teaching, and the teacher
works to contextualize the material within the students' interests.

I have been teaching throughout my adult life, and my experience has taught me
both how to create community and why it is valuable.
%
\begin{itemize}
    %
    \item I worked as a university-level teaching assistant both as an
    undergraduate in engineering and, as  as a PhD student, for the
    graduate-level applied statistics course. I received a university teaching
    award for my teaching in the latter.
%
    \item I was a full-time teacher for two years at the middle and elemenatary
    school level as a Peace Corps volunteer in Kazakhstan teaching math and
    English as a second language (ESL).  For my core class, I wrote a math
    textbook in simple English for ESL students.  During the Peace Corps, I also
    organized multiple extracurricular classes for my community in math, music,
    including classes in pedagogy for Kazakhstani teachers of ESL.
%
    \item As a PhD student, I volunteered for a year and a half teaching math
    courses to inmates at San Quentin prison with the Prison University Project
    (PUP).
%
\end{itemize}

Futhermore, many of my other professional activities, though not explicitly in a
classroom, have had a teaching component.  In these activities, community-based
thinking --- two-way communication, non-judgemental feedback, and the
cultivation of intrinsic motivation --- is as valuable as in the classroom.
%
\begin{itemize}
    %
    \item I have acted as a formal and informal mentor to numerous PhD students,
    both as part of the student mentorship program at UC Berkeley and as a
    postdoctoral researcher at MIT.  At Google, I acted as an official mentor
    for several junior engineers and was the technical lead for a small research
    team.
%
    \item For most of my PhD, I organized and conducted my own reading group for
    any interested students on topics including variational Bayes, Bayesian
    nonparametrics, differential geometry, the bootstrap, and functional
    analysis.
%
    \item I participated in the UC Berkeley statitsical consulting class,
    provided statistical consulting as a fellow in the Berkeley Institute of
    Data Science, and consulted professionally, including participating for
    several years in Berkeley's chapter of the National Security Agency
    Statistical Advising Group (NSASAG).
%
\end{itemize}
%
For the remainder of the essay, I will describe instances when I was able to
enact the community-based practices of two-way communication, non-judgemental
feedback, and the cultivation of intrinsic motivation.


\paragraph{Cultivate intrinsic motivation.}
%
During my second year as a PhD student at UC Berkeley I was asked by Prof. Bin
Yu to be her teaching assistant for the graduate-level course in applied
statistics.  The course was organized around a number labs using real-life
datasets, and my reponsibilties were to give weekly lectures, hold office hours,
and grade the written labs.  In addition, Prof. Yu asked me to add a
reproducible research component to the course based on my experience at Google,
to which end I incorporated Github, code readbility, and unit testing into
the lab requirements.

I quickly realized that simply teaching code readability and making it a
component of the grade was insufficient.  At Google, you cannot submit changes
that violate readability standards, but, in the class, not a single student was
carefully following readability guidelines in their submitted code.  The
students --- who were otherwise very highly motivated --- simply did not see the
importance of readability enough to change bad habits.  To address this, I
designed an in-class exercise in which the students had to ``reproduce'' a
simple analysis written by me.  In my code, I deliberately and systematically
violated all the code readbility guidelines I was trying to teach and, as a
result, it was quite difficult to understand what my analysis was doing.  To
sweeten the pot, I put a small but meaningful error in the code and challenged
the students to find it.  The students loved the puzzle-solving aspect of the
assignment and, to my delight, spent much of the hour complaining about my
terrible style.  Following this assignment, the labs' code readability improved
considerably.


\paragraph{Evaluate productively.}

Evaluating students' performance is a part of every classroom, but its role in a
community-based classroom is ideally productive and as non-threatening as
possible.

Evalaution should allow a teacher to motivate and get feedback from students at
all levels from the most to least accomplished. When I taught an introductory
statistics class at San Quentin University through the Prison University Project
(PUP), the students came from vastly different math backgrounds --- some had
been top students when they were younger, some had only learned to read as
adults through PUP.  To help accomodate the range of abilities and needs, I
reduced the proportion of the class devoted to lectures and increased the time
available for individual or group work while I walked around and answered
questions.  I would design the problem sets with the expectation that \emph{no}
student would be able to complete the whole thing in the time allotted, so that
the faster students could quickly proceed to more challenging problems, the
slower students could spend more time at their level, and (hopefully) nobody
would feel either bored or discouarged.  When I found the same question was
being asked repeatedly, I would bring everyone together for a brief lecture on
the question, and then return to individual work.  In this way, I was able to
create a classroom environment that accomodated everyone.

Useful evaluation is frequent, transparent, and conducted via many modalities.
When I taught in the Peace Corps, at PUP, and at UC Berkeley, as much as
possible I compute students' grades from many small projects rather than a few
large ones, let the students monitor their own progress, and based the grade on
many different modes of performance, including in-class participation, homework,
exams, and group work.  In this way, struggling students can ask for help early
and feel empowered to improve their grade; in the applied statistics course I
helped teach at UC Berkeley, some of the students who began with the weakest
backgrounds went on to become some of the strongest students through frequent
feedback and a lot of one-on-one help in office hours.
%   At the London School of
% Economics, I experienced as an MSc student what I consider to be a poor system
% of evaluation, in which one's entire grade is determined by a single,
% closed-book exam at the end of the semester; under this system, struggling
% students were simply eliminated, some needlessly.




\paragraph{Encourage multi-way communication.}

A teacher can learn from the students whether their teaching is effective, and
can even often gain valuable insights themselves when students interpret
material in new ways.

It is crucial that a good teacher have a written lesson plan, and just as
crucial that they check regularly that students are keeping up with it. For
example, most technical lectures have many points at which minor inferential
steps can be made into a short, minute-long exercise. I build in explicit pauses
for the students to work out such exercises, which both requires the students to
remain actively engaged and can reveal if the exposition is going too quickly.
Similarly, I find short, low-stakes, written in-class quizzes at the beginning
of class to be particularly effective at checking in on students.

Statistical consulting is another venue in which two-way communication is
particularly important.  Rarely, I have found, does a petitioner actually ask a
useful statistical question at first, and a statistical consultant provides the
most value by first listening carefully to the problem details.  For example, as
part of the NSASAG, we were asked how to compute low-rank approximations of
matrices with some given statistical properties.  Upon being asked for the
motivation, we learned that all that was actually needed was the computation of
a t-statistic based on a linear form of a high-dimensional parameter, which I
saw could be computed exactly using the conjugate gradient algorithm with no
recourse to low-rank approximations.



\end{document}
