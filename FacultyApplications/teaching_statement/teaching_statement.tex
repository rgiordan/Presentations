\documentclass{article}
\usepackage{fancyhdr}


\pagestyle{fancy}
\fancyhf{}
\rhead{Ryan Giordano}
\lhead{Research Statement}
\rfoot{Page \thepage}

\usepackage{tabularx}

% \usepackage{fancyhdr}
% \pagestyle{fancy}

% \fancyfoot{}
% \fancyfoot[C]{Rough draft---do not distribute}

\usepackage{etoolbox}


\usepackage{microtype}
\usepackage{graphicx}
\usepackage{subfigure}
\usepackage{booktabs} % for professional tables
\usepackage{xcolor}
\usepackage[hidelinks=True]{hyperref}
\usepackage{xargs}[2008/03/08]

% Documentation
% http://ftp.math.purdue.edu/mirrors/ctan.org/macros/latex/contrib/refstyle/refstyle.pdf
\usepackage{refstyle}
\usepackage{varioref} % Use refstyle instead of varioref directly.

\usepackage{amsmath}
\usepackage{amssymb}
\usepackage{amsfonts}
\usepackage{amsthm}
\usepackage{mathrsfs} % For mathscr
\usepackage{mathtools}

\usepackage[authoryear]{natbib}
\bibliographystyle{apalike}

\usepackage{geometry}
\geometry{margin=1.5in}


\usepackage{enumitem}
\setlist{nolistsep}

\usepackage{geometry}
%\geometry{margin=1.2in}
\geometry{top=0.9in}
\geometry{left=1.4in}
\geometry{right=1.4in}

% \documentclass{article}
% \usepackage[utf8]{inputenc}
%\usepackage{multirow}
%\usepackage{multicol}
%\usepackage{array,booktabs}

\title{Ryan Giordano Teaching Statement}

\author{
  Ryan Giordano \\ \texttt{rgiordan@mit.edu }
}

\begin{document}


\begin{minipage}[t]{0.5\textwidth}
\hspace{-2em} % Easier than doing it right!
{\bf \LARGE Teaching Statement}\\
\end{minipage}
\begin{minipage}[t]{0.5\textwidth}
%    \begin{flushright}
        \hspace{8em} % Easier than doing it right!
        {\LARGE Ryan Giordano}
%    \end{flushright}
\end{minipage}

Good teaching does not only transfer knowledge, it creates intellectual
community---after all, university professors are cultivating future colleagues.
Members of an intellectual community are self-motivated, and good teachers
contextualize course material within the students' own interests. In an
intellectual community, feedback facilitates self-improvement rather than
division, and good teachers use assessment to allow students to monitor their
own progress and for the teacher to assess their own teaching.  And in an
intellectual community, information does not flow only from one person to the
rest; similarly, a good teacher helps students teach and learn from one another.
As I elaborate in a series of vignetts below, I find that these themes of
motivation, feedback, and multi-directional communication recur and co-occur in
many of the successses from my years of teaching at the university level, as a
teacher in the Peace Corps, and as a volunteer with the Prison University
Project.

\paragraph{Motivating code quality.}
%
During my second year as a PhD student at UC Berkeley, I was asked by Prof. Bin
Yu to be her teaching assistant for the graduate-level course in applied
statistics.  The course was organized around a number labs using real-life
datasets, and my responsibilities were to give weekly lectures, hold office hours,
and grade the written labs.  In addition, Prof. Yu asked me to add a
reproducible research component to the course based on my experience at Google,
to which end I incorporated Github, code readability, and unit testing into the
lab requirements.
%
I quickly realized that simply teaching code readability and making it a
component of the grade was insufficient.   The students --- who were otherwise
very highly motivated --- simply did not see the importance of readability
enough to change bad habits.  To address this, I designed an in-class exercise
in which the students had to ``reproduce'' a simple analysis written by me.  In
my code, I deliberately and systematically violated all the code readability
guidelines I was trying to teach.  As a result, it was quite difficult to
understand what my analysis was doing.  To sweeten the pot, I put a small but
meaningful error in the code and challenged the students to find it.  The
students loved the puzzle-solving aspect of the assignment and, to my delight,
spent much of the hour complaining about my terrible style.  I had motivated the
students in a way they understood, and, following this assignment, the labs'
code readability improved considerably.

\paragraph{Working with multiple ability levels.}

Math teachers often have to accommodate a wide range of student abilities and
backgrounds, and my introductory statistics class at San Quentin University
through the Prison University Project (PUP) was particularly extreme in this
regard.  Some students had been at the top of their class when they were
younger, some were very intelligent but had only learned basic arithmetic as
adults through PUP.  My goal was to design exercises which accommodated this
range of abilities and needs without leaving anyone discouraged or bored. To
this end, I reduced the proportion of the class devoted to lectures and
increased the time available for individual or group work while I walked around
and answered questions.  I would design problem sets for such periods with the
expectation that \emph{no} student would be able to complete the whole thing in
the time allotted. In this way, the faster students could quickly proceed to
more challenging problems, the slower students could spend more time with
concepts that were new to them. When I found the same question was being asked
repeatedly, I would bring everyone together for a brief lecture on the question,
and then return to individual work.  By providing evaluation tasks that were
non-threatening and matched to students' ability levels, I helped create an
inclusive classroom environment, and got a lot of feedback for myself about the
effectiveness of my teaching as well.

\paragraph{Two-way communication in statisical consulting.}

Statistical consulting, though not a classroom setting, is teaching-adjacent
venue in which two-way communication is particularly important. I have provided
statistical consulting services in many settings, including in the UC Berkeley
statistical consulting class, as a fellow in the Berkeley Institute of Data
Science, as a private contractor, and, for several years as a member of UC
Berkeley's chapter of the National Security Agency Statistical Advising Group
(NSASAG).  Rarely, I have found, does a petitioner actually ask a useful
statistical question at first, and a statistical consultant provides the most
value by first listening carefully to the problem details.  For example, as part
of the NSASAG, we were asked how to compute low-rank approximations of matrices
with some given statistical properties.  Upon pressing for more information
about the motivation, I learned that all that was actually needed was the
computation of a t-statistic based on a linear form of a high-dimensional
parameter, which I saw could be computed exactly using the conjugate gradient
algorithm with no recourse to low-rank approximations.  Because I promoted
two-way communication rather than simply attempting to convey statistical
knowledge, we were able to come to a much better solution that we would have
otherwise.

\paragraph{Frequent and meaningful evaluation.}

When I was a teaching assistant for the graduate-level applied statistics
course, the students came from a wide array of technical backgrounds, from
statistics to psychology, and some students were little more than auditing,
while some wanted to work hard to push their own boundaries.  I expected that
some students would struggle with the material, and wanted to provide evaluation
that would be useful to all students.  First, I made the rubric for grading the
labs as clear as possible, and allowed for a lab to be successful in many
different aspects, including clarity of exposition, quality of graphics,
analytical creativity, etc..  Next, I made sure that the students were
continually updated with their own progress and on the grade that they were
slated receive based on their performance to-date.  Finally, I offered to give
detailed feedback on how to improve a particular lab if the students were
willing to spend time with me in person.  With the help of frequent and
substantive feedback, some of the students who began with the weakest
backgrounds went on to become some of the strongest by the end of the class, and
one has even gone on to become a professional data scientist.


\paragraph{Short questions during lecture.}

Most technical lectures have many points at which minor inferential steps can be
made into a short, minute-long exercise.  Whenever it is possible to get
feedback from the audience, I always build in such mini-exercises, which both
requires students to remain actively engaged and reveals if the exposition is
going too quickly.  A particularly fun variant of this idea which I developed
during the Peace Corps is the ``deliberate mistake''.  I would warn my
seventh-grade math students that I was going to make a mistake in the next five
minutes. The students were instantly on the edge of their seat.  In their
enthusiasm, the students often identified many ``mistakes'' that were not
actually mistakes, but every such instance  was still a valuable teaching
moment.  By encouraging communication from the students to myself, I was able to
both motivate the students intrinsically and evaluate for myself whether I was
teaching effectively.


\paragraph{Be the teacher your students want, not the teacher you would want.}

As a third-year undergraduate I was asked by my engineering department to be a
teaching assistant for the second-year class in statics, which was a required
course for most engineering majors.  Based on my own intellectual tastes at the
time, I spent my weekly supplementary lectures re-deriving the course material
from a more rigorous mathematical perspective in the form of assumptions and
theorems.  This being my first teaching role, I violated all of the above rules ---
I did not seek meaningful feedback from the students, I did not encourage
students to work together, and, most importantly, I considered only my own
motivations and not theirs.  As a consequence, it was not until I received
teaching evaluations at the end of the semester that I realized that the
students uniformly had wanted more intuition from the supplementary lectures,
not more rigor.  Fortunately, this humbling experience set me on the long  and
never-ending path towards improving my teaching, leading to the more successful
vignettes above.


\end{document}
