\documentclass[twoside,11pt]{article}

\usepackage{amsmath}
\usepackage{amssymb}
\usepackage{amsfonts}

\algblock{Inputs}{EndInputs}
\algnotext{EndInputs}

\algblock{Postprocessing}{EndPostprocessing}
\algnotext{EndPostprocessing}

\newtheorem{assu}{Assumption}
\crefname{assu}{assumption}{assumptions}

\newtheorem{lem}[theorem]{Lemma}
\crefname{lem}{lemma}{lemmas}

\newtheorem{ex}{Example}
\crefname{ex}{example}{examples}

\def\red#1{\textcolor{red}{#1}}
\def\blue#1{\textcolor{blue}{#1}}
\def\green#1{\textcolor{violet}{#1}}

\def\undernote#1#2{\underbrace{#1}_{\mathclap{\substack{#2}}}}
\def\overnote#1#2{\overbrace{#1}^{\mathclap{\substack{#2}}}}

\newcommand{\iid}{\overset{iid}{\sim}}

\newcommand{\dataset}{{\cal D}}
\newcommand{\fracpartial}[2]{\frac{\partial #1}{\partial  #2}}
\newcommand{\fracat}[3]{\left. \frac{#1}{#2} \right|_{#3}}
\newcommand{\trans}{\intercal}
\newcommand{\const}{C}
\newcommand{\diag}[1]{\mathrm{Diag}\left(#1\right)}
\newcommand{\grad}[2]{\nabla_{#1} \left. #2 \right.}
\newcommand{\hess}[2]{\nabla^2_{#1} \left. #2 \right.}
\newcommand{\gradat}[3]{\nabla_{#1} \left. #2 \right|_{#3}}
\newcommand{\norm}[1]{\left\Vert #1\right\Vert}
\newcommand{\abs}[1]{\left| #1\right|}
\newcommand{\normp}[1]{\left\Vert #1\right\Vert_*}
\newcommand{\trace}[1]{\mathrm{Tr}\left(#1\right)}
\newcommand{\oned}[1][\thetadim]{1_{#1}} % Vector of ones
\newcommand{\zerod}[1][2\thetadim]{0_{#1}} % Vector of zeros
\newcommand{\ident}[1][\thetadim]{I_{#1}} % Identity matrix
\newcommand{\dequal}{\stackrel{d}{=}}


\newcommand{\y}{y}  % data
\newcommand{\z}{z}  % Reparameterization trick variable
\newcommand{\Z}{\mathcal{Z}_{\znum}}  % Set of \z
\newcommand{\znum}{N} % Number of elements of \Z
\newcommand{\Zindep}{\tilde{\mathcal{Z}}}  % Set of \z
\newcommand{\p}{\mathcal{P}} % posterior and model probability
\newcommand{\q}{\mathcal{Q}} % vb approximation
\newcommand{\gfull}{\mathcal{G}} % Grad of exact KL
\newcommand{\g}{\hat{\gfull}} % Grad of MC KL
\newcommand{\hfull}{\mathcal{H}} % Hess of exact KL
\newcommand{\h}{\hat{\hfull}} % Hess of MC KL
\newcommand{\scorecov}{\hat{\Sigma}_{s}} % Estimated score covariance
\newcommand{\fun}{f} % Function of interest
\newcommand{\etahat}{\hat{\eta}}
\newcommand{\etatil}{\tilde{\eta}}
%\newcommand{\etastar}{\stackrel{*}{\eta}} % Superscripts are too high
%\newcommand{\etastar}{\eta^{*}} % Double superscript
\newcommand{\etastar}{\accentset{*}{\eta}}
\newcommand{\ghat}{\hat{g}} % MC estimate of model grad
\newcommand{\hhat}{\hat{h}} % MC estimate of model hessian


\newcommand{\qoi}[1]{\phi\left(#1\right)}
\newcommand{\qoigrad}[1]{\nabla_{\eta} \phi\left(#1\right)}
\def\normal#1#2{\mathcal{N}\left(#1 \vert #2\right)}
\def\wishart#1#2{\mathcal{W}\left(#1 \vert #2\right)}
\def\normz{\mathcal{N}_{\mathrm{std}}\left(\z\right)}
\newcommand{\post}{\p(\theta \vert \y)}
\newcommand{\logjoint}{\log \p(\theta, \y)}
\newcommand{\logjointgrad}{\nabla_{\theta} \log \p(\theta, \y)}
\newcommand{\logjointhess}{\nabla^2_{\theta} \log \p(\theta, \y)}
\newcommand{\etaopt}{\hat{\eta}}
\newcommand{\etaoptd}{\hat{\eta}_{D}}
\newcommand{\etaopts}{\hat{\eta}_{S}}

% https://stackoverflow.com/questions/1812214/latex-optional-arguments
\newcommand{\etamu}[1][]{\mu_{#1}}
\newcommand{\etamuhat}[1][]{\hat{\mu}_{#1}}
\newcommand{\etamustar}[1][]{\accentset{*}{\mu}_{#1}}

\newcommand{\etasigma}[1][]{\sigma_{#1}}
\newcommand{\etasigmahat}[1][]{\hat{\sigma}_{#1}}
\newcommand{\etasigmastar}[1][]{\accentset{*}{\sigma}_{#1}}

\newcommand{\etaxi}[1][]{\xi_{#1}}
\newcommand{\etaxihat}[1][]{\hat{\xi}_{#1}}
\newcommand{\etar}[1][]{{R_{#1}}}

\newcommand{\qdom}{\Omega_{\q}}
\newcommand{\thetadom}{\mathbb{R}^\thetadim}
\newcommand{\rdom}[1]{\mathbb{R}^{#1}}
\newcommand{\etadom}{\mathbb{R}^\etadim}

\newcommand{\thetadim}{{D_{\theta}}}
\newcommand{\etadim}{{D_{\eta}}}
\newcommand{\lambdadim}{{D_{\lambda}}}
\newcommand{\gammadim}{{D_{\gamma}}}

\def\expect#1#2{\underset{#1}{\mathbb{E}}\left[#2\right]}
\newcommand{\meanz}{\frac{1}{\znum} \sum_{n=1}^{\znum}}
\def\sumn{\sum_{n=1}^{N}}
\def\meann{\frac{1}{N}\sumn}
\def\sump{\sum_{p=1}^{P}}
\def\sumd{\sum_{d=1}^{\thetadim}}
\def\meann{\frac{1}{N}\sumn}
\def\expecthat#1#2{\underset{#1}{\widehat{\mathbb{E}}}\left[#2\right]}
\def\var#1#2{\underset{#1}{\mathrm{Var}}\left(#2\right)}
\def\lrcovfull#1#2{\underset{#1}{\mathrm{LRCov}}\left(#2\right)}
\def\lrcov#1#2{\underset{#1}{\widehat{\mathrm{LRCov}}}\left(#2\right)}
\def\lrvar#1#2{\underset{#1}{\mathrm{LRVar}}\left(#2\right)}
\def\cov#1#2{\underset{#1}{\mathrm{Cov}}\left(#2\right)}
\def\kl#1#2{\mathrm{KL}\left(#1 || #2\right)}
\def\klfullobj#1{\mathrm{KL}_{\mathrm{VI}}\left(#1\right)}
\def\klobj#1{\widehat{\mathrm{KL}}_{\mathrm{VI}}\left(#1\right)}
\def\expectpost#1{\expect{\post}{#1}}

% High dimensional normals notation
\def\zbar{\bar{z}}
\def\zzbar{\overline{zz^\trans}}
\def\sigmat{S}
\def\zcov{\hat{\Sigma}_{z}}


\DeclareMathOperator*{\argmax}{\mathrm{argmax}}
\DeclareMathOperator*{\argmin}{\mathrm{argmin}}
\DeclareMathOperator*{\esssup}{\mathrm{esssup}}
\DeclareMathOperator*{\essinf}{\mathrm{essinf}}
\DeclareMathOperator*{\argsup}{\mathrm{argsup}}
\DeclareMathOperator*{\arginf}{\mathrm{arginf}}

% Experiment results

\def\method{\textrm{\tiny METHOD}}
\def\nuts{\textrm{\tiny NUTS}}
\def\sadvi{\textrm{\tiny MF-ADVI}}
\def\sadvifr{\textrm{\tiny FR-ADVI}}
\def\raabbvi{\textrm{\tiny RAABBVI}}
\def\dadvi{\textrm{\tiny DADVI}}
\def\lrvb{\textrm{\tiny LRVB}}

\def\muerr{\varepsilon^{\mu}}
\def\sderr{\varepsilon^{\sigma}}
\def\freqsd{\xi}
\def\freqerr{\varepsilon^{\xi}}


\numberwithin{equation}{section}

\usepackage{hyperref}
\usepackage{cleveref}
\usepackage{mathrsfs}
\usepackage{mathtools}
\usepackage{xcolor}

\def\red#1{\textcolor{red}{#1}}
\def\blue#1{\textcolor{blue}{#1}}
\def\green#1{\textcolor{teal}{#1}}


\begin{document}

\title{Bochner's theorem notes}

\author{Ryan Giordano}

\maketitle

\section*{Setup}

Motivating settings:
%
\begin{itemize}
    \item Your collaborator has a crazy kernel fitting method.  How to check whether it's valid?
    \item How can you extend discrete stationary processes to continuous ones?
\end{itemize}
%

Goals:
%
\begin{itemize}
\item How can we tell whether a particular stationary kernel is positive definite?
\item Can we define an expressive class of valid kernels?
\end{itemize}


Subsidiary goals:
%
\begin{itemize}
\item What is a Fourier transform / inverse transform, and how to compute?
\item Motivate some STAT205A material (by using it)
\end{itemize}


\newpage
\section*{Fourier transforms}

Transforms:
%
\begin{align*}
    \fhat(\omega) :={}& \int_{-\infty}^{\infty} \exp(-2\pi i \omega x) f(x) dx &
    \ftil_k :={}& \sum_{n=1}^N \exp(-2 \pi k (n - 1) / N) f_n \\
    f(x) :={}& \int_{-\infty}^{\infty} \exp(2\pi i \omega x) \fhat(\omega) d\omega &
    f_n :={}& \frac{1}{N} \sum_{k} \exp(2 \pi k (n - 1) / N) \ftil_k.
\end{align*}
%

Linear operators:
%
\begin{itemize}
    \item Addition and multiplication
    \item Translation and scaling
    \item Differentiation
    \item Convolution
\end{itemize}
%

Domains:
%
\begin{itemize}
    \item Whole real line
    \item Integers $\leftrightarrow$ $(-1/2, 1/2)$.
    %
    \begin{itemize}
        \item Note that if $\omega = k + r$ where $k \in (-1/2, 1/2)$ and $k \in \mathbb{Z}$,
        then $\exp(2 \pi i \omega x) = \exp(2 \pi i r x)$, so you may as well just
        use $\omega \in (-1/2, 1/2)$.
    \end{itemize}
    \item A bounded domain, WLOG $(-1/2, 1/2)$.
    %
    \begin{itemize}
        \item Reasoning as above, you may as well only use $\omega \in \mathbb{Z}$.
        This corresponds to assuming that the function repeats.
        \item You could also use $\omega \in \mathbb{R}$.  Then non-integer
        values of $\omega$ serve only to set the function to zero outside
        the bounded domain.
    \end{itemize}
    %
    %
\end{itemize}
%
Let $\omega = k + \omega_r$ for $k \in \mathbb{Z}$ and $\omega_r \in (-1/2, 1/2)$.\\
Then for $n \in \mathbb{Z}$, $\exp(2\pi i \omega n) = \exp(2\pi i \omega_r n)$.

Some formulas:
%
\begin{itemize}
    \item $\exp\left(-\frac{1}{2} x^2 \right) \leftrightarrow \sqrt{2\pi} \exp\left(-2 (\pi \omega)^2 \right)$
    \item $1(-1/2 \le x \le 1/2) \leftrightarrow \sinc \omega = \sin \omega / \omega$
    \item $(1 - |x|) 1(|x| < 1) \leftrightarrow (\sinc \omega)^2$
\end{itemize}
%

%



\newpage
\section*{Bochner's theorem}

Preliminaries:
%
\begin{itemize}
    \item Fourier inversion theorem
    \item Fubini's theorem
    \item Fatou's lemma
    \item Characteristic function continuity
    \item Dominated convergence theorem
\end{itemize}
%
Sketch:

%
\begin{align*}
    g(\omega, A) ={}& \frac{1}{A} \int_0^A \int_0^A K(x, y) \exp(-2 \pi i \omega (x - y)) dx dy \\
        ={}& \int_{-\infty}^\infty \mu(\tau / A) K(\tau) \exp(-2 \pi i \omega \tau) d\tau \\
\textrm{where }    \mu(x) ={}& 1(|x| < 1)(1 - |x|) \\
    g(\omega, A, M) ={}& \mu(\omega / 2M) g(\omega, A) \\
    \int_{-\infty}^\infty \left(\frac{\sin t}{t}\right)^2 dt ={}& \pi
\end{align*}
%

Consequences:
%
\begin{itemize}
    \item Check $K(\tau) \propto \exp(-|\tau|)$: FT $ = 1 / (1 + \pi^2 \omega^2) \ge 0$
    \item Rectangular function cannot be a kernel
    \item Bochner's theorem characterizes valid characteristic functions
    \item Counterexamples: $K(\tau) = \delta(\tau)$, $K(\tau) = 1$
    \item Mercer's theorem for bounded domains: $K(\tau): [-1/2,1/2] \mapsto \mathbb{R}$
    %
    \begin{align*}
        K(\tau) = 
            \sum_{k \in \mathbb{Z}} \exp(2 \pi i k (x - y)) \fhat_k = 
            \sum_{k \in \mathbb{Z}} \cos(2 \pi k x) \cos(2 \pi k y) \fhat_k
    \end{align*}
    %
    \item Decay of Fourier coefficients gives smoothness
    \item Check our physicist's kernel
    \item Gaussian Process Kernels for Pattern Discovery and Extrapolation
\end{itemize}





\end{document}
