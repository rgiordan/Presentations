
\begin{frame}{Economist 2016 Election Model \citep{economist:2020:election}}

\begin{minipage}[t]{0.35\textwidth}
    \ElectionData{}
\end{minipage}
\begin{minipage}{0.04\textwidth}
\hfill
\end{minipage}
\begin{minipage}[t]{0.59\textwidth}

A time series model to predict the 2016 US presidential election
outcome from polling data.

%\pause
\spskip
\begin{itemize}
\item $\xvec = \x_1, \ldots, \x_N =$ Polling data ($N = 361$).
\item $\theta = $ Lots of random effects (day, pollster, etc.)
\item $f(\theta) = $ Democratic \% of vote on election day
\end{itemize}

%\pause
\spskip
We want to know $\expect{p(\theta \vert \xvec)}{f(\theta)}$.

%\pause
\spskip
Typically, we compute Markov chain Monte Carlo (MCMC) draws from the
posterior $p(\theta \vert \xvec)$.


\end{minipage}


\pause
% \hrulefill

% \onslide<2->{
% \textbf{Some typical model checking tasks:}
% }

% \begin{itemize}
%     \item<2-> How well are polls fit under cross-validation (CV)? \citep{vehtari:2012:bayesianpredictivemodelassessment}
%     \subitem{Re-fit with data points removed one at a time}
    %
    % \item<3-> 
\question{
    The people who responded to the polls were randomly selected.  \\
    If we had selected a different random sample, how much would our estimate have changed?\\
    How can we estimate $\var{\xfdist}{\expect{p(\theta \vert \xvec)}{f(\theta)}}$?
    }

\pause
%
\begin{itemize}
\item The model is (surely) misspecified
\item The posterior expectation marginalizes over many nuisance parameters
\item We are interested re--sampling for \textit{this} election, not a hypothetical future election
\end{itemize}
%

% }

\end{frame}







\begin{frame}{Results}

\question{
    \textbf{Idea: } Re-fit with bootstrap samples of data \citep{huggins:2023:bayesbag}\\
    \textbf{Problem:} Each MCMC run takes about $10$ hours (Stan, six cores).\\
    \textbf{Proposal:} Use full--data posterior draws
    to form a linear approximation to {\em data reweightings}.
}

\onslide<2->{
\begin{minipage}{0.43\textwidth}
    \ElectionData{}
\end{minipage}
\begin{minipage}{0.43\textwidth}
\begin{tikzpicture}
    \node[anchor=south west,inner sep=0] (image) at (0,0) {
        \ElectionResultsGlobal{}
    };
\end{tikzpicture}
\end{minipage}
}
%
\onslide<3->{
\vfill \begin{center}
%
\begin{center}
\begin{tabular}{ r l }
    {  Compute time for 100 bootstraps:} &
    {  51 days} \spskip\\
    {  Compute time for the linear approximation:} &
    {  Seconds}\\
    {  (But note the approximation has some error)} &
    \\
\end{tabular}
\end{center}

\end{center}
}
%
\end{frame}



% \begin{frame}{Outline}
% %
% \begin{itemize}
% %
% \item Data reweighting
% %
% \begin{itemize}
% %
% \item Write the change in the posterior expectation
%     as \blue{linear component} $+$ \red{error}
% \item The \blue{linear component} can be computed from a
% single run of MCMC
% \item The \blue{linear component} can be used to estimate the frequentist
% variability of posterior expectations
% %
% \end{itemize}
% \item In finite dimensions, the linear approximation gives a consistent
% variance estimator
% \item In problems with high--dimensional nuisance parameters, the linear approximation is
%     %
%     \begin{itemize}
%     \item Inconsistent (!)
%     \item Even for parameters that marginally obey a Bernstein von--Mises theorem (!)
%     \item But the error is $O_p(1)$, and proportional a nuisance parameter posterior covariance.
%     \end{itemize}
% %
% \end{itemize}
% %
% \end{frame}
