




%%%%%%%%%%%%%%%%%%%%%%%%%%%%%%%%%%%%%%%%%%%%%%%%%%%%%%%%%%%%%%%%%%%%%%%%%
%%%%%%%%%%%%%%%%%%%%%%%%%%%%%%%%%%%%%%%%%%%%%%%%%%%%%%%%%%%%%%%%%%%%%%%%%
%%%%%%%%%%%%%%%%%%%%%%%%%%%%%%%%%%%%%%%%%%%%%%%%%%%%%%%%%%%%%%%%%%%%%%%%%

\begin{frame}{What makes an analysis sensitive?  Preliminaries}
%
We are \textbf{robust to data dropping} if, for the $\Delta$
that changes conclusions and $\w^*$ dropping the $\alphan$
most influential points,
%
\begin{align*}
% $
%
\Delta \ge \thetafunlin(\w^*) - \thetafun(\thetahat(\onevec))
    % = {\color{red}- \sum_{n=1}^{\lfloor \alpha N \rfloor} \infl_{(n)} }
    \onslide<3->{
        =:   {\color{black} \noise } { \color{black} \shape}
    }
\onslide<6->{
\quad\Leftrightarrow\quad
    {\color{red}\frac{\Delta}{\noise} \ge \shape}.
}
\end{align*}
%
\begin{itemize}
\onslide<2->{
\item The ``signal'' $\Delta$ is the smallest change that reverses your conclusion
}
\onslide<4->{
\item The ``noise''
    $\noise^2 \rightarrow
        \underset{N \rightarrow \infty}{\mathrm{lim}}
            \mathrm{Var}(\sqrt{N}\phi)$
    (``sandwich'' variance estimator)
}
\onslide<5->{
\item The ``shape''
    $\shape$
    $\rightarrow$ a nonzero constant and is
    $\le \sqrt{\alpha (1 - \alpha)}$
}
\end{itemize}

\hrulefill

\onslide<7->{
\textbf{Contrast with sampling variability.}

\vspace{0.5em}
A 95\% CI is given by
$\thetafun(\thetahat(\onevec)) \pm \frac{1.96}{\sqrt{N}} \noise.$
%
We reject $\thetafun(\thetahat(\onevec)) + \Delta$ when
%
\begin{align*}
%
\thetafun(\thetahat(\onevec)) + \Delta \ge
\thetafun(\thetahat(\onevec)) + \frac{1.96}{\sqrt{N}} \noise
\onslide<8>{
\quad\quad
\Leftrightarrow
\quad\quad
{\color{red}
\frac{\Delta}{\noise} \ge \frac{1.96}{\sqrt{N}.}
}
}
%
\end{align*}
%
}


\hrulefill

%\vspace{1em}
\onslide<6->{
The \textbf{signal to noise ratio} $\frac{\Delta}{\noise}$
determines robustness to data dropping
}
\onslide<1-7>{...}
\onslide<8>{
\textbf{and} sampling variability, but with \textbf{different thresholds}.
}

\end{frame}



%%%%%%%%%%%%%%%%%%%%%%%%%%%%%%%%%%%%%%%%%%%%%%%%%%%%%%%%%%%%%%%%%%%%%%%%%
%%%%%%%%%%%%%%%%%%%%%%%%%%%%%%%%%%%%%%%%%%%%%%%%%%%%%%%%%%%%%%%%%%%%%%%%%
%%%%%%%%%%%%%%%%%%%%%%%%%%%%%%%%%%%%%%%%%%%%%%%%%%%%%%%%%%%%%%%%%%%%%%%%%

\begin{frame}[t]{What makes an analysis sensitive?}
%
\begin{minipage}{0.45\textwidth}
\begin{center}
    Robust to data dropping:\\
    (``dropping robustness'')\\
    \vspace{1em}
    $\textrm{SNR} := \frac{\Delta}{\noise} \ge \shape$
\end{center}
\end{minipage}
%
\begin{minipage}{0.45\textwidth}
\begin{center}
    Robust to sampling variation:\\
    (``sampling robustness'')\\
    \vspace{1em}
    $\textrm{SNR} := \frac{\Delta}{\noise} \ge
        \frac{1.96}{\sqrt{N}}$
\end{center}
\end{minipage}

\vspace{1em}

\hrulefill

\pause
\vspace{0.5em} $\bullet\quad$
\textbf{Dropping robustness $\ne$ sampling robustness in general.\\}
\textit{Proof: }
$\shape \ne \frac{1.96}{\sqrt{N}}$.

\pause
\vspace{0.5em} $\bullet\quad$
\textbf{When the SNR is small, sufficiently large $N$
produces sampling robustness, but not necessarily
dropping robustness.\\}
\textit{Proof: }
$\frac{1.96}{\sqrt{N}} \rightarrow 0$, but $\shape \rightarrow$ a nonzero
constant.

\pause
\vspace{0.5em} $\bullet\quad$
\textbf{Statistical insignificance is dropping non-robust for large $N$.\\}
\textit{Proof: }
%
Insignificance means
$|\thetafun(\thetahat(\onevec))| \le \frac{1.96}{\sqrt{N}} \noise$.

$\Rightarrow$ A result can be made significant by a change of no more than
$\frac{1.96}{\sqrt{N}} \noise$.

$\Rightarrow$ The SNR for a conclusion
of ``insignificance'' is $\frac{\Delta}{\noise} \le \frac{1.96}{\sqrt{N}}
\rightarrow 0 \le \shape$.

\pause
\vspace{0.5em} $\bullet\quad$
\textbf{P-hacking is dropping non-robust for large $N$.\\}
\textit{Proof: }P-hacked effect sizes are of the order
$\frac{1.96}{\sqrt{N}} \noise$.

\end{frame}



%%%%%%%%%%%%%%%%%%%%%%%%%%%%%%%%%%%%%%%%%%%%%%%%%%%%%%%%%%%%%%%%%%%%%%%%%
%%%%%%%%%%%%%%%%%%%%%%%%%%%%%%%%%%%%%%%%%%%%%%%%%%%%%%%%%%%%%%%%%%%%%%%%%
%%%%%%%%%%%%%%%%%%%%%%%%%%%%%%%%%%%%%%%%%%%%%%%%%%%%%%%%%%%%%%%%%%%%%%%%%

\begin{frame}[t]{What makes an analysis sensitive?}
%
\begin{minipage}[t]{0.45\textwidth}
\begin{center}
    Robust to data dropping:\\
    (``dropping robustness'')\\
    \vspace{1em}
    $\textrm{SNR} := \frac{\Delta}{\noise} \ge \shape$
\end{center}
\end{minipage}
%
\begin{minipage}[t]{0.45\textwidth}
\begin{center}
    Robust to gross errors:\\
    (``gross error robustness'')\\
    \vspace{1em}
    Gross outliers cannot produce
    arbitrarily large changes to $\thetafun$.
\end{center}
\end{minipage}

\vspace{1em}
\hrulefill

\pause
\vspace{1em} $\bullet\quad$
\textbf{Dropping non-robustness is not driven by misspecification.\\}
\textit{Proof: }
Small $\Delta$ are dropping non-robust irrespective of specification.

\pause
\vspace{1em} $\bullet\quad$
\textbf{Gross outliers primarily affect dropping robustness through $\noise$.\\}
\textit{Proof: }
For a fixed $\noise$, outliers decrease $\shape$.
(Details in paper.)

\end{frame}




%%%%%%%%%%%%%%%%%%%%%%%%%%%%%%%%%%%%%%%%%%%%%%%%%%%%%%%%%%%%%%%%%%%%%%%%%
%%%%%%%%%%%%%%%%%%%%%%%%%%%%%%%%%%%%%%%%%%%%%%%%%%%%%%%%%%%%%%%%%%%%%%%%%
%%%%%%%%%%%%%%%%%%%%%%%%%%%%%%%%%%%%%%%%%%%%%%%%%%%%%%%%%%%%%%%%%%%%%%%%%

\begin{frame}[t]{How to make an analysis less sensitive?}

\begin{center}
    Robust to data dropping:\\
    (``dropping robustness'')\\
    \vspace{1em}
    $\textrm{SNR} := \frac{\Delta}{\noise} \ge \shape$
\end{center}

\vspace{1em}
\hrulefill

\vspace{1em}
\textbf{To achieve dropping robustness,
reduce $\noise$ and / or increase $\Delta$.\\}
\textit{Proof: }
Across typical distributions, $\shape$ varies little.
(Details in paper.)

% \vspace{1em}
% \hrulefill

\pause
\vspace{1em}
In the Mexico microcredit example,
%
\begin{align*}
%
\noise = \MxNoise
\quad\quad\quad
\thetafun(\thetahat(\onevec)) = \MxBetahat
\quad\quad\quad
N = \MxNobs
%
\end{align*}
%
The study overcame a very low signal to noise ratio with a very large $N$.

\vspace{1em} This (canonical) response to low signal to noise ratio --- to
gather more data --- produces small SEs, but cannot produce dropping
robustness.

\end{frame}
