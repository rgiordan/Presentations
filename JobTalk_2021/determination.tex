




%%%%%%%%%%%%%%%%%%%%%%%%%%%%%%%%%%%%%%%%%%%%%%%%%%%%%%%%%%%%%%%%%%%%%%%%%
%%%%%%%%%%%%%%%%%%%%%%%%%%%%%%%%%%%%%%%%%%%%%%%%%%%%%%%%%%%%%%%%%%%%%%%%%
%%%%%%%%%%%%%%%%%%%%%%%%%%%%%%%%%%%%%%%%%%%%%%%%%%%%%%%%%%%%%%%%%%%%%%%%%

\begin{frame}{What makes an analysis sensitive?}

Suppose we are willing to drop $\alphan$ points.
%
We are ``sensitive to data dropping'' if, for some $\Delta$ large enough to
change our conclusions,
%
\begin{align*}
% $
%
\textrm{``Signal''} :=
\Delta < \thetafunlin(\w^*) - \thetafun(\thetahat)  =
    {\color{red}- \sum_{n=1}^{\lfloor \alpha N \rfloor} \infl_{(n)} }
    =:  {\color{red} \noise } { \color{red} \shape}
%
% $
\end{align*}
%
% \hspace{1em} where \vspace{1em}

\begin{itemize}
\item The ``noise'' $\noise^2 \rightarrow \mathrm{Var}(\sqrt{N}\phi)$
    (``sandwich'' variance estimator)
    % \begin{itemize}
    %     \item $\noise^2 = $ is the robust ``sandwich'' variance estimator
    %     \citep{hampel1986robustbook}
    % \end{itemize}
\item The ``shape''
    $\shape := \frac{-\sum_{n=1}^{\lfloor \alpha N \rfloor} \infl_{(n)}}{\noise}$
    $\rightarrow$ nonzero constant
    $\le \sqrt{\alpha (1 - \alpha)}$
    % determined by $\infl_n$ distribution
    % \begin{itemize}
    %     \item $\shape$
    % \end{itemize}
\end{itemize}

\begin{center}
\begin{minipage}{0.8\textwidth}
\begin{tikzpicture}
    \node[anchor=south west,inner sep=0] (image) at (0,0) {
    \SimInflHistogram{}
    };
    \begin{scope}[x={(image.south east)},y={(image.north west)}]
        \draw [stealth-stealth][thick][white](0.42, 0.35) -- (0.6, 0.35);
        \draw (0.51, 0.35) node[below][text width=3cm][align=center][white]
            {\normalsize $\noise$};

        \draw [stealth-][thick][red](0.3, 0.25) -- (0.3, 0.5);
        \draw (0.25, 0.5) node[above][text width=3cm][align=center][red]
            {\normalsize $-\sum_{n=1}^{\alphan} \infl_{(n)}$};
    \end{scope}
\end{tikzpicture}
\end{minipage}
\end{center}




\end{frame}



%%%%%%%%%%%%%%%%%%%%%%%%%%%%%%%%%%%%%%%%%%%%%%%%%%%%%%%%%%%%%%%%%%%%%%%%%
%%%%%%%%%%%%%%%%%%%%%%%%%%%%%%%%%%%%%%%%%%%%%%%%%%%%%%%%%%%%%%%%%%%%%%%%%
%%%%%%%%%%%%%%%%%%%%%%%%%%%%%%%%%%%%%%%%%%%%%%%%%%%%%%%%%%%%%%%%%%%%%%%%%

\begin{frame}[t]{Example.}

%
\begin{align*}
%
\alpha :={}& \textrm{Proportion of points to drop}\\
\Delta :={}& \textrm{Signal (difference large enough to change conclusions)}\\
\noise :={}& \textrm{Noise (estimator of }\var{}{\sqrt{N} \thetafun}\textrm{)}\\
\shape :={}& \textrm{Shape (some constant determined by }\alpha\textrm{ and } N\infl_n\textrm{ tail shape)}\\
%
\end{align*}
%

Sensitive to data dropping if:
%
\begin{align*}
%
\thetafunlin(\w^*) - \thetafun(\thetahat)  = \noise \shape \ge \Delta
\quad\quad
\Leftrightarrow
\quad\quad
\frac{\Delta}{\noise} \le \shape.
%
\end{align*}
%
The \textbf{signal to noise ratio} $\frac{\Delta}{\noise}$
determines sensitivity to data dropping.

\hrulefill

\onslide<1->{
\footnotesize
\OHIEResultsTable{}
}
\vspace{-1em}
Let's analyze with $\alpha = 0.01 = 1\%$.
%
\begin{align*}
%
\begin{array}{rllrll}
\thetafun(\thetahat) ={}& -0.029  &\textrm{(Increase QOI by defn)}
    &   \Delta ={}& 0.029 \\
\noise ={}& 0.766 & \textrm{(Noise)}
        & \frac{1}{\sqrt{N}} \noise ={}& 0.005 &
        \textrm{(SE)} \\
\shape ={}& 0.046   &   \textrm{(Shape)}
    & \frac{1.96}{\sqrt{N}}  ={}& 0.0128
    & \rightarrow 0 \textrm{ as }N \rightarrow \infty\\
\shape \noise ={}& 0.035 & \textrm{(Data dropping sensitivity)}
    & \frac{1.96}{\sqrt{N}} \noise  ={}& 0.010
    & \textrm{(SE sensitivity)}\\
\end{array}
% %
\end{align*}
%
The noise is much larger than the signal $\Rightarrow$
Sensitive to data dropping.

\end{frame}


%%%%%%%%%%%%%%%%%%%%%%%%%%%%%%%%%%%%%%%%%%%%%%%%%%%%%%%%%%%%%%%%%%%%%%%%%
%%%%%%%%%%%%%%%%%%%%%%%%%%%%%%%%%%%%%%%%%%%%%%%%%%%%%%%%%%%%%%%%%%%%%%%%%
%%%%%%%%%%%%%%%%%%%%%%%%%%%%%%%%%%%%%%%%%%%%%%%%%%%%%%%%%%%%%%%%%%%%%%%%%

\begin{frame}[t]{Corollaries.}
%
Report non-robustness if:
%
\begin{align*}
%
\thetafunlin(\w^*) - \thetafun(\thetahat)  = \noise \shape \ge \Delta
\quad\quad
\Leftrightarrow
\quad\quad
\frac{\Delta}{\noise} \le \shape.
%
\end{align*}
%
The \textbf{signal to noise ratio} $\frac{\Delta}{\noise}$
determines sensitivity to data dropping.


\hrulefill

%\vspace{-1em}

\pause
\vspace{0.5em}
\textbf{Corollary:  Leave-$\lfloor \alpha N \rfloor$-out is different from standard errors.}\\
% \vspace{-0.4em}
% $1.96 / \sqrt{N} \ne \shape$
% \vspace{0.5em}
Standard errors reject when
$\frac{\Delta}{\noise} \le \frac{1.96}{\sqrt{N}} \ne \shape$.

\pause
\vspace{0.5em}
\textbf{Corollary:  Statistical insignificance is asymptotically non-robust.}
%\vspace{-0.4em}

$\frac{1.96 \hat \sigma_\phi}{\sqrt{N}} \rightarrow 0 \le \shape$.

\pause
\vspace{0.5em}
\textbf{Corollary:  Leave-$\lfloor \alpha N \rfloor$-out robustness does not vanish as $N \rightarrow \infty$.}\\
% \vspace{-0.4em}
%
Both $\shape$ and $\noise$ typically converge to nonzero constants.

\pause
\vspace{0.5em}
\textbf{Corollary:  Non-robustness possible even with correct specification.}
%\vspace{-0.4em}
% {\small The noise $\noise$ may be larger than the effect
% $\Delta$ you're trying to measure.}
%
% \pause
% \vspace{0.5em}
% \textbf{Corollary:  Gross outliers primarily affect robustness
% through $\noise$.}
% %\vspace{-0.4em}
% See paper for more details.

\pause
\vspace{0.5em}
\textbf{Corollary:  To robustify, reduce the noise or increase
the signal.}

\end{frame}
