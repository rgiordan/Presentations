
\begin{frame}

\begin{center}
\large
\textbf{
An Automatic Finite-Sample Robustness Metric:
\\Can Dropping a Little Data Make a Big Difference?}
\end{center}

\hrulefill

\onslide<2->{
\vspace{1em} The conclusions of one's statistical analysis may depend on only a
\textbf{small fraction of the data}, even for \textbf{highly significant results
in correctly specified models}.
}

\onslide<3->{ \vspace{1em} We provide a \textbf{generally applicable tool} to
detect such sensitivity.  Our methods are \textbf{efficiently and automatically
computable}, and come with \textbf{finite-sample accuracy guarantees} and
\textbf{clear intuition}. }

\hrulefill
\vspace{1em}

Ryan Giordano (\texttt{rgiordan@mit.edu})\footnotemark[1]

January 2022

\footnotetext[1]{With coauthors Rachael Meager (LSE) and Tamara Broderick (MIT)}

\end{frame}

%%%%%%%%%%%%%%%%%%%%%%%%%%%%%%%%%%%%%%%%%%%%%%%%%%%%%%%%%%%%%%%%%%%%%%%%
%%%%%%%%%%%%%%%%%%%%%%%%%%%%%%%%%%%%%%%%%%%%%%%%%%%%%%%%%%%%%%%%%%%%%%%%
%%%%%%%%%%%%%%%%%%%%%%%%%%%%%%%%%%%%%%%%%%%%%%%%%%%%%%%%%%%%%%%%%%%%%%%%

\begin{frame}[t]{Dropping data: Mexico Microcredit}

% The success of statistics has led to data from particular populations
% being used increasingly to draw general conclusions.
% %
% Drawing the right conclusions affects peoples' lives!

\vspace{1em}
\textbf{Example:} \citet{angelucci2015microcredit}, a randomized controlled
trial study of the efficacy of microcredit in Mexico based on 16,560 data
points.

The variable ``Beta" estimates the effect of microcredit in US dollars.

%\MicrocreditMexicoRerunTable{}

%\hspace{0.05em}

\begin{table}[ht]
\centering
\begin{tabular}{rll} \hline
  & \onslide<2->{Left out points} & Beta (SE) \\ \hline
Original result & \onslide<2->{0} & -4.55 (5.88) \\ \hline
\onslide<3->{``Significant'' negative change & 14 & -10.96 (5.57) \\ \hline}
\onslide<4->{``Significant'' positive change & 15 & 7.03 (2.55) \\ \hline}
\end{tabular}
\end{table}

\vspace{-1em}
\onslide<1-> { \textbf{The original conclusion: }
No evidence that microcredit is effective...
}
\onslide<5->{\\...can be reversed by dropping less than 0.1\% of the data.}

\vspace{1em}
$\Rightarrow$ Standard errors can be inadequate summaries of data sensitivity!

\hrulefill
\vspace{1em}

Cannot find influential subsets by brute force!
${16,560 \choose 15} \approx 1.5 \cdot 10^{51}$

\vspace{1em}
\textbf{We provide a fast, automatic tool to  approximately identify the
most influential set of points.}


\end{frame}


% %%%%%%%%%%%%%%%%%%%%%%%%%%%%%%%%%%%%%%%%%%%%%%%%%%%%%%%%%%%%%%%%%%%%%%%%
% %%%%%%%%%%%%%%%%%%%%%%%%%%%%%%%%%%%%%%%%%%%%%%%%%%%%%%%%%%%%%%%%%%%%%%%%
% %%%%%%%%%%%%%%%%%%%%%%%%%%%%%%%%%%%%%%%%%%%%%%%%%%%%%%%%%%%%%%%%%%%%%%%%


\begin{frame}{Outline}
\begin{itemize}
    \item Why and when might you care about sensitivity to data dropping?
    \item How does our approximation work, and when is it accurate?
    \begin{itemize}
        \item[] (A formalization of the problem and the
            class of estimators we study.)
    \end{itemize}
    \item Examine real-life examples of analyses: some sensitive, some not.
    \begin{itemize}
        \item[] (The results may defy your intuition.)
    \end{itemize}
    \item What kinds of analyses are sensitive to data dropping?
    \begin{itemize}
        \item[] (Including comparison to standard errors and
            gross-error robustness.)
    \end{itemize}
\end{itemize}
\end{frame}

% %%%%%%%%%%%%%%%%%%%%%%%%%%%%%%%%%%%%%%%%%%%%%%%%%%%%%%%%%%%%%%%%%%%%%%%%
% %%%%%%%%%%%%%%%%%%%%%%%%%%%%%%%%%%%%%%%%%%%%%%%%%%%%%%%%%%%%%%%%%%%%%%%%
% %%%%%%%%%%%%%%%%%%%%%%%%%%%%%%%%%%%%%%%%%%%%%%%%%%%%%%%%%%%%%%%%%%%%%%%%

\begin{frame}{Dropping data: Motivation}

Would you be concerned if you could \textbf{reverse your conclusion} by removing
a \textbf{small proportion} of your data?

Not always!  But sometimes, surely yes.

Thinking without random noise can be helpful.

Suppose you have a farm, and want to know whether
your average yield is $>$ 170 bushels per acre.
At harvest, you measure 200 bushels per acre.

\begin{itemize}
    \item Scenario one: If your yield is greater than 170 bushels
        per acre, you make a profit.
        \begin{itemize}
            \item Don’t care about sensitivity to small subsets
        \end{itemize}
    \item Scenario two: You want to recommend your farming
    methods to a friend across the valley.
    \begin{itemize}
        \item Might care about sensitivity to small subsets
    \end{itemize}
\end{itemize}

For example, often in economics:
%
\begin{itemize}
\item Policy population is different from analyzed population,
\item Small fractions of data are missing not-at-random,
\item We report a convenient summary (e.g. mean) of a complex effect.
% \item Models are stylized proxies of reality.
\end{itemize}


\end{frame}
