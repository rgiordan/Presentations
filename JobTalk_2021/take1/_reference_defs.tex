
%%%%%%%%%%%%%%%%%%%%
% amsthm commands

\theoremstyle{plain}
\newtheorem{lem}{Lemma}
\newtheorem{thm}{Theorem}
\newtheorem{prop}{Proposition}
\newtheorem{cond}{Condition}
\newtheorem{assu}{Assumption}
\newtheorem{cor}{Corollary}
\newtheorem{conj}{Conjecture}

\theoremstyle{definition}
\newtheorem{defn}{Definition}

%%%%%%%%%%%%%%%%%%%%%%%%%%%%%

% Example:
% https://latex.org/forum/viewtopic.php?t=10815

\newcounter{point}[subsubsection]
\setcounter{point}{1}

\newenvironment{point}[1]%
{
\refstepcounter{point}
\textbf{(\alph{point}) #1.}
}%
{}

\newcommand{\pointlabel}[1]{\label{point:#1}}
\renewcommand{\thepoint}{(\alph{point})}
\newcommand{\pointref}[1]{Point \ref{point:#1}}
\newcommand{\secpointref}[2]{\secref{#1}, paragraph \ref{point:#2}}

% To reference a point, use
% \secpointref{why}{signal_noise_shape}
% Unfortunately, you have to specify the section manyally.  There does not
% seem to be an easy way to get the section of a reference in LaTeX.
% (The zref package might be a way but it looks complicated.)

%%%%%%%%%%%%%%%%%%%%
% refstyle commands

\newref{sec}{
    name=Section~, %
    Name=Section~,
    names=Sections~, %
    Names=Sections~
    }

\newref{app}{
    name=Appendix~, %
    Name=Appendix~
    }

\newref{eq}{
    name=Eq.~, %
    Name=Eq.~,
    names=Eqs.~, %
    Names=Eqs.~
    }

\newref{fig}{
    name=Figure~, %
    Name=Figure~,
    names=Figures~, %
    Names=Figures~
    }

\newref{table}{
    name=Table~, %
    Name=Table~,
    names=Tables~, %
    Names=Tables~
    }

\newref{def}{
    name=Definition~, %
    Name=Definition~,
    names=Definitions~, %
    Names=Definitions~
    }

\newref{assu}{
    name=Assumption~, %
    Name=Assumption~,
    names=Assumptions~, %
    Names=Assumptions~,
    }

\newref{cond}{
    name=Condition~, %
    Name=Condition~,
    names=Conditions~, %
    Names=Conditions~
    }

\newref{prop}{
    name=Proposition~, %
    Name=Proposition~,
    names=Propositions~, %
    Names=Propositions~
    }

\newref{lem}{
    name=Lemma~, %
    Name=Lemma~
    }


\newref{cor}{
    name=Corollary~, %
    Name=Corollary~,
    names=Corollaries~,
    Names=Corollaries
    }

\newref{thm}{
    name=Theorem~, %
    Name=Theorem~
    }

\newref{proof}{
    name=Proof~, %
    Name=Proof~
    }

\newref{conj}{
    name=Conjecture~, %
    Name=Conjecture~
    }

\newref{algr}{
    name=Algorithm~, %
    Name=Algorithm~,
    names=Algorithms~, %
    Names=Algorithms~,
    }


%% refstyle examples:
% \Secref[vref]{introduction} contains \secref{introduction}.
% \Secref[vref]{ack} does not contain \secref{introduction}.
%
% \begin{align}
%     x=y \eqlabel{myeq}
% \end{align}


% Maybe you should do something like this.
% https://tex.stackexchange.com/questions/201956/how-do-i-make-all-the-theorem-alike-environments-have-an-ending-symbol
%
% \newtheorem{eg}[theorem]{Example}
% \newenvironment{example}[1][]{\begin{eg}[#1]\pushQED{\hfill $\blacksquare$}}{\popQED \end{eg}}
