\begin{frame}{Dropping data: Motivation}

More data \& cheaper computation $\Rightarrow$\\
Statistical analyses are
playing larger roles in decision making.

\vspace{1em}
Decisions are important: We want \textbf{trustworthy} conclusions.\\
Data / models not always perfect: We want \textbf{robust} conclusions.

% \begin{itemize}
%     \item Gather + clean exchangeable data,
%     \item Specify and fit a model, and
%     \item Drawn a qualitative conclusion from your fit
%     \\(e.g., based on the sign / significance of
%         some estimated parameter).
% \end{itemize}

\vspace{1em}
Would you be concerned if you could \textbf{reverse your conclusion} by removing
a \textbf{small proportion} (say, $0.1\%$) of your data?

\vspace{1em}
\textbf{Running example:} \citet{angelucci2015microcredit}, a
randomized controlled trial study of the efficacy of microcredit based
on 16,560 data points.  \\We can reverse the studies qualitative
conclusions by removing $15$ observations ($< 0.1 \%$ of the data).

\vspace{1em}
\textbf{How do we find sets of influential points?}  Difficult in general!\\

\vspace{1em}
We provide a \textbf{automatic approximation} with finite-sample
guarantees.

\vspace{1em}
The approximation gives the \textbf{causes of sensitivity} to data dropping.

\end{frame}

%%%%%%%%%%%%%%%%%%%%%%%%%%%%%%%%%%%%%%%%%%%%%%%%%%%%%%%%%%%%%%%%%%%%%%%%
%%%%%%%%%%%%%%%%%%%%%%%%%%%%%%%%%%%%%%%%%%%%%%%%%%%%%%%%%%%%%%%%%%%%%%%%
%%%%%%%%%%%%%%%%%%%%%%%%%%%%%%%%%%%%%%%%%%%%%%%%%%%%%%%%%%%%%%%%%%%%%%%%

\begin{frame}{Dropping data: Mexico Microcredit}

Consider \citet{angelucci2015microcredit}, a randomized controlled trial study
of the efficacy of microcredit in Mexico based on 16,560 data points.

The variable ``Beta" estimates the effect of microcredit in US dollars.

%\MicrocreditMexicoRerunTable{}

\begin{table}[ht]
\centering
\begin{tabular}{lll} \hline
  & \onslide<2->{Left out points} & Beta (SE) \\\hline
\hspace{0.05em} Original result & \onslide<2->{0} & -4.55 (5.88) \\ \hline
\onslide<2-> {\hspace{0.05em} Change sign & 1 & 0.4 (3.19) \\\hline }
\onslide<3-> {Change significance & 14 & -10.96 (5.57) \\\hline }
\onslide<4-> {Change sign and significance & 15 & 7.03 (2.55) \\\hline }
\end{tabular}
\end{table}

\vspace{-1em}
\onslide<1-> { \textbf{Original conclusion: }\\
There is no evidence that microcredit is effective.}

\vspace{1em}
\onslide<5-> { \textbf{Potential conclusions after data dropping: }\\
The effect of microcredit is positive (negative) \&
statistically significant.}

% \vspace{1em}
% \onslide<6-> { By removing very few data points ($15 / 16560 \approx 0.1\% $),
% we can reverse the qualitative conclusions of the original study! }

\vspace{1em}
\onslide<6-> { \textbf{The culprit is signal to noise ratio. }\\
By the end of the talk, we will see that the sensitivity is due to
\begin{itemize}
\item High variability of the outcome (hosehold profit) relative to
\item A small signal driving the conclusion
(statistical significance)
\end{itemize}
}



\end{frame}

% %%%%%%%%%%%%%%%%%%%%%%%%%%%%%%%%%%%%%%%%%%%%%%%%%%%%%%%%%%%%%%%%%%%%%%%%
% %%%%%%%%%%%%%%%%%%%%%%%%%%%%%%%%%%%%%%%%%%%%%%%%%%%%%%%%%%%%%%%%%%%%%%%%
% %%%%%%%%%%%%%%%%%%%%%%%%%%%%%%%%%%%%%%%%%%%%%%%%%%%%%%%%%%%%%%%%%%%%%%%%

\begin{frame}{Dropping data: Motivation}

Would you be concerned if you could \textbf{reverse your conclusion} by removing
a \textbf{small proportion} (say, $0.1\%$) of your data?

\pause
Not always!  But sometimes, surely yes.

Thinking without random noise can be helpful.

Suppose you have a farm, and want to know whether
your average yield is greater than 170 bushels per acre.
At harvest, you measure 200 bushels per acre.

\begin{itemize}
    \item Scenario one: If your yield is greater than 170 bushels
        per acre, you make a profit.
        \begin{itemize}
            \item Don’t care about sensitivity to small subsets
        \end{itemize}
    \item Scenario two: You want to recommend your farming
    methods to a friend across the valley.
    \begin{itemize}
        \item Might care about sensitivity to small subsets
    \end{itemize}
\end{itemize}

For example, often in economics:
%
\begin{itemize}
\item Small fractions of data are missing not-at-random,
\item Policy population is different from analyzed population,
\item We report a convenient summary (e.g. mean) of a complex effect,
\item Models are stylized proxies of reality.
\end{itemize}


\end{frame}
