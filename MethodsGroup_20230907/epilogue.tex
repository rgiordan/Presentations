

\begin{frame}{}
    \begin{center}
        {\Huge \textbf{Fall 2023 Epilogue}}
    \end{center}
\end{frame}
    
\begin{frame}{Follow-on work: Robustness certification}

    Some authors have been using our methods to argue that their analyses are
    robust \citep{eubank:2022:enfranchisement, martinez:2022:trustdictator,
    finger:2022:pesticidefreeadoption, turnbull:2022:mobilising,
    falck:2022:systematic}.  But, technically, we can prove only non-robustness!

    \vspace{1em}

    For \emph{low-dimensional least squares problems}, some follow-on work has
    proposed more computationally expensive, but sometimes more accurate,
    approximations to the maximally influential set
    \citep{moitra:2022:provablyauditingols, freund:2023:robustnessauditingols}.
%
    % \vspace{1em}
%
    Unlike our work, these alternatives can provide \emph{certificates of
    robustness} rather than our \emph{certificates of non-robustness}.

    \vspace{1em}

    Fast and general methods for certifying robustness --- even 
    in moderate-dimensional linear regression problems --- are still lacking.    

\end{frame}    

\begin{frame}{Follow-on work: High-dimensional Bayesian posteriors}

    \vspace{1em}
    We have also studied (and are still studying) the empirical influence
    function for Bayesian posteriors approximated with MCMC
    \citep{giordano:2023:bayesij}. Accomodating MCMC noise has been an
    interesting challenge.

    \vspace{1em}

    We prove accuracy of the AMIP linear approximation under conditions similar
    to a Bayesian central limit theorem.  
    But we also show that the linear approximation can fail when
    integrating out high-dimensional latent variables!

    \vspace{1em}

    So there is reason to believe AMIP will fail for many Bayes-like estimators
    (e.g. applications of the EM algorithm).  
    Special cases can still work (e.g. some hierarchical models), and our work
    provides a framework for analyzing these special cases.  

    \vspace{1em}

    Much more work is still to be done!

\end{frame}    


% \begin{frame}{Follow-on work: Generalizing \texttt{zaminfluence}}

%     The R package \texttt{zaminfluence} has gotten some attention 

% \end{frame}    
