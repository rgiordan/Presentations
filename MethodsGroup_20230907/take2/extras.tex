


%%%%%%%%%%%%%%%%%%%%%%%%%%%%%%%%%%%%%%%%%%%%%%%%%%%%%%%%%%%%%%%%%%%%%%%%%
%%%%%%%%%%%%%%%%%%%%%%%%%%%%%%%%%%%%%%%%%%%%%%%%%%%%%%%%%%%%%%%%%%%%%%%%%
%%%%%%%%%%%%%%%%%%%%%%%%%%%%%%%%%%%%%%%%%%%%%%%%%%%%%%%%%%%%%%%%%%%%%%%%%


\begin{frame}{A simulation}

For $N=\SimNumObs$ data points, compute the OLS estimator from:

\vspace{1em}
\begin{tabularx}{\textwidth}{YYY}
%\begin{tabular}{ccc}
    Regressors  &   Residuals   &   Responses \\
    $x_n \sim \mathcal{N}(0, \sigma_x^2)$   &
    $\varepsilon_n \sim \mathcal{N}(0, \sigma_\varepsilon^2)$   &
    $y_n = \SimTrueTheta x_n + \varepsilon_n$
\end{tabularx}
%

\SimGridNormalGraph{}

\end{frame}



%%%%%%%%%%%%%%%%%%%%%%%%%%%%%%%%%%%%%%%%%%%%%%%%%%%%%%%%%%%%%%%%%%%%%
%%%%%%%%%%%%%%%%%%%%%%%%%%%%%%%%%%%%%%%%%%%%%%%%%%%%%%%%%%%%%%%%%%%%%
%%%%%%%%%%%%%%%%%%%%%%%%%%%%%%%%%%%%%%%%%%%%%%%%%%%%%%%%%%%%%%%%%%%%%


\begin{frame}[t]{Influence function}

The present work is based on the {\em empirical influence function}.
%
Consider:
%
\begin{itemize}
%
\item True, unknown distribution function $\flim(x) = p(X \le x)$
\item Empirical distribution function $\fhat(x) = \meann \ind{\x_n \le x}$
\item A statistical functional $T(F)$.
%
\end{itemize}

\only<2>{
We estimate with $T(\flim)$ with $T(\fhat)$.

Sample means are an example:
%
\begin{align*}
%
T(F) := \int \x \, F(\dee x).
%
\end{align*}
%

Z-estimators are, too:
%
\begin{align*}
%
T(F) := \theta\textrm{  such that  }\int G(\theta, \x) F(\dee x) = 0.
%
\end{align*}
%
}
\only<3->{

Form an (infinite-dimensional) Taylor series expansion at some $F_0$:
%
\begin{align*}
%
T(F) = T(F_0) + T'(F_0) (F - F_0) + \mathrm{residual}.
%
\end{align*}
%
When the derivative operator takes the form of an integral
%
\begin{align*}
%
T'(F_0) \Delta = \int \infl(\x; F_0) \Delta(\dee x)
%
\end{align*}
%
then $\infl(\x; F_0)$ is known as the {\em influence function}.

{
Where to form the expansion? There are at least two reasonable choices:
%
\begin{itemize}
%
\item The limiting influence function $\infl(\x, \flim)$
\item The empirical influence function $\infl(\x, \fhat)$
%
\end{itemize}
%
}

}


\end{frame}



%%%%%%%%%%%%%%%%%%%%%%%%%%%%%%%%%%%%%%%%%%%%%%%%%%%%%%%%%%%%%%%%%%%%%
%%%%%%%%%%%%%%%%%%%%%%%%%%%%%%%%%%%%%%%%%%%%%%%%%%%%%%%%%%%%%%%%%%%%%
%%%%%%%%%%%%%%%%%%%%%%%%%%%%%%%%%%%%%%%%%%%%%%%%%%%%%%%%%%%%%%%%%%%%%


\begin{frame}[t]{Influence function}

\begin{itemize}
%
\item The limiting influence function (LIF) $\infl(\x, \flim)$
    \begin{itemize}
        \item Used in a lot of classical statistics
            \citep{
                mises1947asymptotic,
                huber1981robust, hampel1986robustbook,
                bickel1993semiparametric}
        \item Unobserved, asymptotic
        \item Requires careful functional analysis
            \citep{reeds1976thesis}
    \end{itemize}
\item The empirical influence function (EIF) $\infl(\x, \fhat)$
    \begin{itemize}
        \item The basis of the present work
            (also \citep{giordano2019swiss, giordano2019higherorder})
        \item Computable, finite-sample
        \item Requires only finite-dimensional calculus
    \end{itemize}
%
\end{itemize}

\vspace{1em}
Typically the {\em semantics} of the EIF derive from study of the LIF.

Example: $\meann (N \infl_n)^2 \approx \var{}{\sqrt{N}\thetafun(\thetahat)}$.

\vspace{1em}
But the EIF measures what happens when you perturb the data at hand.

\vspace{1em}
Other data perturbations will admit an analysis similar to ours!

\end{frame}



%%%%%%%%%%%%%%%%%%%%%%%%%%%%%%%%%%%%%%%%%%%%%%%%%%%%%%%%%%%%%%%%%%%%%
%%%%%%%%%%%%%%%%%%%%%%%%%%%%%%%%%%%%%%%%%%%%%%%%%%%%%%%%%%%%%%%%%%%%%
%%%%%%%%%%%%%%%%%%%%%%%%%%%%%%%%%%%%%%%%%%%%%%%%%%%%%%%%%%%%%%%%%%%%%


\begin{frame}{Local robustness}

The present work is an application of {\em local robustness}.  Consider:
%
\begin{itemize}
%
\item Model parameter $\lambda$  (e.g., data weights $\lambda = \w$)
\item Set of plausible models $\mathcal{S}_\lambda$
    (e.g. $\mathcal{S}_\lambda = W_\alpha$)
\item Estimator $\thetahat(x, \lambda)$ for data $\x$ and
    $\lambda \in \mathcal{S}_\lambda$
    (e.g. a Z-estimator)
%
\end{itemize}
%
% Specify data $x$, a set of plausible models $\lambda \in \mathcal{S}_\lambda$,
% and an estimator $\thetahat(x, \lambda)$.

\hrulefill

\begin{tabular}{ccc}
    Global robustness:   &
%
$
\left(
\inf_{\lambda \in \mathcal{S}_\lambda} \thetahat(x, \lambda),
\sup_{\lambda \in \mathcal{S}_\lambda} \thetahat(x, \lambda)
\right)
$
& (Hard in general!)
\end{tabular}


\hrulefill

\begin{tabular}{cc}
Local robustness:
&
$
\left(
\inf_{\lambda \in \mathcal{S}_\lambda} \thetahat^{lin}(x, \lambda),
\sup_{\lambda \in \mathcal{S}_\lambda} \thetahat^{lin}(x, \lambda)
\right)
$
%
\end{tabular}

...where $\thetahat^{lin}(\x, \lambda) :=
\thetahat^{lin}(\x, \lambda_0) +
    \fracat{\partial \thetahat^{lin}(\x, \lambda)}{\partial \lambda}{\lambda_0}
        (\lambda  - \lambda_0)$.

\hrulefill

\textbf{Many variants are possible!}

\begin{itemize}
    \item Cross-validation \citep{giordano2019swiss}
    \item Prior sensitivity in Bayesian nonparametrics \citep{giordano2021bnp}
    \item Model sensitivity of MCMC output \citep{giordano2018covariances}
    \item Frequentist variances of MCMC posteriors (in progress)
\end{itemize}

% My work emphasizes \textbf{efficient computation of the derivative}
% and \textbf{accuracy as an approximation to global robustness}.

\end{frame}
