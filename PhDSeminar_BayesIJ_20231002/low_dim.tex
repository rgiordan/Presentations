
\begin{frame}[t]{Low dimensional problems}

Let's consider models which obey a \textbf{Bayesian central limit theorem} (BCLT).

% Example: \textbf{Overdispersed Poisson (negative binomial) observations.}
Example: \textbf{Negative binomial models with an unkown parameter
$\gamma$.}

%
% \begin{minipage}[t]{0.45\textwidth}
% %
% \begin{align*}
% %
%     &\textrm{Fix }\alpha\text{, }\beta.
%     \textrm{  For }n={} 1,\ldots,N\textrm{, }\\
%     &\x_n |\gamma \iid{}
%         \mathrm{NegativeBinomial}\left(
%             \alpha,  \frac{\beta}{\gamma + \beta}\right).
% %
% \end{align*}
% %
% \end{minipage}
% %
% \begin{minipage}[t]{0.45\textwidth}
% %
% \begin{align*}
% %
% %
% &\textrm{Here, }\theta = \gamma \textrm{, }f(\gamma) = \gamma\textrm{, and}\\
% &\log \p(\xvec \vert \gamma, \w) =
%     \sumn \w_n \ell_n(\gamma).\\
% % &\textrm{Why this parameterization? }\\
% % &\textrm{This is the marginal of }\\
% % &\lambda_n \sim \mathrm{Gamma}(\alpha, \beta) \textrm{ and }\\
% % &\x_n \vert \gamma, \lambda \sim \mathrm{Poisson}(\gamma \lambda_n).
% %
% \end{align*}
% %
% \end{minipage}

% \vspace{-1em}
% (This is the marginal of $\lambda_n \sim \mathrm{Gamma}(\alpha, \beta)$ and $\x_n
% \vert \lambda_n \sim \mathrm{Poisson}(\lambda_n)$.)

% \spskip
% \begin{minipage}{0.35\linewidth}
% As $N \rightarrow \infty$, the posterior $\p(\gamma \vert \xvec)$
% \textbf{concentrates} at a point.
%
% \spskip
% Covariances go to zero.
% \end{minipage}
% %
% \begin{minipage}{0.6\linewidth}
% \BCLTDensitiesGraph{}
% \end{minipage}


\onslide<2->{
\hrulefill

\textbf{Bayesian central limit theorem (BCLT) fact: }Suppose that $\p(\gamma
\vert \xvec)$ obeys a BCLT. For functions $\bar{a}(\gamma)$,
$\bar{b}(\gamma)$, $\bar{c}(\gamma)$ satisfying some regularity conditions
\citep{kass:1990:posteriorexpansions},
%
\begin{align*}
%
\expect{\p(\gamma \vert \xvec)}
    {\bar{a}(\gamma) \bar{b}(\gamma)} ={}& O_p(N^{-1})
    &
\expect{\p(\theta \vert \xvec)}
    {\bar{a}(\gamma) \bar{b}(\gamma) \bar{c}(\gamma)} ={}&
    O_p(N^{-2}).
%
\end{align*}
%
\hrulefill
}
%

\onslide<3->{
\begin{align*}
%
\expect{p(\gamma \vert \xvec, \w_n)}{\gamma} -
    \expect{p(\gamma \vert \xvec)}{\gamma} ={}&
\blue{\infl_n} (\w_n - 1) && + \red{\err(\w_n)}
\onslide<2->{
\\={}&
    \undernote{
        \blue{\expect{\p(\gamma \vert \xvec)}{ \gammabar \ellbar_n(\gamma)}}
        }{O_p(N^{-1})}
    (\w_n - 1)
    &&
    + \frac{1}{2}
    \undernote{
    \red{\expect{\p(\gamma \vert \xvec, \wtil_n)}{
        \gammabar
        \ellbar_n(\gamma)
        \ellbar_n(\gamma)}}
    }{O_p(N^{-2})}
    (\w_n - 1)^2
}
\\
% \quad\quad\quad&
% \blue{\infl_n = O_p(N^{-1})} \quad\quad\gg
% && \red{\err(\w_n) = O_p(N^{-2})}
%
\end{align*}

The map $\w_n \mapsto N \left( \expect{\p(\gamma \vert \xvec, \w_n)}{\gamma} -
\expect{\p(\gamma \vert \xvec)}{\gamma} \right)$ becomes linear as $N
\rightarrow \infty$.
}

\end{frame}


%%%%%%%%%%%%%%%%%%%%%%%%%%%%%%%%%%%%%%%%%%%%%%%%%%%%%%%%%%%%%%
%%%%%%%%%%%%%%%%%%%%%%%%%%%%%%%%%%%%%%%%%%%%%%%%%%%%%%%%%%%%%%
%%%%%%%%%%%%%%%%%%%%%%%%%%%%%%%%%%%%%%%%%%%%%%%%%%%%%%%%%%%%%%


%
%
% \begin{frame}[t]{Low dimensional problems}
% %
% \begin{align*}
% %
% \expect{p(\gamma \vert \xvec, \w_n)}{\gamma} -
%     \expect{p(\gamma \vert \xvec)}{\gamma} ={}&
% \blue{\infl_n} (\w_n - 1) && + \red{\err(\w_n)}
% \onslide<2->{
% \\={}&
%     \undernote{
%         \blue{\expect{\p(\gamma \vert \xvec)}{ \gammabar \ellbar_n(\gamma)}}
%         }{O_p(N^{-1})}
%     (\w_n - 1)
%     &&
%     + \frac{1}{2}
%     \undernote{
%     \red{\expect{\p(\gamma \vert \xvec, \wtil_n)}{
%         \gammabar
%         \ellbar_n(\gamma)
%         \ellbar_n(\gamma)}}
%     }{O_p(N^{-2})}
%     (\w_n - 1)^2
% }
% \\
% \quad\quad\quad&
% \blue{\infl_n = O_p(N^{-1})} \quad\quad\gg
% && \red{\err(\w_n) = O_p(N^{-2})}
% %
% \end{align*}
% %
% % \onslide<3->{
% % \spskip
% % \textbf{Theorem (Giordano et al. 2023): }
% % The BCLT result is square summable over $n$.
% % %
% % \begin{align*}
% % %
% % \blue{\meann (N \infl_n)^2 ={} O_p(1)}
% % %
% % \quad\quad\quad\quad
% % \red{\meann (N^2 \err(\w_n))^2 ={} O_p(1)}
% % %
% % \end{align*}
% %
% % \spskip
% % \textbf{Corollary: }
% % The linear approximation is consistent for LOO-CV.
% % }
%
% \end{frame}
%


\begin{frame}{Experiments}

\LowDimAccuracyGraph{}

\end{frame}
