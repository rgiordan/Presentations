
\begin{frame}{High dimensional problems}
    %
    \question{Example: Exponential families  with random effects (REs)
    $\lambda$ and fixed effects $\gamma$.}

    If the observations per random effect remains bounded as $N \rightarrow
    \infty$, then
    %
    \begin{itemize}
    %
    \item Parameter $\lambda$ (``local'') grows in dimension with $N$.
    \item Parameter $\gamma$ (``global'') is finite-dimensional.
    \item Marginally $\p(\lambda \vert \xvec)$ does not concentrate.
    \item Marginally, $\p(\gamma \vert \xvec)$ concentrates.
    % \item Conditionally, $\p(\lambda \vert \gamma, \xvec)$ does not concentrate either.
    %
    \end{itemize}
    

\spskip
\pause
\question{
In general, we cannot hope for an asymptotic analysis of
    $\expect{\p(\lambda, \gamma \vert \xvec)}{f(\lambda)}$. \\
%
\spskip
Can we save the approximation when {\em some} parameters concentrate?\\
%
\spskip
Does the residual vanish asymptotically for
$\w_n \mapsto \expect{\p(\gamma \vert \xvec, \w_n)}{f(\gamma)}$?
}

%
\end{frame}




%%%%%%%%%%%%%%%%%%%%%%%%%%%%%%%%%%%%%%%%%%%%%%%%%%%%%%%%%%%%%%%%%%%%%%%%%%%%%%
%%%%%%%%%%%%%%%%%%%%%%%%%%%%%%%%%%%%%%%%%%%%%%%%%%%%%%%%%%%%%%%%%%%%%%%%%%%%%%
%%%%%%%%%%%%%%%%%%%%%%%%%%%%%%%%%%%%%%%%%%%%%%%%%%%%%%%%%%%%%%%%%%%%%%%%%%%%%%

\begin{frame}{High dimensional problems}
%
We assume that $\p(\gamma \vert \xvec)$ concentrates but
$\p(\lambda \vert \xvec)$ does not.  By our series expansion:
%
\begin{align*}
%
% \MoveEqLeft
&
\expect{\p(\gamma, \lambda \vert \xvec, \w_n)}{\gamma} -
\expect{\p(\gamma, \lambda \vert \xvec)}{\gamma}
={}
\\
&\quad\blue{\infl_n}(\w_n - 1) && +\red{\err(\w_n)}
% \\&
\onslide<2->{
\\={}&
\blue{
\expect{\p(\gamma, \lambda \vert \xvec)}{
    \gammabar \ellbar_n(\gamma,\lambda)}
}
(\w_n - 1)
    && +
    \frac{1}{2}
    \red{
    \expect{\p(\gamma, \lambda \vert \xvec, \wtil_n)}{
        \gammabar
        \ellbar_n(\gamma,\lambda)^2 }
    }
        (\w_n - 1)^2
% \\&
}
\onslide<3->{
\\={}&
%
\blue{
\expectop{\p(\gamma \vert \xvec)} \Big[
    \gammabar
    \undernote{
        \expect{\p(\lambda \vert \gamma, \xvec)}
                {\ellbar_n(\gamma,\lambda)}
            }{F_1(\gamma)}
    \Big]
}
        (\w_n - 1)
    && +
    \frac{1}{2}
    \red{
    \expectop{\p(\gamma \vert \xvec, \wtil_n)} \Big[
        \gammabar
        \undernote{
            \expect{\p(\lambda \vert \xvec, \gamma, \wtil_n)}{
                \ellbar_n(\gamma,\lambda)^2}
            }{F_2(\gamma)}
        \Big]
    }
        (\w_n - 1)^2
}
\onslide<4->{
\\={}&
\undernote{
\blue{
\expect{\p(\gamma \vert \xvec)}{
    \gammabar F_1(\gamma)}
}
}{O_p(N^{-1})\\\text{(by $\p(\gamma \vert \xvec)$ concentration)}}
    (\w_n - 1)
    && +
    \frac{1}{2}
\undernote{
\red{
\expect{\p(\gamma \vert \xvec, \wtil_n)}{
    \gammabar
    F_2(\gamma) }
}
}{O_p(N^{-1})\\\text{(by $\p(\gamma \vert \xvec)$ concentration)}}
        (\w_n - 1)^2
%
\\& \Rightarrow
}
\onslide<5->{
\blue{\infl_n = O_p(N^{-1})} &&
\red{\err(\w_n) = O_p(N^{-1})}
}
%
\end{align*}
%
\onslide<6->{
% The error is the same order as the linear approximation, and does not vanish after rescaling!

\theorem{
    \textbf{Corollary \citep{giordano:2023:bayesij}:}\\
    In general, $\w_n \mapsto N \left( \expect{\p(\gamma \vert \xvec, \w_n)}{\gamma} -
    \expect{\p(\gamma \vert \xvec)}{\gamma} \right)$
    remains non-linear as $N \rightarrow \infty$.    
}

}

\end{frame}

%%%%%%%%%%%%%%%%%%%%%%%%%%%%%%%%%%%%%%%%%%%%%%%%%%%%%%%%%%%%%%%%%%%%%%%%%%%
%%%%%%%%%%%%%%%%%%%%%%%%%%%%%%%%%%%%%%%%%%%%%%%%%%%%%%%%%%%%%%%%%%%%%%%%%%%
%%%%%%%%%%%%%%%%%%%%%%%%%%%%%%%%%%%%%%%%%%%%%%%%%%%%%%%%%%%%%%%%%%%%%%%%%%%

\begin{frame}{Experiments}
%
\question{Example: Poisson regression with Gamma-distributed random effects}
\begin{align*}
\textrm{For }g=1,\ldots,G,\textrm{ }
    \lambda_g \iid{}& \mathrm{Gamma}(\alpha, \beta) \textrm{ for fixed }\alpha,\beta\\
\textrm{For }n = 1,\ldots,N,\textrm{ }
    g_n \iid{}& \mathrm{Categorical}(1,\ldots,G),\textrm{ }
    y_n | \lambda_n, \gamma, g_n \iid{}
        \mathrm{Poisson}(\gamma \lambda_{g_n}).\\
\x_n ={} (y_n, g_n)
\textrm{ are IID given } \lambda, \gamma. &
\textrm{   Write }\log \p(\xvec \vert \lambda, \gamma, \w) =
   \sumn \w_n \ell_n(\lambda, \gamma).
%
\end{align*}

\pause

% \begin{minipage}[t]{0.45\textwidth}
% %
% \begin{align*}
%     % \textrm{Fix }&\alpha,\beta\textrm{. }\\
%     \textrm{For }n={}&1,\ldots,N\textrm{, }g=1,\ldots,G,\\
%     \lambda_g \iid{}& \mathrm{Gamma}(\alpha, \beta) \textrm{ for fixed }\alpha,\beta\\
%     g_n \iid{}& \mathrm{Categorical}(1,\ldots,G)\\
%     y_n \iid{}&
%         \mathrm{Poisson}(\gamma \lambda_{g_n}) 
%         \textrm{ given }
%         \lambda_n, \gamma, g_n\\
% \end{align*}
% %
% \end{minipage}
% \hspace{1em}
% %
% \begin{minipage}[t]{0.45\textwidth}
% %
% \begin{align*}
% %
% & \x_n ={} (y_n, g_n)
% \textrm{ are IID given }\lambda, \gamma.\\
% &\textrm{Write }\log \p(\xvec \vert \lambda, \gamma, \w) =
% % \\
%    \sumn \w_n \ell_n(\lambda, \gamma).
% %
% \end{align*}
% %
% \end{minipage}
%
% \vspace{-2em}
\HighDimAccuracyGraph{}

\end{frame}
    



% von Mises stuff
\def\gdist{\mathbb{G}}
\def\fndist{\mathbb{F}_N}
\def\ftdist{\mathbb{F}_N^{\t}}
\def\fwdist{\mathbb{F}_{N}^{w}}
\def\T{T}   % Stastistical functional
\def\Tlin{T_{\mathrm{lin}}} % linearized functional
\def\t{t}   % Path variable for von Mises expansion
\def\g{g}   % Function of interest
\def\xn{\x_{0}}   % Function of interest
\def\prior{\pi}
\def\ttil{\tilde{\t}}   % IVT value of t
% \def\abs#1{\left\vert #1\right\vert }


%%%%%%%%%%%%%%%%%%%%%%%%%%%%%%%%%%%%%%%%%%%%%%%%%%%%%%%%%%%%%%%%%%%%%%%%%%%
%%%%%%%%%%%%%%%%%%%%%%%%%%%%%%%%%%%%%%%%%%%%%%%%%%%%%%%%%%%%%%%%%%%%%%%%%%%
%%%%%%%%%%%%%%%%%%%%%%%%%%%%%%%%%%%%%%%%%%%%%%%%%%%%%%%%%%%%%%%%%%%%%%%%%%%

\begin{frame}[t]{Bayesian von--Mises Expansion}

\question{How can we apply the single--weight result to variance computations?}

\pause

Define the ``generalized posterior'' functional
%
\begin{align*}
    %
    \T(\gdist, N) := \frac{
        \int \g(\theta) \exp\left( N \int \ell(\xn \vert \theta) \gdist(d\xn)\right)
            \prior(\theta) d\theta
    }
    {
    \int \exp\left( N \int \ell(\xn \vert \theta) \gdist(d\xn)\right)
        \prior(\theta) d\theta
    }.
    %
\end{align*}
%
Let $\fndist$ denote the empirical distribution.  Then
%
\begin{align*}
    \expect{\post}{\g(\theta)} =
    \frac{
        \int \g(\theta) \exp\left( N \meann \ell(\x_n \vert \theta) \right)
            \prior(\theta) d\theta
    }
    {
    \int \exp\left( N \meann \ell(\x_n \vert \theta) \right)
        \prior(\theta) d\theta
    } = \T(\fndist, N).
\end{align*}
%
\pause

Let $\fdist$ denote the true distribution of $\x_n$, and
let $\ftdist  = \t \fndist + (1 - \t) \fdist$. 

We can study the \textit{von Mises expansion}:
%
\begin{align*}
    \sqrt{N} \left(\expect{\post}{\g(\theta)} - \T(\fdist, N) \right) ={}&
        \blue{\sqrt{N} \fracat{\partial \T(\ftdist, N)}{\partial \t}{\t = 0} (\fndist - \fdist)}
        & + \red{\err(\ttil)}
\\={}&
\undernote{
    \blue{\sqrt{N} \sumn (\infl_n - \overline{\infl})} }
    {\textrm{\blue{Infinitesimal jackknife estimator}}} + o_p(1) 
    & + \red{\err(\ttil)}.
\end{align*}
%
%Consistency follows if $\red{\err(\ttil)}$ vanishes.
Inconsistency is suggested if $\red{\err(\ttil)}$ fails to vanish.
%
\end{frame}










\begin{frame}[t]{Bayesian von--Mises Expansion Results}

\theorem{
    \textbf{Theorem 3 \citep{giordano:2023:bayesij} (sketch): }\\
    \textbf{(Consistency of the von--Mises expansion in finite dimensions) }\\
    Under slightly stronger conditions our original finite--dimensional posterior 
    consistency result, 
    %
\begin{align*}
    \sup_{\ttil \in [0,1]} | \red{\err(\ttil)} | \rightarrow 0
    \quad\textrm{in the Bayesian von--Mises expansion.}
\end{align*}
%
}

\pause

\theorem{
\textbf{Theorem 4 \citep{giordano:2023:bayesij} (sketch, not yet on arxiv):}\\
\textbf{(Inconsistency of the von--Mises expansion in infinite dimensions) }\\
    Assume that $\x_n$ comes with a random group assignment $g_n \in 1,\ldots,G$.    
    Conditional on $g$, $\x_n$ is modeled as a finite-dimensional exponential
    family given $\lambda,\gamma$:
    %
    \begin{align*}
        \log \p(x_n \vert g_n=g, \gamma, \lambda) = 
            \tau(x_n)^\trans \eta_{g}(\gamma, \lambda) + \textrm{Constant}.
    \end{align*}
    %
Define the average product of second moments:
%
\begin{align*}
    \mathcal{V_N} :=
    \frac{1}{N} \sum_{g=1}^G 
        \expect{\fdist(\x_n)}{\tau(\x_n) \tau(\x_n)^\trans}
        \cov{\p(\lambda, \gamma \vert \fdist)}{\eta_{g}(\gamma, \lambda)}.
\end{align*}
%
If $\mathcal{V_N}$ is strictly bounded away from $0$ as $N\rightarrow \infty$, then
%
\begin{align*}
    \sup_{\ttil \in [0,1]} | \red{\err(\ttil)} | \rightarrow \infty
    \quad\textrm{in the Bayesian von--Mises expansion.}
\end{align*}
%
}

\end{frame}




\begin{frame}{More experimental results for Gamma--Poisson mixtures}
    \begin{minipage}[c]{0.35\textwidth}
        We ran simulations of the Gamma--Poisson mixture with
        different ratios of $N / G$ \\(average observations per group).

        \spskip
        %
        \begin{itemize}
            \item When $N/G$ is small:
            %
            \begin{itemize}
                \item IJ is biased significantly downwards
                \item Bootstrap is biased somewhat downwards
            \end{itemize}
            %
            \item When $N/G$ is larger:
            %
            \begin{itemize}
                \item Both improve
                \item Both remain somewhat biased
                \item The IJ and bootstrap perform similarly
            \end{itemize}
            %
        \end{itemize}
    %
    \end{minipage}
    \begin{minipage}[c]{0.6\textwidth}
    \PoissonREGraph{}
    \end{minipage}
\end{frame}
