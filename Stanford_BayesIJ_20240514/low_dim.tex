

    
%%%%%%%%%%%%%%%%%%%%%%%%%%%%%%%%%%%%%%%%%%%%%%%%%%%%%%%%%%%%%%
%%%%%%%%%%%%%%%%%%%%%%%%%%%%%%%%%%%%%%%%%%%%%%%%%%%%%%%%%%%%%%
%%%%%%%%%%%%%%%%%%%%%%%%%%%%%%%%%%%%%%%%%%%%%%%%%%%%%%%%%%%%%%



\begin{frame}[t]{Variance consistency theorem}

\question{How do the results for a single weight translate into variance estimates?}
%
\begin{align*}
    \var{\p(\w)}{\expect{p(\theta \vert \xvec, w)}{f(\theta)}}
    = 
    \frac{1}{N^2} \sumn \left(
        \blue{\infl_n - \overline{\infl}}
    \right)^2 +
    \red{\textrm{Term involving }
        \red{\err}(\w_n)
        \textrm{ for }n =1,\dots,N
    }
\end{align*}
%
% Optima and related features of the log likelihood
% \def\scorecov{\Sigma}
% \def\scorecovhat{\hat\Sigma}
\def\thetatrue{\theta_{\infty}}
\def\thetahat{\hat\theta}

% \def\info{\mathcal{I}}
% \def\infoev{\lambda_\mathcal{I}}
% \def\infoevhat{\hat{\lambda}_\mathcal{I}}
% \def\infohat{\hat{\mathcal{I}}}

\def\normdist{\mathcal{N}}

% Covariances of g(\theta)
% \def\gmeantrue{\mu^g}   % The true average posterior expectation
\def\gcovtrue{V^g}
% \def\gcovmaphat{\hat{V}^{\mathrm{MAP}}}
% \def\gcovhat{\hat{V}^{g}}
\def\gcovij{V^{\mathrm{IJ}}}
% \def\gcovijhat{\hat{V}^{\mathrm{IJ}}}
% \def\gcovboothat{\hat{V}^{\mathrm{Boot}}}
% \def\gcovboot{V^{\mathrm{Boot}}}
% \def\gcovbayeshat{\hat{V}^{\mathrm{Bayes}}}
% \def\gcovbayes{V^{\mathrm{Bayes}}}

\pause
% \textbf{Assumptions sketch:}
%
\begin{itemize}
    \item Assume: A well--behaved MAP \textit{maximum a posteriori} estimator $\thetahat$ exists.
    \begin{itemize}
        \item The dimension of $\theta$ is fixed as $N \rightarrow \infty$.
        \item The expected log likelihood has a unique maximum at $\thetatrue$
        \item The observed log likelihood statisfies $\thetahat \rightarrow \thetatrue$
        \item The expected log likelihood Hessian $\info$ is negative definite at $\thetatrue$ 
    \end{itemize}
    \item Assume: We can apply standard asymptotics.
    \begin{itemize}
    \item The log prior and log likelihood are four times continuously differentiable
    \item The prior is proper, and a technical set of squared expectations are finite
    \item The log likelihood derivatives are dominated by a square--integrable envelope function in a
          neighborhood of $\theta_\infty$.
    \end{itemize}
\end{itemize}
%
\pause

\theorem{
\textbf{Theorem 2 \citep{giordano:2023:bayesij}: }
Under the above assumptions,
%
\begin{align*}
\sqrt{N} \left( \expect{\post}{g(\theta)} - g(\thetatrue) \right)
    \dlim{}& \normdist\left(0, \gcovtrue\right)
    %\label{eq:expectation_limit}
    & \textrm{\citep{kleijn:2012:bvm}}\\
\textrm{and} \quad
\gcovij := \frac{1}{N} \sumn \left(
    \blue{\infl_n - \overline{\infl}}
    \right)^2  
    \plim {}& \gcovtrue. 
    & \textrm{(our contribution)}\\
    \nonumber
\end{align*}
}


\end{frame}



%%%%%%%%%%%%%%%%%%%%%%%%%%%%%%%%%%%%%%%%%%%%%%%%%%%%%%%%%%%%%%
%%%%%%%%%%%%%%%%%%%%%%%%%%%%%%%%%%%%%%%%%%%%%%%%%%%%%%%%%%%%%%
%%%%%%%%%%%%%%%%%%%%%%%%%%%%%%%%%%%%%%%%%%%%%%%%%%%%%%%%%%%%%%





\begin{frame}{Negative binomial experiment}

    \question{Example: Negative binomial models with an unknown parameter $\gamma$.}
    %
    \begin{align*}
    \textrm{For }n={} 1,\ldots, N \textrm{ let }
        x_n |  \gamma \iid{}& \textrm{NegativeBinomial}\left(\alpha, \gamma \right)
        \textrm{ for fixed }\alpha.\\
    \textrm{Write }\log \p(\xvec \vert \lambda, \gamma, \w) ={}&
           \sumn \w_n \ell_n(\gamma).
        %
    \end{align*}
    %
    \pause
    %
    \LowDimAccuracyGraph{}
    %   
\end{frame}
        


%%%%%%%%%%%%%%%%%%%%%%%%%%%%%%%%%%%%%%%%%%%%%%%%%%%%%%%%%%%%%%
%%%%%%%%%%%%%%%%%%%%%%%%%%%%%%%%%%%%%%%%%%%%%%%%%%%%%%%%%%%%%%
%%%%%%%%%%%%%%%%%%%%%%%%%%%%%%%%%%%%%%%%%%%%%%%%%%%%%%%%%%%%%%

\begin{frame}{Data Analysis Using Regression and Multilevel/Hierarchical Models.}

    \begin{minipage}{0.44\textwidth}
    
    We ran \texttt{rstanarm} on 56 different models on 13 different datasets
    from \citet{gelman:2006:arm},
    including Gaussian and logistic regression, fixed and mixed-effects models.
    
    \spskip Across all models, we estimate 799 distinct covariances (regression
    coefficients and log scale parameters).
    
    \spskip
    Using the bootstrap as ground truth, compute the relative errors:
    %
    \begin{align*}
    %
    \frac{\gcovbayes - \gcovboot}{|\gcovboot|}
    \,\,\,
    \textrm{ and }
    \,\,\,
    \frac{\gcovij - \gcovboot}{|\gcovboot|}.
    %
    \end{align*}
    %
    % \spskip
    % For each model:
    % %
    % \begin{itemize}
    %     \item From one MCMC chain, compute $\gcovijhat$, $\gcovbayes$.
    %     \item From 200 bootstraps, compute $\gcovboot$.
    %     \item Compare the relative error to the bootstrap.
    %     %\item Compute Monte Carlo SEs of everything.
    % \end{itemize}
    \end{minipage}
    \hspace{1em}
    \begin{minipage}{0.44\textwidth}
        %\ARMZFig{}
        % \ARMRelFig{}
        \ARMGraphDiff
        \textbf{Total compute time for all models:}
    
        \begin{tabular}{ll}
        Initial fit: & 1.6 hours \\
        Bootstrap: & 381.5 hours\\
        \end{tabular}
\end{minipage}

\end{frame}



\begin{frame}[c]{How to connect to the election data?}

\question{Problem: MCMC is only interesting when the posterior doesn't concentrate.}

\begin{minipage}{0.48\textwidth}
    \ElectionData{}
\end{minipage}
\begin{minipage}{0.48\textwidth}
    \ElectionResultsGlobal{}
\end{minipage}


\end{frame}