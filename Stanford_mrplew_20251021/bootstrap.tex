



%%%%%%%%%%%%%%%%%%%%%%%%%%%%%%%%%%%%%%%%%%%%%%%%%%%%%%%%%%%%%%%%%
%%%%%%%%%%%%%%%%%%%%%%%%%%%%%%%%%%%%%%%%%%%%%%%%%%%%%%%%%%%%%%%%%
%%%%%%%%%%%%%%%%%%%%%%%%%%%%%%%%%%%%%%%%%%%%%%%%%%%%%%%%%%%%%%%%%

\begin{frame}[c]{The weights can look very different!}
\centering
%\only<1>
{
Does this mean anything?  \\
% \textbf{Yes: }We can meaningful sum these weights against regressors.\\[1em]
% What else might it mean?\\
}

\onslide<2->
{
\textbf{Does the spread relate to frequentist variance?}
}

\splitpagenoline{
    \AlexanderWeightPlot{}
}{
    \LaxWeightPlot{}
}
\end{frame}



%%%%%%%%%%%%%%%%%%%%%%%%%%%%%%%%%%%%%%%%%%%%%%%%%%%%%%%%%%%%%%%%%
%%%%%%%%%%%%%%%%%%%%%%%%%%%%%%%%%%%%%%%%%%%%%%%%%%%%%%%%%%%%%%%%%
%%%%%%%%%%%%%%%%%%%%%%%%%%%%%%%%%%%%%%%%%%%%%%%%%%%%%%%%%%%%%%%%%

\begin{frame}[c]{Standard error estimation}


Let $\varemp{\cdot}$ denote the sample variance.

\begin{block}{Calibration weighting standard errors: (sketch)
    \footnote{E.g.~, \textcite{deville:1993:generalizedraking,fuller:2011:sampling}.}
}

Suppose we have $\muhatcw = \surcol{\meansur \w_i \y_i}$ and a
consistent residual estimate $\surcol{\varepsilon_i}$.

Then $\varemp{\w_i \varepsilon_i} \approx \var{}{\sqrt{\Nsur} \muhatcw}$.

\end{block}
\pause

\begin{block}{Standard error consistency theorm: (sketch)}

For Bayesian hierarchical logictic regression, define
$\surcol{\varepsilon_i = \y_i - \expect{\post}{\m(\x_i^\trans \theta)}}$.
% \quad\textrm{and}\quad
% \surcol{\psi_i := \Nsur \w_i^\mrp \varepsilon_i}.

We state mild conditions under which, as $\Nsur \rightarrow \infty$,
for some $\tarcol{\mu_\infty}$ and variance $V$,
$$
\begin{aligned}
    \sqrt{\Nsur}\left(\muhatmrp - \tarcol{\mu_\infty} \right) \rightarrow{}&
    \mathcal{N}\left(0, V\right)
    \quad\textrm{ and }\quad
    \varemp{ \w_i^\mrp \varepsilon_i}
%\surcol{\meansur (\psi_i - \overline{\psi})^2}
    \rightarrow{} V.
\end{aligned}
$$
\end{block}

The use of $\surcol{\w_i^\mrp}$ is exactly analogous to the use of raking weights
for standard error estimation.

This builds on our earlier work on the Bayesian
infinitesimal jackknife.\footcite{giordano:2024:bayesij}

\end{frame}

%%%%%%%%%%%%%%%%%%%%%%%%%%%%%%%%%%%%%%%%%%%%%%%%%%%%%%%%%%%%%%%%%
%%%%%%%%%%%%%%%%%%%%%%%%%%%%%%%%%%%%%%%%%%%%%%%%%%%%%%%%%%%%%%%%%
%%%%%%%%%%%%%%%%%%%%%%%%%%%%%%%%%%%%%%%%%%%%%%%%%%%%%%%%%%%%%%%%%


\begin{frame}[t]{Standard error estimation}
\BootstrapPlot{}
\end{frame}


%%%%%%%%%%%%%%%%%%%%%%%%%%%%%%%%%%%%%%%%%%%%%%%%%%%%%%%%%%%%%%%%%
%%%%%%%%%%%%%%%%%%%%%%%%%%%%%%%%%%%%%%%%%%%%%%%%%%%%%%%%%%%%%%%%%
%%%%%%%%%%%%%%%%%%%%%%%%%%%%%%%%%%%%%%%%%%%%%%%%%%%%%%%%%%%%%%%%%

\begin{frame}[c]{Other uses}
\centering


Does this mean anything?\\
% \textbf{Yes: }We can meaningful sum these weights against regressors.\\[1em]
% What else might it mean?\\
Yes: The ``spread'' relates to frequentist variance just as in calibration weighting.\\[1em]

\onslide<2->
{
\textbf{What about covariate balance?}
}

\splitpagenoline{
    \AlexanderWeightPlot{}
}{
    \LaxWeightPlot{}
}
\end{frame}
