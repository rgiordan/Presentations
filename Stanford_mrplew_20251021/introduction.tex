
\begin{frame}{Are US non-voters becoming more Republican?}

\splitpage{
    \centering
    \textbf{Blue Rose research says yes:}
    \\[1em]
    ``Politically disengaged voters have become much more Republican,
        and because less-engaged voters swung away from [Democrats], an expanded
        electorate meant a more Republican electorate.''
    \\[1em]
    \parencite{blueroseresearch:2024} \\
    (major professional pollsters)
}{
    \textbf{\emph{On Data and Democracy} says no:}
    \\[1em]
    \centering
    ``Claims of a decisive pro-Republican shift among the overall non-voting population are
    not supported by the most reliable, large-scale post-election data currently available.''
    \\[1em]
    \parencite{datademocracyblog:2025} \\
    (major professional researchers)
}
\vspace{1em}
\pause
% Several factors drive the disagreement:
%
\hrule
\begin{itemize}
\item The problem is very hard (it's difficult to accurately poll non--voters)
\item Different data sources
%
% \begin{itemize}
%     \item Blue Rose aggregates its own private data
%     \item The \emph{On Data and Democracy} posts use public data, e.g.~the
%           cooperative election study (CES)
% \end{itemize}
%
\item $\star$$\star$$\star$ \textbf{Different statistical methods}
\begin{itemize}
    \item Blue Rose uses Bayesian hierarchical modeling (MrP)
    \item On Data and Democracy is using calibration weighting (CW)
\end{itemize}
%
\end{itemize}

\pause

\begin{block}{Our contribution}
    We define ``MrP local equivalent weights'' (MrPlew) that:
    %
\begin{itemize}
    \item Are easily computable from MCMC draws and standard software, and
    \item Provide MrP versions of key weighting estimator diagnostics.
\end{itemize}
%
\textbf{$\Rightarrow$ MrPlew provides direct comparisons between
MrP and calibration weighting.}
%
\end{block}

\end{frame}




\begin{frame}{Outline}
%
Weighting (linear) estimators are great --- they come with easy-to-understand diagnostics.\\[1em]

This talk is about making versions of such diagnostics for \textbf{complicated
non-linear models}.\\[2em]

\pause

The key idea is to convert the diagnostic into a \emph{local sensitivity analysis} of this form:

%
\begin{enumerate}
\item Assume your initial model was accurate
\item Select some perturbation your model should be able to capture
\item Use local sensitivity to detect whether the change is what you expect
\item If the change is not what you expect, either (1) or (2) was wrong
\end{enumerate}
%
\pause
\vspace{2em}
I'll do this carefully for covariate balance and MCMC.\\[1em]
But many other variants are possible!
\end{frame}

%%%%%%%%%%%%%%%%%%%%%%%%%%%%%%%%%%%%%%%%
%%%%%%%%%%%%%%%%%%%%%%%%%%%%%%%%%%%%%%%%
%%%%%%%%%%%%%%%%%%%%%%%%%%%%%%%%%%%%%%%%


\begin{frame}{Outline}
%
%
\begin{itemize}
    \item Introduce the statisical problem
    %
    \begin{itemize}
        \item Contrast calibration weighting and MrP
        \item Prior work: Equivalent weights for linear models
        \item Equivalent weights and implicit weights for non--linear models
        \item Our task: Rigorously justify using locally equivalent weights for diagnostics
    \end{itemize} \pause
    %
    \item Locally equivalent weights for frequentist variance estimation \pause
    \item Locally equivalent weights for covariate balance
    \begin{itemize}
        \item Describe classical covariate balance
        \item Introduce a MrPlew ``local empirical consistency check''
        \item Theoretical support
        \item Examples of real-world results
    \end{itemize} \pause
    \item Other directions
    \begin{itemize}
            \item High--level restatement of the logic of our procedure
            \item Local versions of other common diagnostics for linear estimators
            \item Ongoing and future work
    \end{itemize}
\end{itemize}
%
\end{frame}
