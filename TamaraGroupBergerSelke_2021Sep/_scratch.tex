
%%%%%%%%%%%%%%%%%%%%%%%%%%%%%%%%%%%%%%%%%%%%%%%%%%%%%%%%%%%
%%%%%%%%%%%%%%%%%%%%%%%%%%%%%%%%%%%%%%%%%%%%%%%%%%%%%%%%%%%
%%%%%%%%%%%%%%%%%%%%%%%%%%%%%%%%%%%%%%%%%%%%%%%%%%%%%%%%%%%


\begin{frame}{Classical Hypothesis Testing}

% \section{Classical Hypothesis Testing}

Suppose we have data, $X$.  Today, we will assume that the distribution has the
form $p(X | \theta)$ for $\theta \in \mathbb{R}$, where distinct $\theta$ mean
distinct distributions.

Today, our ``null hypothesis'' will be that $\theta = \theta_0$ for some given
$\theta_0$.  Write $H_0$ for the null hypothesis.

A valid ``hypothesis test'' for $H_0$ with ``level'' $\alpha$ is a (possibly
randomized) statistic, $T(X)$, and  region, $A(\alpha)$, such that
%
\begin{align*}
%
p(T(X) \in A(\alpha) | H_0) \le \alpha.
%
\end{align*}

If we observe $T(x) \in A(\alpha)$, we ``reject'' $H_0$, since we have observed
something unlikely under $H_0$.

Typically the rejection regions are nested, so that $\alpha < \alpha'
\Rightarrow A(\alpha) \subset A(\alpha')$.  When this is the case, the $p-$value
is the smallest level $\alpha$ at which we would reject the observed $x$.

\end{frame}


%%%%%%%%%%%%%%%%%%%%%%%%%%%%%%%%%%%%%%%%%%%%%%%%%%%%%%%%%%%
%%%%%%%%%%%%%%%%%%%%%%%%%%%%%%%%%%%%%%%%%%%%%%%%%%%%%%%%%%%
%%%%%%%%%%%%%%%%%%%%%%%%%%%%%%%%%%%%%%%%%%%%%%%%%%%%%%%%%%%



\begin{frame}{Running example}
%
The paper's running example is
%
\begin{align*}
%
X ={}& (X_1, ..., X_i, ..., X_n) \textrm{ are IID}\\
X_i | \theta \sim{}& \mathcal{N}(\theta, \sigma^2)\textrm{, where }\sigma\textrm{ is known}.
%
\end{align*}
%


The paper's running example is the familiar test using the sample mean
$\bar{X}$:
%
\begin{align*}
%
T(X) ={}& \sqrt{n} \frac{|\bar{X} - \theta_0|}{\sigma}
\quad\textrm{and}\quad
A = \left\{x: T(x) > \Phi^{-1}\left(1 - \frac{\alpha}{2}\right) \right\}.
%
\end{align*}

\end{frame}


\begin{frame}{What is a p-value?}

\section{P-values}


\textbf{Example 2.}
%
\begin{align*}
%
T(X) \sim{}& \mathrm{Uniform}[0,1]
\quad\textrm{and}\quad
A = \{x: T(x) < \alpha \}.
%
\end{align*}

\end{frame}
