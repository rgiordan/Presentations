
Some exercises:

\begin{itemize}
%
    \item What happens if you use \v{print} instead of \v{cat} in the
    \v{r\_example2} chunk of \secref{experiement_one}?

    \item What happens if refer to the undefined variable \v{y} instead of \v{x}
    in the \v{r\_example2} chunk of \secref{experiement_one}?  How does the
    error appear?

    \item What do you seen when you set \v{knitr\_debug <- TRUE} in the
    \v{setup} chunk of \secref{experiement_one}?

    \item Suppose nothing works and you think the \v{data\_path} variable is
    messed up.  How can you print the value of \v{data\_path} in the LaTeX pdf
    to debug it?

    \item Add a new TeX macro so you can refer to the standard deviation of $x$
    in the text.  (Edit the \v{define\_macros.R} file.)

    \item What do you seen when you set \v{knitr\_cache <- TRUE} in the
    \v{setup} chunk of \secref{experiement_one}?  (Try knitting before and after
    removing the \v{figure} directory.)

    \item Add a column for $\epsilon$ and a row for the median to
    \tabref{amazing_table}.

    \item Add vertical lines to \tabref{amazing_table}.

    \item Replace the text ``Epsilon'' with the Greek character in the title
    of the second ggplot of \figref{eone_hist}.  (Hint: look at the
    x-axis of \figref{eone_scatter}.)

    \item Change the printed height and width of the figures by passing
    arguments to \v{SetImageSize()}.

    \item Make the figure lines thicker and text larger by changing the
    \v{base\_figure\_width} variable in \v{initialize.R}.

    \item Make the figure lines thicker and text larger by changing the
    \v{base\_figure\_width} variable in \v{initialize.R}.

    \item Make two plots side by side that share a common legend using the
    \v{GetLegend()} command.  The layout should be a horizontal row containing
    first figure 1, then figure 2, then the shared legend.

    \item  \Figref{eone_hist} uses \v{GridExtra::grid\_arrange} to make
    side-by-side images.  Can you do the same thing with float environments and
    two separate images?

\end{itemize}
