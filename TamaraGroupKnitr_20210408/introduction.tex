Knitr is a framework for inserting dynamic R code into documents.  I'll focus on
using Knitr for LaTeX, but it can be used for other systems too (e.g. markdown).
This is not intended to be an introduction to knitr!  This is a working example
of tools and an organizational framework I've developed over the years with help
and advice from all my co-authors for interfacing R code and LaTeX.  Even if you
don't know knitr, you could probably start to use it by copying this framework,
and learning what you need as you go along.

What is knitr?  In short, knitr converts an \v{Rnw} file into a \v{tex} file.
This is done via an R command.  Here, for convenience, I wrapped the command in
the shell script \v{knit\_to\_tex.sh}.

What is in a \v{Rnw} file?  It's basically ordinary TeX, which is essentially
unchanged during the knitting process, interleaved with R code chunks, which are
delimited with special characters (beginning with \v{<<...>>=} and ending with
\v{@}).  When you knit a document, all the code chunks are executed in order. If
you ask it to, the output is printed into the TeX document, either as R code,
verbatim, or as graphs.  R code can also run silently to prepare for future code
chunks, and the arguments to code chunks can refer to the R environment.  At
this level, knitr is an exteremely flexible way to generate LaTeX dynamically.
For example, compare the knitr file \v{experiment\_two.Rnw} (which is what you
would write) with the automatically generated \v{experiment\_two.tex}.

If you read knitr tutorials, you might get the idea that you're supposed to
write your document as one big knitr file and run your analysis inside your
document.  Maybe that works for some people, but it can make compiling your
document really slow, and can be extremely annoying when you're working on some
part of your document that doesn't really need knitr's functionality. For that
reason, I use a setup like the one here.  Most of the document is simply in
\v{tex} (see \v{main.tex}, which would normally contain a lot of theory, all in
\v{tex}), but I \v{input} the knitted output as \v{tex} files (in this case,
 \v{experiment\_one.tex}. and  \v{experiment\_two.tex}).  Every time you make
 changes to the experiment files \v{experiment\_*.Rnw} you need to re-knit,
 of course, but you can ignore knitr when editing the rest of the document.
 In order to make this work, you need to include (once) some knitr boilerplate,
 which is contained in \v{\_knitr\_header.tex}.

I try to do as little actual analysis as possible in the document itself, again
to speed up rendering the paper.  Remember that every time you change the text
of your \v{Rnw} file you need to re-knit; if you're solving an optimization
problem or even reshaping big dataframes each time you compile your pdf, it will
be annoying.  For that reason I usually write a preprocessing script that saves
all the data I need in a nice format (here,
\v{R\_scripts/data/preprocess\_data.R}).  Then I load that data into knitr and
basically just use R to make graphs and define macros (see \v{load\_data.R}
and \v{define\_macros.R} scripts).

Going even further, I usually put as little actual R code in the \v{Rnw}
document itself, because debugging in knitr is a pain.  Instead, I source \v{.R}
scripts that generate the output I need.  Typically I develop these scripts
interactively by sourcing them in Rstudio until they look like what I want, and
then just use the same code in knitr.  Both \secref{experiement_one,
experiement_two} here have multiple examples of this.

Finally, there's a ton of fiddly stuff, especially around labeling, floats,
and image sizes.  Fortunately you only need to figure this out once, and then
you can just keep using what you know works.  This document contains working
examples of the some basic things that you can play with.  Feel free to
explore and change things on your own or, for some guidance, you can try
the exercises in \secref{exercises}.
