Some exercises:

\begin{itemize}
    \item What happens if you use \v{print} instead of \v{cat}
        in the \v{r\_example2} chunk of \secref{experiement_one}?

    \item What happens if refer to the undefined variable \v{y} instead of \v{x}
        in the \v{r\_example2} chunk of \secref{experiement_one}?  How
        does the error appear?

    \item What do you seen when you set \v{knitr\_debug <- TRUE} in the
        \v{setup} chunk of \secref{experiement_one}?

    \item Suppose nothing works and you think the \v{data\_path} variable
        is messed up.  How can you print its value in the LaTeX pdf?

    \item Add a new TeX macro so you can refer to the standard deviation of $x$
        in the text.  (Edit the \v{define_macros.R} file.)

    \item What do you seen when you set \v{knitr\_cache <- TRUE} in the
        \v{setup} chunk of \secref{experiement_one}?  (Try knit-ing before
        and after removing the \v{figure} directory.)

    \item  \Figref{eone_hist} uses \v{GridExtra::grid\_arrange} to make
        side-by-side images.  Can you do the same thing with float environments
        and two separate images?

\end{itemize}

The iteresting stuff in is in \secref{experiement_one, experiement_two}, which
are generated by knitr.
