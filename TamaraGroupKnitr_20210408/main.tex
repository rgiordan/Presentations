\documentclass[11pt]{article}

% my added packages: copied from the swiss IJ paper
\usepackage{microtype}
\usepackage{graphicx}
\usepackage{subfigure}
\usepackage{booktabs} % for professional tables
\usepackage{siunitx}
\usepackage{hyperref}
\usepackage{xargs}[2008/03/08]

% Documentation
% http://ftp.math.purdue.edu/mirrors/ctan.org/macros/latex/contrib/refstyle/refstyle.pdf
\usepackage{refstyle}
\usepackage{varioref} % Use refstyle instead of varioref directly.
\usepackage[authoryear]{natbib}

\usepackage{amsmath}
\usepackage{amssymb}
\usepackage{amsfonts}
\usepackage{amsthm}
\usepackage{mathrsfs}
\usepackage{mathtools}
\usepackage{algpseudocode, algorithm} %typical alg typesetting packages

\usepackage{listings}
\usepackage{pdfpages}

% This picks up the knitr boilerplate, allowing us to \input partial knitr
% documents.
\input{_knitr_header.tex}

% This defines math macros.
% Math definitions

% Some definitions for the text

\def\loset{MIS}
\def\loeff{MIP}
\def\loalpha{PIP}
\def\aloset{AMIS}
\def\aloeff{AMIP}
\def\aloalpha{APIP}
\def\na{\texttt{NA}}

%%%%%%%%%%%%%%%%%%%%%%%%
% Rachael's math defs

\DeclareMathOperator*{\argmax}{arg\,max}
\DeclareMathOperator*{\argmin}{arg\,min}

\DeclarePairedDelimiter\abs{\lvert}{\rvert}
\DeclarePairedDelimiter\norm{\lVert}{\rVert}

% Swap the definition of \abs* and \norm*, so that \abs
% and \norm resizes the size of the brackets, and the
% starred version does not.
\makeatletter
\let\oldabs\abs
\def\abs{\@ifstar{\oldabs}{\oldabs*}}

%\let\oldnorm\norm
\def\norm{\@ifstar{\oldnorm}{\oldnorm*}}

%%%%%%%%%%%%%%%%%%%%%%%%
% Ryan's math defs

\def\expect#1#2{\mathbb{E}_{#1}\left[#2\right]}%

\def\thetadom{\Omega_\theta}
\def\fhat{\hat{F}}
\def\flim{F_{\infty}}
\def\inflbar{\tilde{\infl}}
\def\thetalim{\theta_{\infty}}
\def\alphan{\lfloor \alpha N \rfloor}

\global\long\def\cop{C_{op}}%
\global\long\def\cmoment{H_2}%
\global\long\def\cxxlo{\xi_1}%
\global\long\def\cxepslo{\xi_2}%
\global\long\def\cball{\mathscr{B}}%
\global\long\def\cijdiff{\Delta_{lin}}%
\global\long\def\copscaled{\tilde{C}_{op}}%
\global\long\def\cgh{C_{gh}}%
\global\long\def\lh{L_{h}}%
\global\long\def\cij{C_{IJ}}%


% \norm was taken, so use \vnorm for "vector norm".
\global\long\def\info{\mathcal{I}}%

% These are all used for the finite sample bounds section.
\global\long\def\betalin{\hat\beta^{\mathrm{lin}}}%
\global\long\def\nset{\mathcal{S}}%


\global\long\def\cop{C_{op}}%
\global\long\def\cmoment{H_2}%
\global\long\def\cxxlo{\xi_1}%
\global\long\def\cxepslo{\xi_2}%
\global\long\def\cball{\mathscr{B}}%
\global\long\def\cijdiff{\Delta_{lin}}%

\global\long\def\copscaled{\tilde{C}_{op}}%

% calculus
\newcommand{\dee}{\mathrm{d}}

% Influence score and related quantities
\def\infl{\psi}
\def\inflscale{\hat{\sigma}_{\infl}}
\def\inflscalelim{\sigma_{\infl}}
\def\ind#1{\mathbb{I}\left(#1\right)}
\def\mis#1{S_{#1}}
\def\mip#1{\Psi_{#1}}
\def\amis#1{\hat{S}_{#1}}
\def\amip#1{\hat{\Psi}_{#1}}
\def\loprop#1{\alpha^*_{#1}}
\def\aloprop#1{\hat{\alpha}^*_{#1}}
\def\thetafun{\phi}
\def\thetafunlin{\thetafun^{\mathrm{lin}}}%
\def\thetalin{\thetahat^{\mathrm{lin}}}%

\def\shape{\hat{\mathscr{T}}_\alpha}
\def\noise{\hat\sigma_{\phi}}
\def\plim{\overset{p}{\rightarrow}}
\def\x{x}

% Vectors
%\def\x{\vec{x}}
\def\d{d} % A general data point
\def\dvec{\vec{d}} % A general data point
\def\p{p} % The index into the parameter vector
\def\P{P} % The length of the parameter vector
\def\w{\vec{w}}
\def\wnorm{\vec{\omega}} % Used only in the \thetafun lemma

\def\zP{0_{\P}}
\def\zPN{0_{\P \times N}}
\def\thetahat{\hat{\theta}}
\def\thetavec{\vec{\theta}}
\def\onevec{\vec{1}}
\def\inflvec{\vec{\infl}}

% Operators
\def\sumn{\sum_{n=1}^N}
\def\meann{\frac{1}{N}\sum_{n=1}^N}
\def\var#1#2{\underset{#1}{\mathrm{Var}}\left(#2\right)}
\def\vnorm#1{\left\Vert #1\right\Vert }%
\def\falphanorm#1#2{\left\Vert #2\right\Vert_{#1, \alpha} }%
\def\at#1#2{\left.#1\right|_{#2}}%
\def\fracat#1#2#3{\at{\frac{#1}{#2}}{#3}}%
\def\mbe{\mathbb{E}}%
\def\argmaxover#1{\underset{#1}{\mathrm{argmax}}\,}
\def\maxover#1{\underset{#1}{\mathrm{max}}\,}


% This specifies the formatting for references (sections, theorems, etc.)
%%%%%%%%%%%%%%%%%%%%
% amsthm commands

\theoremstyle{plain}
\newtheorem{lem}{Lemma}
\newtheorem{thm}{Theorem}
\newtheorem{prop}{Proposition}
\newtheorem{cond}{Condition}
\newtheorem{assu}{Assumption}
\newtheorem{cor}{Corollary}
\newtheorem{conj}{Conjecture}

\theoremstyle{definition}
\newtheorem{defn}{Definition}
%\newtheorem{ex}{Example}

% Example environment with a terminating symbol.
% https://tex.stackexchange.com/questions/16453/denoting-the-end-of-example-remark
\theoremstyle{definition}
\newtheorem{examplex}{Example}
\newenvironment{ex}
  {\pushQED{\qed}\renewcommand{\qedsymbol}{$\triangle$}\examplex}
  {\popQED\endexamplex}


%%%%%%%%%%%%%%%%%%%%
% refstyle commands

% This needs to be fig to work with knitr
\newref{fig}{
    name=Figure~, %
    Name=Figure~
    }

% This needs to be fig to work with knitr
\newref{tab}{
    name=Table~, %
    Name=Table~
    }

\newref{sec}{
    name=Section~, %
    Name=Section~,
    names=Sections~, %
    Names=Sections~
    }

\newref{app}{
    name=Appendix~, %
    Name=Appendix~
    }

\newref{eq}{
    name=Eq.~, %
    Name=Eq.~,
    names=Eqs.~, %
    Names=Eqs.~
    }

\newref{fig}{
    name=Figure~, %
    Name=Figure~
    }

\newref{def}{
    name=Definition~, %
    Name=Definition~
    }

\newref{assu}{
    name=Assumption~, %
    Name=Assumption~,
    names=Assumptions~, %
    Names=Assumptions~,
    }

\newref{cond}{
    name=Condition~, %
    Name=Condition~,
    names=Conditions~, %
    Names=Conditions~
    }

\newref{prop}{
    name=Proposition~, %
    Name=Proposition~,
    names=Propositions~, %
    Names=Propositions~
    }

\newref{lem}{
    name=Lemma~, %
    Name=Lemma~,
    names=Lemmas~, %
    Names=Lemmas~
    }

\newref{ex}{
    name=Example~, %
    Name=Example~}

\newref{cor}{
    name=Corollary~, %
    Name=Corollary~
    }

\newref{thm}{
    name=Theorem~, %
    Name=Theorem~,
    names=Theorems~, %
    Names=Theorems~
    }

\newref{proof}{
    name=Proof~, %
    Name=Proof~
    }

\newref{conj}{
    name=Conjecture~, %
    Name=Conjecture~
    }

\newref{algr}{
    name=Algorithm~, %
    Name=Algorithm~,
    names=Algorithms~, %
    Names=Algorithms~,
    }


%% refstyle examples:
% \Secref[vref]{introduction} contains \secref{introduction}.
% \Secref[vref]{ack} does not contain \secref{introduction}.
%
% \begin{align}
%     x=y \eqlabel{myeq}
% \end{align}
%


\begin{document}

\title{Maybe You Should Use Knitr}

\author{Ryan Giordano \texttt{rgiordan@mit.edu}}

\maketitle

\begin{abstract}
This is an abstract.
\end{abstract}

\section{Introduction}\seclabel{introduction}
\begin{frame}{Dropping data: Motivation}

More data \& cheaper computation $\Rightarrow$\\
Statistical analyses are
playing larger roles in decision making.

\vspace{1em}
Decisions are important: We want \textbf{trustworthy} conclusions.\\
Data / models not always perfect: We want \textbf{robust} conclusions.

% \begin{itemize}
%     \item Gather + clean exchangeable data,
%     \item Specify and fit a model, and
%     \item Drawn a qualitative conclusion from your fit
%     \\(e.g., based on the sign / significance of
%         some estimated parameter).
% \end{itemize}

\vspace{1em}
Would you be concerned if you could \textbf{reverse your conclusion} by removing
a \textbf{small proportion} (say, $0.1\%$) of your data?

\vspace{1em}
\textbf{Running example:} \citet{angelucci2015microcredit}, a
randomized controlled trial study of the efficacy of microcredit based
on 16,560 data points.  \\We can reverse the studies qualitative
conclusions by removing $15$ observations ($< 0.1 \%$ of the data).

\vspace{1em}
\textbf{How do we find sets of influential points?}  Difficult in general!\\

\vspace{1em}
We provide a \textbf{automatic approximation} with finite-sample
guarantees.

\vspace{1em}
Studying the approximation reveals the causes of non-robustness.

\end{frame}

%%%%%%%%%%%%%%%%%%%%%%%%%%%%%%%%%%%%%%%%%%%%%%%%%%%%%%%%%%%%%%%%%%%%%%%%
%%%%%%%%%%%%%%%%%%%%%%%%%%%%%%%%%%%%%%%%%%%%%%%%%%%%%%%%%%%%%%%%%%%%%%%%
%%%%%%%%%%%%%%%%%%%%%%%%%%%%%%%%%%%%%%%%%%%%%%%%%%%%%%%%%%%%%%%%%%%%%%%%

\begin{frame}{Dropping data: Mexico Microcredit}

Consider \citet{angelucci2015microcredit}, a randomized controlled trial study
of the efficacy of microcredit in Mexico based on 16,560 data points.

The variable ``Beta" estimates the effect of microcredit in US dollars.

%\MicrocreditMexicoRerunTable{}

\begin{table}[ht]
\centering
\begin{tabular}{lll} \hline
  & \onslide<2->{Left out points} & Beta (SE) \\\hline
\hspace{0.05em} Original result & \onslide<2->{0} & -4.55 (5.88) \\ \hline
\onslide<2-> {\hspace{0.05em} Change sign & 1 & 0.4 (3.19) \\\hline }
\onslide<3-> {Change significance & 14 & -10.96 (5.57) \\\hline }
\onslide<4-> {Change sign and significance & 15 & 7.03 (2.55) \\\hline }
\end{tabular}
\end{table}

\vspace{-1em}
\onslide<1-> { \textbf{Original conclusion: }\\
There is no evidence that microcredit is effective.}

\vspace{1em}
\onslide<5-> { \textbf{Potential conclusions after data dropping: }\\
The effect of microcredit is positive (negative) \&
statistically significant.}

% \vspace{1em}
% \onslide<6-> { By removing very few data points ($15 / 16560 \approx 0.1\% $),
% we can reverse the qualitative conclusions of the original study! }

\vspace{1em}
\onslide<6-> { \textbf{The culprit is signal to noise ratio. }\\
By the end of the talk, we will see that the sensitivity is due to
\begin{itemize}
\item High variability of the outcome (hosehold profit) relative to
\item A small signal driving the conclusion
(statistical significance)
\end{itemize}
}



\end{frame}

% %%%%%%%%%%%%%%%%%%%%%%%%%%%%%%%%%%%%%%%%%%%%%%%%%%%%%%%%%%%%%%%%%%%%%%%%
% %%%%%%%%%%%%%%%%%%%%%%%%%%%%%%%%%%%%%%%%%%%%%%%%%%%%%%%%%%%%%%%%%%%%%%%%
% %%%%%%%%%%%%%%%%%%%%%%%%%%%%%%%%%%%%%%%%%%%%%%%%%%%%%%%%%%%%%%%%%%%%%%%%

\begin{frame}{Dropping data: Motivation}

Would you be concerned if you could \textbf{reverse your conclusion} by removing
a \textbf{small proportion} (say, $0.1\%$) of your data?

\pause
Not always!  But sometimes, surely yes.

Thinking without random noise can be helpful.

Suppose you have a farm, and want to know whether
your average yield is greater than 170 bushels per acre.
At harvest, you measure 200 bushels per acre.

\begin{itemize}
    \item Scenario one: If your yield is greater than 170 bushels
        per acre, you make a profit.
        \begin{itemize}
            \item Don’t care about sensitivity to small subsets
        \end{itemize}
    \item Scenario two: You want to recommend your farming
    methods to a friend across the valley.
    \begin{itemize}
        \item Might care about sensitivity to small subsets
    \end{itemize}
\end{itemize}

For example, often in economics:
%
\begin{itemize}
\item Small fractions of data are missing not-at-random,
\item Policy population is different from analyzed population,
\item We report a convenient summary (e.g. mean) of a complex effect,
\item Models are stylized proxies of reality.
\end{itemize}


\end{frame}


\section{Experiment One}\seclabel{experiement_one}
% experiment_one.tex is generated by knitr
%%%%%%%%%%%%%%%%%%%%%%%%%%%%%%%%%%%%%%
%%%%%%%%%%%%%%%%%%%%%%%%%%%%%%%%%%%%%%
% Do not edit the TeX file your work
% will be overwritten.  Edit the Rnw
% file instead.
%%%%%%%%%%%%%%%%%%%%%%%%%%%%%%%%%%%%%%
%%%%%%%%%%%%%%%%%%%%%%%%%%%%%%%%%%%%%%





\newcommand{\EoneNumObs}{1,000}


In between the knitr chunks, this is just an ordinary LaTeX document. The
content inside the code chunks gets run in R.  By default, the code runs
silently.  If you add the option \texttt{results="asis"} then the output gets
inserted verbatim into the tex document.  This can be used to make tables or
define macros.



1 + 10 = 11

$x$ = 6.000000

Figure insertion uses a special set of semantics --- see below
for examples.


I use the \v{define\_macros.R} script to specify macros defined from the
\v{Rdata} file.  Examples follow. For this experiment, we generated
$\EoneNumObs$ observations.  They looked like a mess, as you can see in
\figref{eone_scatter}.

% In theory you can set a figure caption using R code in the fig.cap
% argument, but I find it's awkward, especially with line wrapping.
% So instead I just store the caption in a variable in a chunk just
% before the image.  Conveniently, that's a good place to set the size
% of the subsequent image as well.  Note that the SetImageSize() sets
% chunk defaults, and so must be run in a chunk /before/ the chunk with
% the image whose size you want to set.

\begin{knitrout}
\definecolor{shadecolor}{rgb}{0.969, 0.969, 0.969}\color{fgcolor}\begin{figure}[!h]

{\centering \includegraphics[width=0.980\linewidth,height=0.784\linewidth]{figure/eone_scatter-1} 

}

\caption[It's nice to have a long figure caption that allows easy access to latex stuff like there were $\EoneNumObs$ draws of $x$ and $\epsilon$ that went into this plot]{It's nice to have a long figure caption that allows easy access to latex stuff like there were $\EoneNumObs$ draws of $x$ and $\epsilon$ that went into this plot.}\label{fig:eone_scatter}
\end{figure}


\end{knitrout}

% Knitr automatically labels figures with fig:chunk_name.
And \figref{eone_hist} as well.  What garbage.

% Note that chunk names cannot be repeated.

\begin{knitrout}
\definecolor{shadecolor}{rgb}{0.969, 0.969, 0.969}\color{fgcolor}\begin{figure}[!h]

{\centering \includegraphics[width=0.980\linewidth,height=0.392\linewidth]{figure/eone_hist-1} 

}

\caption[You can reuse this variable for other captions]{You can reuse this variable for other captions.}\label{fig:eone_hist}
\end{figure}


\end{knitrout}
%


\section{Experiment Two}\seclabel{experiement_two}
% experiment_two.tex is generated by knitr
%%%%%%%%%%%%%%%%%%%%%%%%%%%%%%%%%%%%%%
%%%%%%%%%%%%%%%%%%%%%%%%%%%%%%%%%%%%%%
% Do not edit the TeX file your work
% will be overwritten.  Edit the Rnw
% file instead.
%%%%%%%%%%%%%%%%%%%%%%%%%%%%%%%%%%%%%%
%%%%%%%%%%%%%%%%%%%%%%%%%%%%%%%%%%%%%%





\newcommand{\ETwoNumObs}{1,000}
\newcommand{\ETwoBeta}{5}


We generated $\ETwoNumObs$ observations again, but now we used an offset $\beta =
\ETwoBeta$.  They looked better, but we already showed you how to make a graph,
so instead we'll show it in \tabref{amazing_table}.


\begin{table}[h]
\begin{center}% latex table generated in R 3.6.3 by xtable 1.8-4 package
% Fri Apr  9 19:40:17 2021
\begin{tabular}{lrr}
  \hline
metric & x & y \\ 
  \hline
mean & -0.0751 & -0.3913 \\ 
  sd & 1.0112 & 5.2012 \\ 
  max & 2.9757 & 16.0883 \\ 
   \hline
\end{tabular}
\end{center}
\caption{Amazing summary stats\tablabel{amazing_table}}
\end{table}


\bibliography{references}
\end{document}
