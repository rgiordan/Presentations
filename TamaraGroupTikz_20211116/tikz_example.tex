\documentclass[8pt]{beamer}\usepackage[]{graphicx}\usepackage[]{color}

\def\expect#1#2{\underset{#1}{\mathbb{E}}\left[#2\right]}
\def\sumn{\sum_{n=1}^N}
\def\x{x}
\def\xvec{X}
\def\w{w}
\def\onevec{1_N}
\def\infl{\psi}

\usetheme{metropolis}           % Use metropolis theme
\usepackage{amsmath}
\usepackage{tikz}


\usepackage{listings}
\lstset
{
    language=[LaTeX]TeX,
    breaklines=true,
    basicstyle=\tt\scriptsize,
    keywordstyle=\color{blue},
    identifierstyle=\color{magenta},
}

\title{TikZ and Beamer for Image Annotation}
\author{Ryan Giordano}
\date{Nov 16th, 2021}
\institute{Massachusetts Institute of Technology}

\begin{document}



%%%%%%%%%%%%%%%%%%%%%%%%%%%%%%%%%%%%%%%%%%%%%%%%%%%%%%%%%%%%%%%%%%%%%%%
%%%%%%%%%%%%%%%%%%%%%%%%%%%%%%%%%%%%%%%%%%%%%%%%%%%%%%%%%%%%%%%%%%%%%%%
%%%%%%%%%%%%%%%%%%%%%%%%%%%%%%%%%%%%%%%%%%%%%%%%%%%%%%%%%%%%%%%%%%%%%%%

\begin{frame}[fragile]{How does the IJ work?  Data re-weighting.}

\begin{center}
\begin{minipage}{0.38\textwidth}
\includegraphics[width=\textwidth]{e_beta_w}
\end{minipage}
\end{center}

%%%%%%%%%%%%%%%%%%%%%%
\hrulefill

Let's annotate this graphic using TikZ.

\begin{lstlisting}
\begin{center}
\begin{minipage}{0.38\textwidth}
\includegraphics[width=\textwidth]{e_beta_w}
\end{minipage}
\end{center}
\end{lstlisting}

From now on, everything I'm going to do will be within the minipage.

\end{frame}





%%%%%%%%%%%%%%%%%%%%%%%%%%%%%%%%%%%%%%%%%%%%%%%%%%%%%%%%%%%%%%%%%%%%%%%
%%%%%%%%%%%%%%%%%%%%%%%%%%%%%%%%%%%%%%%%%%%%%%%%%%%%%%%%%%%%%%%%%%%%%%%
%%%%%%%%%%%%%%%%%%%%%%%%%%%%%%%%%%%%%%%%%%%%%%%%%%%%%%%%%%%%%%%%%%%%%%%

\begin{frame}[fragile]{How does the IJ work?  Data re-weighting.}

\begin{center}
\begin{minipage}{0.38\textwidth}
    \begin{tikzpicture}
        \node[anchor=south west,inner sep=0] (image) at (0,0) {
            \includegraphics[width=\textwidth]{e_beta_w}
        };
        \begin{scope}[x={(image.south east)},y={(image.north west)}]
            \draw[color=red] (0, 0) circle (0.1);
            \draw[color=green] (1.0, 1.0) circle (0.1);
            \draw[color=blue] (0.1, 0.5) circle (0.1);
            \node[color=red] (hello) at (0.5, 0.5) {HELLO};
        \end{scope}
    \end{tikzpicture}
\end{minipage}
\end{center}

%%%%%%%%%%%%%%%%%%%%%%
\hrulefill

\begin{lstlisting}
\begin{tikzpicture}
    \node[anchor=south west,inner sep=0] (image) at (0,0) {
        \includegraphics[width=\textwidth]{e_beta_w}
    };
    \begin{scope}[x={(image.south east)},y={(image.north west)}]
        \draw[color=red] (0, 0) circle (0.1);
        \draw[color=green] (1.0, 1.0) circle (0.1);
        \draw[color=blue] (0.1, 0.5) circle (0.1);
        \node[color=red] (hello) at (0.5, 0.5) {HELLO};
    \end{scope}
\end{tikzpicture}
\end{lstlisting}

\end{frame}




%
%
% %%%%%%%%%%%%%%%%%%%%%%%%%%%%%%%%%%%%%%%%%%%%%%%%%%%%%%%%%%%%%%%%%%%%%%%
% %%%%%%%%%%%%%%%%%%%%%%%%%%%%%%%%%%%%%%%%%%%%%%%%%%%%%%%%%%%%%%%%%%%%%%%
% %%%%%%%%%%%%%%%%%%%%%%%%%%%%%%%%%%%%%%%%%%%%%%%%%%%%%%%%%%%%%%%%%%%%%%%
%
% \begin{frame}{How does the IJ work?  Data re-weighting.}
%
% Augment the problem with {\em data weights} $\w_1, \ldots, \w_N$.
% %
% We can write $\expect{p(\theta \vert \xvec, \w)}{\theta}$.
%
% \begin{minipage}{0.49\textwidth}
%     \hspace{\textwidth}
% \end{minipage}
% \begin{minipage}{0.49\textwidth}
%     % https://www.overleaf.com/learn/latex/TikZ_package
%     % https://latexdraw.com/how-to-annotate-an-image-in-latex/
%     % https://tex.stackexchange.com/questions/9559/drawing-on-an-image-with-tikz
%     \begin{tikzpicture}
%         \onslide<6-> {
%         \node[anchor=south west,inner sep=0] (image) at (0,0) {
%             \includegraphics[width=0.98\textwidth]{e_beta_w}
%         };
%         }
%         \onslide<7->{
%         \begin{scope}[x={(image.south east)},y={(image.north west)}]
%             \draw[blue, thick, <-] (0.2,0.23) -- ++(0.1,0.25)
%                     node[above,black,fill=white]
%                     {\small $\expect{p(\theta \vert \x)}{\theta}$};
%         \end{scope}
%         }
%         \onslide<6->{
%         \begin{scope}[x={(image.south east)},y={(image.north west)}]
%             \draw[blue, thick, <-] (0.8,0.8) -- ++(-0.1,0.1)
%                     node[left,black,fill=white]
%                     {\small $\expect{p(\theta \vert \x, \w_n)}{\theta}$};
%         \end{scope}
%         }
%         \onslide<9->{
%         \begin{scope}[x={(image.south east)},y={(image.north west)}]
%             \draw[red, thick, -] (0.18,0.18) -- ++(1.2 * 0.6, 1.2 * 0.48);
%         \end{scope}
%         }
%         \onslide<10->{
%         \begin{scope}[x={(image.south east)},y={(image.north west)}]
%             \draw[blue, thick, <-] (0.8,0.65) -- ++(0.02,-0.1)
%                     node[below,black,fill=white]
%                     {\small Slope $ = \infl_n$};
%         \end{scope}
%         }
%     \end{tikzpicture}
% \end{minipage}
%
%
% \end{frame}
%
%
%

\begin{frame}{Source and further reading}

Beamer:
{\tiny
% \url{https://www.overleaf.com/learn/latex/Beamer_Presentations%3A_A_Tutorial_for_Beginners_(Part_1)%E2%80%94Getting_Started}\\
\begin{itemize}
\item Google ``beamer tutorial''
\item \url{https://warwick.ac.uk/fac/sci/physics/research/cfsa/people/pastmembers/wuensch/workshoplatex/beamertutorialkwuensch.pdf}
\item \url{https://www.texdev.net/2014/01/17/the-beamer-slide-overlay-concept/}\\
\end{itemize}
}

TikZ:
{\tiny
\begin{itemize}
\item \url{https://www.overleaf.com/learn/latex/TikZ_package}
\item \url{https://www.math.uni-leipzig.de/~hellmund/LaTeX/pgf-tut.pdf}
\item \url{https://latexdraw.com/how-to-annotate-an-image-in-latex/}
\item \url{https://tex.stackexchange.com/questions/9559/drawing-on-an-image-with-tikz}
\end{itemize}
}

\end{frame}


\end{document}
