\documentclass[8pt]{beamer}\usepackage[]{graphicx}\usepackage[]{color}


\usetheme{metropolis}           % Use metropolis theme
\usepackage{amsmath}
\usepackage{mathrsfs}
\usepackage{tabularx}

% For custom oversets
\usepackage{accents}


\usepackage[style=authoryear]{biblatex}
\addbibresource{references.bib}
\usepackage{cleveref}
\renewcommand*{\bibfont}{\footnotesize}



\def\p#1{\mathbb{P}\left(#1\right)}
\def\q#1{\mathbb{Q}\left(#1\right)}
\def\y{y}
\def\z{z}
\def\normz{\mathcal{N}(z)}
\def\etahat{\hat{\eta}}
\def\ellhat{\hat{\ell}}
\def\sumn{\sum_{n=1}^N}
\def\meann{\frac{1}{N} \sumn}
\newcommand{\etastar}{\accentset{*}{\eta}}
\def\Z{\mathcal{Z}}
\def\expect#1#2{\mathbb{E}_{#1}\left[#2\right]}
\def\kl#1{\mathrm{KL}\left(#1\right)}
\def\klhat#1{\widehat{\mathrm{KL}}\left(#1\right)}
\def\ind#1{1\left(#1\right)}

\DeclareMathOperator*{\argmax}{\mathrm{argmax}}
\DeclareMathOperator*{\argmin}{\mathrm{argmin}}
\DeclareMathOperator*{\esssup}{\mathrm{esssup}}
\DeclareMathOperator*{\essinf}{\mathrm{essinf}}
\DeclareMathOperator*{\argsup}{\mathrm{argsup}}
\DeclareMathOperator*{\arginf}{\mathrm{arginf}}

\title{Inductive logic: Introduction and context}
\author{Ryan Giordano}
\date{Sep 23rd, 2022}
\institute{Massachusetts Institute of Technology}

\begin{document}




%%%%%%%%%%%%%%%%%%%%%%%%%%%%%%%%%%%%%%%%%%%%%%%%%%%%%%%%%%%%%%%%%%%%%%%
%%%%%%%%%%%%%%%%%%%%%%%%%%%%%%%%%%%%%%%%%%%%%%%%%%%%%%%%%%%%%%%%%%%%%%%
%%%%%%%%%%%%%%%%%%%%%%%%%%%%%%%%%%%%%%%%%%%%%%%%%%%%%%%%%%%%%%%%%%%%%%%

\begin{frame}{Outline}
%
\begin{itemize}
%
\item Logic
\item Deduction and induction
\item Classes of inductive questions (from philosophical induction to probability)
\item Extreme resolutions: ``Bayesian,'' ``falsificationist,'' ``conventionalist''
%
\end{itemize}
%
\end{frame}

%%%%%%%%%%%%%%%%%%%%%%%%%%%%%%%%%%%%%%%%%%%%%%%%%%%%%%%%%%%%%%%%%%%%%%%
%%%%%%%%%%%%%%%%%%%%%%%%%%%%%%%%%%%%%%%%%%%%%%%%%%%%%%%%%%%%%%%%%%%%%%%
%%%%%%%%%%%%%%%%%%%%%%%%%%%%%%%%%%%%%%%%%%%%%%%%%%%%%%%%%%%%%%%%%%%%%%%

\begin{frame}{Logic}
%
As far as we know, among the great ancient civilizations, only the Greeks
studied the formal validity of argumentation, a.k.a., logic
(\cite{shenefelt:2013:ifathenb}).  This presentation will borrow a lot from the
lucid and readable reference \cite{hacking:2001:introduction}.

\pause
Logic studies the {\em validity} of an argument, not the {\em truth} of
its conclusions.

An argument is valid if it is logically sound.

A proposition is a statement which is either true or false.

\vspace{1em}
\textbf{Example: }\\
If James wants a job, then he will get a haircut tomorrow.\\
James will get a haircut tomorrow.\\
So: James wants a job.

\vspace{1em}
If James wants a job, then he will get a haircut tomorrow.\\
James wants a job.\\
So: James will get a haircut tomorrow.

\pause
\vspace{1em}
\textbf{Questions:} \\
Which argument is valid?  \\
What are the propositions?\\
Which propositions are true?
%
\end{frame}


\begin{frame}{Bibliography}
\printbibliography{}
\end{frame}

%%%%%%%%%%%%%%%%%%%%%%%%%%%%%%%%%%%%%%%%%%%%%%%%%%%%%%%%%%%%%%%%%%%%%%%
%%%%%%%%%%%%%%%%%%%%%%%%%%%%%%%%%%%%%%%%%%%%%%%%%%%%%%%%%%%%%%%%%%%%%%%
%%%%%%%%%%%%%%%%%%%%%%%%%%%%%%%%%%%%%%%%%%%%%%%%%%%%%%%%%%%%%%%%%%%%%%%

\begin{frame}{Logic}

Aristotle identified four ``categorical propositions'' that form the basis
of his logic:

%
\begin{itemize}
%
\item All As are Bs.
\item No As are Bs.
\item Some As are Bs.
\item No As are not Bs.
%
\end{itemize}
%
%
\end{frame}

%%%%%%%%%%%%%%%%%%%%%%%%%%%%%%%%%%%%%%%%%%%%%%%%%%%%%%%%%%%%%%%%%%%%%%%
%%%%%%%%%%%%%%%%%%%%%%%%%%%%%%%%%%%%%%%%%%%%%%%%%%%%%%%%%%%%%%%%%%%%%%%
%%%%%%%%%%%%%%%%%%%%%%%%%%%%%%%%%%%%%%%%%%%%%%%%%%%%%%%%%%%%%%%%%%%%%%%

\begin{frame}{Logic}
%
\textbf{Which of these arguments are valid?}
(\cite[Ch.1 Question 7]{hacking:2001:introduction})

%
\begin{itemize}
%
\item I follow three major league teams.  Most of their top hitters chew
tobacco at the plate.\\
$\Rightarrow$ Chewing tobacco improves batting average.
%
\item The top six hitters in the National League chew tobacco at the plate.\\
$\Rightarrow$ Chewing tobacco improves batting average.
%
\item A study by the American Dental Association of 158 players on
seven major league teams during the 1988 season showed that the mean batting
average for chewers was 0.238 compared to 0.248 for non-users.  Abstainers
also had a higher fielding average.\\
$\Rightarrow$ Chewing tobacco does not improve batting average.
%
\item In 1921, every major league pitcher who chewed tobacco when up to
bat had a higher batting average than any major league pitcher who did
not.\\
$\Rightarrow$ Chewing tobacco does not improve batting average.
%
\end{itemize}
%
\pause
\textbf{None of them are valid.}  But some are better than other.
%
\end{frame}



\end{document}
