\documentclass[8pt]{beamer}\usepackage[]{graphicx}\usepackage[]{color}

\documentclass{article}
\usepackage{fancyhdr}


\pagestyle{fancy}
\fancyhf{}
\rhead{Ryan Giordano}
\lhead{Research Statement}
\rfoot{Page \thepage}

\usepackage{tabularx}

% \usepackage{fancyhdr}
% \pagestyle{fancy}

% \fancyfoot{}
% \fancyfoot[C]{Rough draft---do not distribute}

\usepackage{etoolbox}


\usepackage{microtype}
\usepackage{graphicx}
\usepackage{subfigure}
\usepackage{booktabs} % for professional tables
\usepackage{xcolor}
\usepackage[hidelinks=True]{hyperref}
\usepackage{xargs}[2008/03/08]

% Documentation
% http://ftp.math.purdue.edu/mirrors/ctan.org/macros/latex/contrib/refstyle/refstyle.pdf
\usepackage{refstyle}
\usepackage{varioref} % Use refstyle instead of varioref directly.

\usepackage{amsmath}
\usepackage{amssymb}
\usepackage{amsfonts}
\usepackage{amsthm}
\usepackage{mathrsfs} % For mathscr
\usepackage{mathtools}

\usepackage[authoryear]{natbib}
\bibliographystyle{apalike}

\usepackage{geometry}
\geometry{margin=1.5in}


\def\red#1{\textcolor{red}{#1}}
\def\blue#1{\textcolor{blue}{#1}}
\def\green#1{\textcolor{violet}{#1}}

\def\undernote#1#2{\underbrace{#1}_{\mathclap{\substack{#2}}}}
\def\overnote#1#2{\overbrace{#1}^{\mathclap{\substack{#2}}}}

\DeclareMathOperator*{\argmax}{\mathrm{argmax}}
\DeclareMathOperator*{\argmin}{\mathrm{argmin}}
\DeclareMathOperator*{\esssup}{\mathrm{esssup}}
\DeclareMathOperator*{\essinf}{\mathrm{essinf}}
\DeclareMathOperator*{\argsup}{\mathrm{argsup}}
\DeclareMathOperator*{\arginf}{\mathrm{arginf}}

\newcommand{\fracat}[3]{\left. \frac{#1}{#2} \right|_{#3}}

\def\expect#1#2{\underset{#1}{\mathbb{E}}\left[#2\right]}
\def\sumn{\sum_{n=1}^{N}}
\def\meann{\frac{1}{N}\sumn}
\def\var#1#2{\underset{#1}{\mathrm{Var}}\left(#2\right)}
\def\kl#1#2{\mathrm{KL}\left(#1 || #2\right)}

\def\qdom{\mathcal{Q}}
\def\qstar{q^*}
\def\q{q}
\def\p{p}

\begin{document}


%%%%%%%%%%%%%%%%%%%%%%%%%%%%%%%%%%%%%%%%%%%%%%%%%%%%%%%%%%%%%%%%%%%%%%%
%%%%%%%%%%%%%%%%%%%%%%%%%%%%%%%%%%%%%%%%%%%%%%%%%%%%%%%%%%%%%%%%%%%%%%%
%%%%%%%%%%%%%%%%%%%%%%%%%%%%%%%%%%%%%%%%%%%%%%%%%%%%%%%%%%%%%%%%%%%%%%%

% \title{Black Box Variational Bayes with a Deterministic Objective}

% \subtitle{Faster, More Accurate, and Even More Black Box} 
% \date{May 25th, 2023}

% \author{Giordano, Ryan \inst{1} \and Ingram, Martin\inst{2} \and Broderick, Tamara\inst{1}}

% \institute[] % (optional)
% {
%   \inst{1}%
%   Massachusetts Institute of Technology
%   \and
%   \inst{2}%
%   University of Melbourne, Australia
% }


% \frame{\titlepage}

%%%%%%%%%%%%%%%%%%%%%%%%%%%%%%%%%%%%%%%%%%%%%%%%%%%%%%%%%%%%%%%%%%%%%%%
%%%%%%%%%%%%%%%%%%%%%%%%%%%%%%%%%%%%%%%%%%%%%%%%%%%%%%%%%%%%%%%%%%%%%%%
%%%%%%%%%%%%%%%%%%%%%%%%%%%%%%%%%%%%%%%%%%%%%%%%%%%%%%%%%%%%%%%%%%%%%%%

\begin{frame}{Discussion of ``The Shrinkage-Delinkage Trade-off''}

Variational inference (VI) finds
$q^* := \argmin_{\q \in \qdom} \kl{\q}{\p}$ for an unknown target $\p$.
%
% \begin{align*}
% %
% %
% \end{align*}
%

What should $\qdom$ be?

\pause

Classical VI takes a simple $\qdom$. Then $\p \notin \qdom$, but 
you get computational benefits!

\pause

But when $\p \notin \qdom$, can get poor posterior approximations even in simple
cases.  

What to do?

\pause
%
\begin{enumerate}
%
\item Don't care (``machine learning'')
%
\begin{itemize}
    \item Evaluate by other criteria than
    poterior approximations (e.g. prediction)
    \item Maybe fine for some machine learning tasks
\end{itemize}
%
\pause
\item Make $\qdom$ more expressive (``modern VI'')
%
\begin{itemize}
    \item Strong theoretical guarantees
    \item High computational cost!
\end{itemize}
\pause
%
\item Try to capture important properties of $\p$ with simple $\qdom$
% \item Understand how simple $\qdom$ fails for things we care about
%
\begin{itemize}
    % \item We don't actually want the whole posterior
    \item Begins with understanding how things go wrong (\textbf{this paper}!)
    \item Hope to have our cake and eat it too (e.g. marginals \emph{and} easy computation)
    \item Much harder!  But important, with big potential benefits
\end{itemize}
%
\end{enumerate}
%

I would love to see more work like this!


\end{frame}


% \begin{frame}{References}

%     \footnotesize
    
%     \bibliographystyle{plainnat}
%     % Hide the references header
%     % https://tex.stackexchange.com/questions/22645/hiding-the-title-of-the-bibliography/370784
%     \begingroup
%     \renewcommand{\section}[2]{}%
%     \bibliography{references}
%     \endgroup
    
%     %
% \end{frame}

\end{document}
