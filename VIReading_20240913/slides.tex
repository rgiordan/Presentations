
\begin{frame}{Economist 2016 Election Model \citep{economist:2020:election}}

\begin{minipage}[c]{0.45\textwidth}
    \ElectionData{}
\end{minipage}
%
\begin{minipage}[c]{0.45\textwidth}
    Model each poll as
%
\begin{align*}
    \y_i | \pi_i \sim{}& \mathrm{Binomial}(\pi_i) \\
    \mathrm{Logit }\, \pi_i ={}& \mu^b_{s[i],t[i]} + \alpha_i +  \zeta_i^{state} + \xi_{s[i]}\\
\end{align*}
%

\vspace{1em}
Pack everything we don't know into 

$$\theta \in \mathbb{R}^{15098}.$$

\vspace{1em}
If we knew $\theta$, we'd know the outcome of the election up to Binomial randomness.

\vspace{1em}
The questions is: \textbf{which values of $\theta$ are consistent with the data we saw?}

\end{minipage}
\end{frame}



\begin{frame}{Forward and inverse problems}
    \href{https://rgiordan.github.io/posts/2019-08-30-bayesian_as_inverse_problem.html}{(link to blog)}
\end{frame}


\begin{frame}{How to compute the posterior?}
\end{frame}



