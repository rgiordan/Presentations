\documentclass{article}
\usepackage{fancyhdr}


\pagestyle{fancy}
\fancyhf{}
\rhead{Ryan Giordano}
\lhead{Research Statement}
\rfoot{Page \thepage}

\usepackage{tabularx}

% \usepackage{fancyhdr}
% \pagestyle{fancy}

% \fancyfoot{}
% \fancyfoot[C]{Rough draft---do not distribute}

\usepackage{etoolbox}


\usepackage{microtype}
\usepackage{graphicx}
\usepackage{subfigure}
\usepackage{booktabs} % for professional tables
\usepackage{xcolor}
\usepackage[hidelinks=True]{hyperref}
\usepackage{xargs}[2008/03/08]

% Documentation
% http://ftp.math.purdue.edu/mirrors/ctan.org/macros/latex/contrib/refstyle/refstyle.pdf
\usepackage{refstyle}
\usepackage{varioref} % Use refstyle instead of varioref directly.

\usepackage{amsmath}
\usepackage{amssymb}
\usepackage{amsfonts}
\usepackage{amsthm}
\usepackage{mathrsfs} % For mathscr
\usepackage{mathtools}

\usepackage[authoryear]{natbib}
\bibliographystyle{apalike}

\usepackage{geometry}
\geometry{margin=1.5in}


\title{Research Statement}

\author{
  Ryan Giordano \\ \texttt{rgiordan@mit.edu }
}

\begin{document}


\section{Overview}

Data analysts ask increasingly sophisticated questions of larger and more
complex datasets, statistical models are becoming both more complex and more
time-consuming to estimate.   For problems as disparate as the production of
astronomical catalogues, analysis of internet user data, to the meta-analysis of
randomized controlled trials in development economics.

As models grow in complexity, the need to interrogate their assumptions,
to propagate uncertainty amongst their components, and to perform non-parametric
checks on their data sensitivity grows commensurately, but so does the
computational cost of doing so.

Classical procedures such as Markov Chain Monte Carlo (MCMC) or the bootstrap,
which require evaluating a model for many distinct parameter values or datasets,
respectively, can be prohibitively expensive.

My research focuses on applications of sensitivity analysis that use only
properties of a single model fit to extrapolate to alternatives without
expensive re-running.  My work is conceptually unified around a single theme,
but disparate in applications.

My guiding motivation is uncertainty quantification, providing easy-to-use,
general tools that allow analysts to reason about both aleatoric and epistemic
uncertainty in their analysis tasks.

Sensitivity analysis is traditionally construed both more broadly and more
narrowly than the methods I investigate.  Traditional sensitivity analysis
encompasses both local methods, which are based on extrapolations from
a given model fit, and global methods, which typically require fitting a model
multiple times at a variety of configurations (though exceptions exist
in cases with special structure).  Traditionally, sensitivity analysis focuses
only on assessing whether small changes in arbitrary modeling decisions
can affect the outcome.  Due to the computational expense of re-fitting
many modern statistical models, and the relative ease of computing
derivatives (especially with modern automatic differentiation tools),
I focus on local sensitivity analysis.  And a core theme of my work is that
the domain of sensitivity analysis in fact encompasses much of classical
frequentist uncertainty quantification, and can provide valuable insights
into Bayesian uncertainty quantification as well.


\section{Projects}

\subsection{Prior sensitivity in Bayesian analysis}

Bayesian analysis allows analysts to reason coherently about unknown parameters,
but only if the user specifies a complete generating process for the parameters
and data, including both prior distributions for the parameters and precise
likelihoods for the data.  Often aspects of this model are at best a considered
simplifcation, and at worst chosen only for computational convenience.  It is
critical to ask whether the analysis would have changed substantively had
different modeling choices been made.

\paragraph{Bayesian nonparametrics.}

A commonly asked question in unsupervised clustering is how many distinct
clusters are present in a dataset.  Discrete Bayesian nonparametrics allows the
question to be addressed using Bayesian inference, but one must specify a prior
on how distinct clusters are generated.  A particularly common choice is
the stick-breaking representation of a Dirichlet process prior, a mathematical
abstraction arguably better justified by its mathematical convenience than
its realism.  The prior must be specified in terms of random stick lengths
to be broken off successively, the lengths of the sticks determining
the a priori cluster sizes.  The standard approach is to model the stick
lengths with $Beta(1, \alpha)$ distribution, the $\alpha$ being a scalar
tuning parameters.  Other classes of distributions are possible in principle
but hardly ever considered in practice, in part because the Beta distribution
enjoys some computational conveniences.

In CITE, we provide sensitivity measures that allow the user to explore
alternative stick breaking distributions from a single fit using the standard
and convenient Beta prior.  We linearly approximating the dependence of the
optimum on the functional form of a parameterized class of priors.  A natural
parameterized class is the set of $Beta(1, \alpha)$ distributions (parameterized
by $\alpha$), but we also consider arbitrarily functional perturbations.
In current work in progress, we evaluate the worst-case perturbations.
On a real-world clustering problem, a human genome dataset, we find that
the number of distinct inferred populations is in fact quite sensitive
to the prior.


\paragraph{Partial pooling in meta-analysis.}

A popular form of meta-analysis is to place a hierarchial model on a set of
related experimental results, which both ``shrinks'' the individual estimates
towards a common mean, potentially decreasing mean squared error, and
allowing direct estimation of the average effect and diversity of effects.
These advantages come at the cost of positing a precise generative
process for the effects in question, however, and it is reasonable to
interrogate whether the estimation procedure is robust to varaibility
in these effects.  In CITE, we apply sensitivity analysis to a published
meta-analysis of the effectiveness of microcredit interventions in seven
developing countries.  We find that the conclusion are highly sensitive
to the assumed covaraince structure between the base level of business
profitability and the microcredit effect, a covariance which is a priori
difficult to ascertain.  In this way, we were able to easily diagnose
a conceptual problem in a model which was time-consuming to fit.


\paragraph{Hyperparameter sensitivity for MCMC.}

A classical result in Bayesian sensitivity analysis states that derivatives
of posterior expectations take the form of particular posterior covariances.
The resulting sensitivities can be automatically computed in a black-box
manner when the posterior is implemented in software that supports
automatic differentiation, such as the popular Hamiltonian Monte Carlo
sampler and modleing language, Stan.  I have written an R package CITE
that allows Stan users to specify a ``hyperparameters'' modeling block,
from which one can automatically compute hyperparameter sensitivity from
a single MCMC run with no additional computation.  I apply these principles
in a related work on frequentist variance below.


\subsection{Data sensitivity: cross validation and frequentist variance}

Frequentist variability is ultimately concerned with the value of an
estimation procedure if the data were different than that observed.  A classical
mainfestiation of this idea is the nonparametric bootstrap, which estimates
frequentist variability by evaluating a particular estimator at pseudo-datasets
with observations draw with replacement from the observed dataset.  Similarly,
cross-validation (CV) in its various forms evaluates how a statistical procedure
performs on data that were not included as part of estimation, and can be
thought of as a non-parametric estimator of the bias induced by evaluating
a loss function using the same data that were used to fit a model.

Both the bootstrap and CV require re-fitting a model with new, nearby datasets
multiple times.  When the model is differentiable, and model re-fitting is
expensive, it can be advantageous to approximate the effect of re-fitting rather
than perform actual re-fitting.  One way of doing so it so is to perform a
Taylor series expansion of the estimator, as a funciton of the empirical
distribution, around the original empirical distribution.  This is the core
concept behind the related classical tools known as the ``infinitesimal
jackknife,'' ``von-Mises Expansion,'' and ``empirical influence function,''
though until recently these differnetial approximations were used most
prominently to facilitate theoretical analysis.


\paragraph{Accuracy bounds for leave-k-out CV.}

We, and several other authors, obsevered that these differnetial methods
could speed up the evaluation of cross validation in large machine learning
models which are expensive to re-fit.  In CITE, we bridged the gap from
some of the classical literature, providing finite-sample accuracy bounds
for approximate leave-k-out cross validation, even when the derivatives of
the objective function are unbounded.


\paragraph{Higher orders, k-fold CV, and the bootstrap}

A follow-on work in progress (CITE) expands the results to higher-order
expansions and to larger perturbations, including k-fold CV and the bootstrap.
They key to all this work is a set complexity condition, in light of which it is
clear that one can provide accuracy bounds uniformly over small perturbations,
and over randomly-chosen large perturbations, but not for uniform bounds over
large perturbations.  Because we show that the linear approximation approaches
the bootstrap closer than the bootstrap approaches the truth, our work
should allow for practical differential approximations to prohibitively
expensive procedures such as the bootstrap-after-the-bootstrap.


\paragraph{Frequentist properties of Bayesian posteriors}

By combining the above approach to frequentist variance with the MCMC-based
measures of sensitivity, we are able to derive the Bayesian infinitesimal
jackknife (IJ), which can be used to compute the frequentist variability
of Bayeisan posterior means without bootstrapping or computing a maximum
a-posteriori (MAP) estimate.  Such frequentist variances are particularly
important when there is a possibility of model misspecification (in which
case the Bayesian posterior variance is not particularly meaningful),
or when the data comes from a random sample, the variability of which
is meaningful in its own right.  In CITE, we prove the consistency
of the Bayesian IJ and show its accuracy as an approximation to the bootstrap
for a larger number of examples.


\paragraph{Adversarial sensitivity for M-estimators.}

In the social sciences, it is common to run randomized trials on a particular
population to esitmate an effect that we hope generalizes to different
populations.  Classical standard errors measure the variability that one
would expect from random sampling from the same distribution as that that
gave rise to the observed data.  However, in order to generalize to radically
different contexts, one might expect the conclusions to be robust to more
adversarial perturbations.  In CITE, co-authors an I ask whether some
common econometrics analyses are robust to the removal of a small proportion
(e.g. one tenth of one percent of the data).  Again using the empirical
influence function, we provide an automated method and software package
to answer this question.

We found that some common ``gold-standard'' applied econometrics papers
can have central claims reversed by removing only a very small number of
data points, even though there is no clear evidence of misspecification.
A key theoretical takeaway of our work is that datsets with a low signal
to noise ratio (defined as the effect size over the estimator standard
deviation times $\sqrt{N}$) will always be sensitive to adversarial removal
of a very small number of datapoints.  Though the precise meaning will
depend on the context, the broad implication is that studies which
attempt to overcome low SNR with large sample sizes will be inherently
non-robust, even in the absence of misspecification or gross errors.


\subsection{Propagation of uncertainty in scalable Bayesian inference}

Complex scientific inference procedures, such as the creation of astronomical
catalogues, often exhibit uncertainty in many aspects of the model.  For
instance, in order to infer whether a handful of pixels on a telescopic image is
a dim star or a distant galaxy, one must know the distortion (aka the point
spread function) of the telescope, the lightness of the sky background, the
noise of the photoreceptors, and the identity of nearby celestial objects, all
of which quantities must themselves be inferred with some uncertainty.

Bayesian procedures coherently propagate uncertainty between all such model
quantities, but classical MCMC procedures do not scale well, and are far beyond
computational reach for astronomical catalogues.  Researchers often turn to
optimization-based mean field Variational Bayes (MFVB) procedures as a scalable
alterative to MCMC, but MFVB does not estimate posterior correlations, and is
known to underestimate marginal posterior uncertainties.
\footnote{The frequentist expectation-maximization, or EM, algorithm,
can be understood as a MFVB procedure, and the present criticism applies
to it as well.}

In CITE, I develop a method to recover accurate posterior uncertainties from
MFVB approximations without needing to fit a more complex model, or indeed to
re-fit the original model.  The idea is to exploit a duality between posterior
covariances the sensitivity of posterior means and use the sensitivity of the
MFVB approximation to infinitesimal perturbations as an estimator of the
posterior covariance.  We call the method ``linear response variational Bayes''
(LRVB) after the idea's progenitor as a method in statistical mechanics for
inferring microscopic intensive thermodynamic quantities from macroscopic
perturbations of extensive quantities.  Computing the LRVB covaraince requires
solving a linear system, which in scientific applications is often sparse and
can be solved using iterative techniques such as conjugate gradient.

We compare LRVB covariances to MCMC on a large number of real-world datasets,
including logistic regression on internet-scale data, the Cormack-Jolly-Seber
model from ecology, and hierarchical generalized linear models from the
social sciences, and demonstrated accurate posterior covariances computed
over an order of magnitude faster than MCMC.


\section{Selected Future work}

There is a lot to be done simply applying the above methodology to
applied problems in conjuction with collaborators with domain expertise.
However, in this section, I will focus instead on new methodological directions
suggested by the above work.

\paragraph{The bootstrap and the bootstrap after the bootstrap.}

Our work on the higher-order infinitesimal jackknife could be applied to other
random reweighting schemes, particularly the bootstrap.  The bootstrap is known
to have frequentist properties that are asymptotically more accurate than the
normal approximation in certain circumstances CITE, but the bootstrap requires
re-computing an estimator as many time as there are bootstrap samples. However,
a sufficiently high-order IJ estimate will approach the bootstrap estimator at a
rate faster than the bootstrap's extra accuracy, strongly suggesting that the IJ
will inherit all of the bootstrap's attractive properties at a fraction of the
computational cost.  The IJ is particularly appealing for
bootstrap-after-bootstrap procedures, which have attractive theoretical
properties but are computationally prohibitive even on medium-sized statistical
problems.  It seems plausible that the HOIJ could open up a range of bootstrap
applications that are presently out of reach.


\paragraph{Bayesian model criticism.}

Essentially all tools for checking the accuracy of Bayesian models are
frequentist in nature --- e.g. checking whether the data is likely under
draws from the prior or posterior, or evaluating its predicive performance
on a held-out dataset.  Predictive model checks, such as leave-one-out
CV are attractive, but currently have few practical implementations that
avoid multiple runs of the MCMC algorithm.  However, the possibility of
forming higher-order expansions of the Bayesian posterior as a function of
the empirical distribtuion could change that.


\paragraph{Partitioned Bayesian inference with theoretical bounds.}

Given a large, complicated problem, it is often computationally convenient to
perform inference in separate computatoinal steps, while still propagating
uncertainty from one step to another.  For example, in CITE, we first fit spline
regressions to time series of gene expression data, and then clustered the
spline fits to group together genes with similar behavior.  The Bayesian ideal,
which is the simultaneous estimation of the clusters and spline regression, was
too computationally prohibitive, and would intuitively have given a similar
result to the sequential analysis, to the extent that the cluster center priors
did not shrink the spline regressions too much.


\paragraph{Incorporating LRVB corrections into MFVB approximations.}



\paragraph{Bootstrapping simulation-based inference.}




\bibliography{references}
\bibliographystyle{plainnat}

\end{document}
