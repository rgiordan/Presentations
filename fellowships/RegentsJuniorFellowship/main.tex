\documentclass{article}
\usepackage{fancyhdr}


\pagestyle{fancy}
\fancyhf{}
\rhead{Ryan Giordano}
\lhead{Research Statement}
\rfoot{Page \thepage}

\usepackage{tabularx}

% \usepackage{fancyhdr}
% \pagestyle{fancy}

% \fancyfoot{}
% \fancyfoot[C]{Rough draft---do not distribute}

\usepackage{etoolbox}


\usepackage{microtype}
\usepackage{graphicx}
\usepackage{subfigure}
\usepackage{booktabs} % for professional tables
\usepackage{xcolor}
\usepackage[hidelinks=True]{hyperref}
\usepackage{xargs}[2008/03/08]

% Documentation
% http://ftp.math.purdue.edu/mirrors/ctan.org/macros/latex/contrib/refstyle/refstyle.pdf
\usepackage{refstyle}
\usepackage{varioref} % Use refstyle instead of varioref directly.

\usepackage{amsmath}
\usepackage{amssymb}
\usepackage{amsfonts}
\usepackage{amsthm}
\usepackage{mathrsfs} % For mathscr
\usepackage{mathtools}

\usepackage[authoryear]{natbib}
\bibliographystyle{apalike}

\usepackage{geometry}
\geometry{margin=1.5in}

\usepackage{enumitem}
\setlist{nolistsep}

\usepackage{geometry}
%\geometry{margin=1.2in}
\geometry{top=1.0in}
\geometry{left=1.1in}
\geometry{right=1.1in}

\title{Ryan Giordano Regents’ Junior Faculty Fellowships}

\author{
  Ryan Giordano \\ \texttt{rgiordano@berkeley.edu }
}

\begin{document}

\begin{minipage}[t]{0.7\textwidth}
\hspace{-2em} % Easier than doing it right!
{\bf \LARGE Regents’ Junior Faculty Fellowship}\\
\end{minipage}
\begin{minipage}[t]{0.3\textwidth}
%    \begin{flushright}
%        \hspace{8em} % Easier than doing it right!
        {\LARGE Ryan Giordano}
%    \end{flushright}
\end{minipage}

I propose to spend the summer of 2024 working on two collaborative research
projects.  The first, ``neural network classifiers for Bayesian posteriors,''
promises to introduce a completely new set of Bayesian inference techniques
based on ideas from simulation--based inference. The second, ``black--box
computable diagnostic weights for survey sampling,'' will bring a much--needed
set of diagnostic tools to the vast majority of modern applied survey sampling.
These two projects are different in scope --- the first represents
ground--breaking methodological research, and the second an application of my
existing research to an urgent applied problem --- but each rests on and
contributes to my existing work and expertise on approximate Bayesian
computation and sensitivity analysis. 





\section*{Neural network classifiers for Bayesian posteriors}

Bayesian statistical techniques are a conceptually powerful set of tools for
representing and quantifying uncertainty, and are increasingly popular across
the physical and social sciences.  Often, a statistical analysis involves a
single quantity of interest, such as the effect of a policy intervention
\citep{meager:2019:microcredit}, the type of an astronomical object
\citep{regier:2019:cataloging}, the outcome of an election
\citep{economist:2020:election}, or the identity of an ancestral genetic
population \citep{pritchard:2000:inference}.  Bayesian statistics is able to
propagate uncertainty from any unknown latent modeling quantities to the final
estimate.  But this conceptual strength is a computational weakness, since even
approximately accounting for a large number of latent quantities is
computationally intensive. Bayesian estimates often take hours to days to
compute, and it is of considerable interest to develop computationally
efficient, approximate Bayesian procedures \citep{blei:2017:variational,aabi}.

In consultation with a staff scientist at LBNL, I have recently developed a new
approach to Bayesian inference based on neural network classifiers (NNC). The
idea is derived from a technique for point estimation in simulation--based
inference (SBI),\footnote{That is, inference in problems without a tractable
likelihood function.} a technique I will refer to as SBI-NNC
\citep{cranmer:2020:frontierofsimulation}.  Rather than learning a likelihood
directly, SBI-NNC exploits the fact that optimal neural network classifiers
learn likelihood ratios. I have shown that a variant of the SBI-NNC trick can be
applied to learn Bayesian marginals without having to learn the distribution of
all the latent variables, at the cost of training a NNC on a single
classification task.  I will refer to my technique as Bayes--NNC.

Both classical Bayesian procedures and the existing SBI-NNC trick are difficult
to validate in practice, due to the lack of a computable ground truth.
Amazingly, Bayes--NNC does not suffer from this shortcoming, and its accuracy is
readily testable using simulation--based calibration (SBC)
\citep{talts:2018:sbc}.  Put together, Bayes--NNC and SBC offer a way to learn
Bayesian posterior densities of low--dimensional quantities of interest with
strong, computable statistical accuracy guarantees. Interestingly, SBC is
well--known but rarely used in practice, since it is typically computationally
prohibitive to compute the posterior at many different datapoints.  However,
Bayes--NNC learns the posterior for many datasets simultaneously, permitting
efficient use of SBC in practice.  

To my knowledge, there are no existing Bayesian techniques that offer the
advantages of Bayes-NNC and SBC.  Bayesian approaches to simulation--based
inference are not new, but existing techniques are built on high--dimensional
density approximation, such as normalizing flows
\citep{cranmer:2020:frontierofsimulation,papamakarios:2021:normalizing}.  As
with other approximate inference techniques, this set of tools approximates the
entire posterior, even when only a low--dimensional marginal is of interest.  To
the best of my (and my LBNL collaborator's) knowledge, Bayes--NNC is new, and
offers a distinct and advantageous set of computational tradeoffs relative to
existing Bayesian inference methods.  

I would stress that Bayes--NNC is not at all limited to SBI problems. There are
many problems with tractable likelihoods which are easy to simulate from and
extremely computationally intensive to sample from.  There is every reason to
believe that a training a NNC will be competitive with existing computational
techniques, even neglecting the advantage of Bayes--NNC's statistical
guarantees.  A large number of fundamental statistical applications,
superficially distinct from SBI problems, are immediate candidates for
Bayes--NNC, such as sparse model selection (e.g.
\citep{rovckova:2018:spikeandslablasso}), Gaussian process hyperparameter
calibration (e.g. \citep{hensman:2015:gpclassificationscalable}), and
unsupervised clustering (e.g. \citet{mcauliffe:2006:dpeb}).



\section*{Black--box computable diagnostic weights for survey sampling}

Most modern surveys --- such as polling about the upcoming presidental election
--- must overcome the fact that their sampled population is different from the
target population \citep{gelman:2007:surveystruggles}.  For example, the set of
people responding to an internet survey about political preferences is likely to
differ systematically from the full population of voters, and it is extremely
useful to be able to check that the re-weighting is accurate, for example by
checking that key demographic variables are balanced by the re-weighting
\citep{li:2018:balancingcovariates,benmichael:2021:multilevel}. Unfortunately,
the most accurate and most commonly used statistical procedures for inferring
the polling responses of rare demographic groups are nonlinear, and so do not
readily admit diagnostic weights
\citep{gelman:1997:poststratification,gelman:2007:surveystruggles}. 

In collaboration with a UC Berkeley professor of public policy, I have shown
that one can compute ``local diagnoistic weights'' for non--linear statistical
procedure, provided a much--needed diagnostic that is currently unavailable. The
local weights I derive are closely related to the classical ``influence
function'' of robust statistics
\citep{mises:1947:asymptotic,hampel:1986:robust,giordano:2019:swiss}.  Though
the influence function is well--studied in the frequentist literature, it has
been relatively neglected in the Bayesian literature (with my own recent work,
\citet{giordano:2023:bayesij}, being a notable exception).

Importantly, the local weights can be automatically computed with a small
library built on top of existing open--source software which is commonly used
for survey analysis \citep{lopezmartin:2022:mrptutorial}.  I have already
implemented a similar package for a different style of sensitivity analysis
\citep{giordano:2020:amip,giordano:2024:zaminfluence}, and we expect to be able
to release open--source software relatively quickly.  


% \newpage

\bibliography{references}
\bibliographystyle{plainnat}

\end{document}
