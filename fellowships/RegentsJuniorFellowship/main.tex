\documentclass{article}
\usepackage{fancyhdr}


\pagestyle{fancy}
\fancyhf{}
\rhead{Ryan Giordano}
\lhead{Research Statement}
\rfoot{Page \thepage}

\usepackage{tabularx}

% \usepackage{fancyhdr}
% \pagestyle{fancy}

% \fancyfoot{}
% \fancyfoot[C]{Rough draft---do not distribute}

\usepackage{etoolbox}


\usepackage{microtype}
\usepackage{graphicx}
\usepackage{subfigure}
\usepackage{booktabs} % for professional tables
\usepackage{xcolor}
\usepackage[hidelinks=True]{hyperref}
\usepackage{xargs}[2008/03/08]

% Documentation
% http://ftp.math.purdue.edu/mirrors/ctan.org/macros/latex/contrib/refstyle/refstyle.pdf
\usepackage{refstyle}
\usepackage{varioref} % Use refstyle instead of varioref directly.

\usepackage{amsmath}
\usepackage{amssymb}
\usepackage{amsfonts}
\usepackage{amsthm}
\usepackage{mathrsfs} % For mathscr
\usepackage{mathtools}

\usepackage[authoryear]{natbib}
\bibliographystyle{apalike}

\usepackage{geometry}
\geometry{margin=1.5in}

\usepackage{enumitem}
\setlist{nolistsep}

\usepackage{geometry}
%\geometry{margin=1.2in}
\geometry{top=1.0in}
\geometry{left=1.1in}
\geometry{right=1.1in}

\title{Ryan Giordano Regents’ Junior Faculty Fellowships}

\author{
  Ryan Giordano \\ \texttt{rgiordano@berkeley.edu }
}

\begin{document}

\begin{minipage}[t]{0.7\textwidth}
\hspace{-2em} % Easier than doing it right!
{\bf \LARGE Regents’ Junior Faculty Fellowship}\\
\end{minipage}
\begin{minipage}[t]{0.3\textwidth}
%    \begin{flushright}
%        \hspace{8em} % Easier than doing it right!
        {\LARGE Ryan Giordano}
%    \end{flushright}
\end{minipage}

I propose to spend the summer of 2024 working on two collaborative research
projects.  The first, ``neural network classifiers for Bayesian posteriors,''
promises to introduce a completely new set of Bayesian inference techniques with
different computational tradeoffs than existing methods. The second,
``black--box computable diagnostic weights for survey sampling,'' will bring a
much--needed set of diagnostic tools to the vast majority of modern applied
survey sampling.  These two projects are different in scope --- the first
represents ground--breaking methodological research, and the second an
application of my existing research to an urgent applied problem --- but each
rests on and contributes to my existing work on approximate Bayesian computation
and sensitivity analysis. 





\section*{Neural network classifiers for Bayesian posteriors}

Bayesian statistical techniques are a conceptually powerful set of tools for
representing and quantifying uncertainty, and are increasingly popular across
the physical and social sciences (CITE).  Often, a statistical analysis involves
a single quantity of interest, such as the effect of a policy intervention
(CITE), the type of an astronomical object (CITE), the outcome of an election
(CITE), or the identity of an ancestral genetic population (CITE), and Bayesian
statistics abile to propagate uncertainty from a any unknown latent modeling
quantities to the final estimate.  But this conceptual strength is a
computational weakness, since even approximately accounting for a large number
of latent quantities is computationally intensive.  Bayesian estimates often
take hours to days to compute, and it is of considerable interest to develop
computationally efficient, approximate Bayesian procedures.

I have recently shown that a slight modification of the preceding procedure can
use neural network (NN) classifiers to learn low-dimensional marginals of
Bayesian posteriors using only simulated data.  The idea is based on the
estimation of log--likelihood ratios in simulation--based inference.
Furthermore, the accuracy of the approximateion can be easily checked using
simulation--based calibration (SBC), a well--known Bayesian validation
procedure.  Note that SBC is rarely used in practice, since it is prohibitively
computationally expensive in most   classical Bayesian procedures.  However, for
a posterior approximation based on a NN classifier, SBC is computationally
cheap.  For the cost of training a NN classifier, one can get closed-form
estimates of Bayesian posterior marginals with strong, computable statistical
accuracy guarantees.

To my knowledge, there are no existing techniques that offer the advantages of
my proposed method. There has been recent interest in statistical procedures
that improve computability by accounting only approximately for the dependence
between large numbers of latent variables.  However, these approximations come
without accuracy guarantees, and are known to be inaccurate in certain practical
cases. The vast majority of existing research in approximate Bayesian
computation attempts to model the entire high--dimensional distribution, even
when only one variable is of interest.

Though superficially distinct, recent work in simulation--based inference points
towards a completely new set of Bayesian inference techniques for
low--dimensional marginals.  Simulation--based inference is developed for
problems which can be simulated, but for which no likelihood is available.  By a
clever construction of simualted data, researchers are able to estimate log
likelihood ratios using a neural network classifier (CITE).  These likelihood
ratios are then used to compute maximum likelihood (points) estimates, rather
than full distributions, and it is difficult to assess whether the likelihood
ratios are accurate.


Note that Bayesian approaches to simulation--based inference are not new, but
existing techniques are built on high--dimensional density approximation, such
as normalizing flows.  As with other approximate inference techniques, this set
of tools approximates the entire posterior, even when only a low--dimensional
marginal is of interest.  To the best of my and my collaborator's knowledge, the
technique above is new.  

A large number of applications are immediate candidates for the above ideas.



\section*{Black--box computable diagnostic weights for survey sampling}

Most modern surveys --- such as polling about the upcoming presidental election
--- must overcome the fact that their sampled population is different from the
target population.  For example, the set of people responding to an internet
survey about political preferences is likely to differ systematically from the
full population of voters.  As a consequence, survey responses are typically
reweighted by key demographic variables in order to become more representative.
As a diagnostic, it is very useful to be able to check that these weights are
able to ``balance'' demographic quantities.  Unfortunately, the best and most
commonly used statistical procedures for inferring the polling responses of rare
demographic groups are nonlinear, and so do not readily admit diagnostic
weights.

In order to provide diagnostic weights for such non--linear procedures, I
propose forming local weights using a variant of the ``influence function'' of
classical statistics.  With my collaborators, I have shown that local weights
can automatically detect demographic imbalance in the implicit non--linear
statistical procedure, provided a much--needed diagnostic that is currently
unavailable.  Though the influence function is well--studied in the frequentist
literature, it has been relatively neglected in the Bayesian literature until my
recent work.

Importantly, the local weights can be automatically computed with a 
small library built on top of existing open--source software which
is commonly used for survey analysis (CITE).  I have already implemented
a similar package for a different sensitivity analysis (CITE).








\newpage

\bibliography{references}
\bibliographystyle{plainnat}

\end{document}
